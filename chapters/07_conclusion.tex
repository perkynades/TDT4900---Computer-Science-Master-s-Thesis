\chapter{Conclusion} \label{sec:conclusion}
As the Norwegian government is moving the Norwegian public sector to offer connected social services through a shared digital platform. A range of non-functional functional requirements becomes shared between agencies in the Norwegian public sector in order to maintain the trust of Norwegian citizens. Leading to what can be suspected as a shared view on quality assurance between the agencies in the Norwegian public sector. The goal of this study has therefore been to understand through a case study, how software quality assurance is being practised, and what challenges are encountered in software quality assurance in the Norwegian public sector.

The case study shows that several agencies have moved away from the traditional ISO mindset, instead focusing on feedback from their users. Leading to the agencies implementing methodologies such as DevOps and Agile to assure quality in software through user feedback. The assurance of technical quality in software is also shown important for the agencies in this study, as practices such as inspection, measurements and testing are used to ensure high technical quality of the software. The challenges revealed in software quality assurance were shown to not be directly connected with the practices, but rather organisational factors relating to the context of the Norwegian public sector. Of being lack of resources and budgeting, pushing the agencies to use the criticised project wizard delivered by the Norwegian digitisation agency. In combination with legal requirements that leave little room for assurance of aspects like usability.

When comparing the software quality assurance practices of the agencies in this study with the scientific literature, it can seem that the practices are what is recommended. However, being in the context of the public sector is the main challenge for producing software of high quality. And these challenges need to be addressed for the Norwegian public sector to be able to create software of high quality in order to better serve the Norwegian public. And these challenges must be addressed before the Norwegian government can move forwards to deliver shared social services through a shared digital platform.

The findings of this study do have some limitations. As only 3 of the many agencies in the Norwegian public sector were included, and a few amounts of employees at each agency were interviewed. As well as the research role as a software developer in the Norwegian public sector could lead to bias in favour of the agencies, especially NAV, and in disservice to the Norwegian government. The lack of scientific literature in the context of the Norwegian public sector could also lead to validity challenges for the findings of this study.

The results of this study should be used by agencies in either the Norwegian or any other public sector that is struggling with software quality assurance. As the practices discovered in this study can be useful. While the challenges revealed are recommended to be further studied to uncover methods to solve the challenges and reveal the potential benefits of these challenges being solved.