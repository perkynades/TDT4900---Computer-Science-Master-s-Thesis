\chapter{Discussion}
As mentioned by multiple participants in the interviews, it is important to consider the context of the software, when assuring quality in that software. This could be seen as important for each agency in the Norwegian public sector, and the whole sector itself. Some aspects regarding software quality and its assurance such as DevOps and product teams in the Norwegian public sector are true for the whole software development industry. While other factors such as allocation of resources and it's effect on how software is developed and it's quality seems to be a somewhat unique for the Norwegian public sector.

\textcolor{red}{Kanskje viktig å nevne at definisjonen av programvare kvalitet er veldig kompleks, og det å sikre god kvalitet har vist seg å være enda mer komplekst}

\textcolor{red}{When striving for high software quality, there are several challenges in the process of assuring software quality. (I) Software requirement challenges, (II) Stakeholders perspective challenges, and (III) General challenges \cite{sh_2018}. Software requirement challenges concerns challenges related to the project itself and effects of software quality which are not visible until after the projects implementation or maintenance period \cite{sh_2018}. Stakeholder perspectives challenges concerns the challenges which the different stakeholders might have when asserting quality in software \cite{sh_2018}. General challenges concerns challenges in assuring quality in software which are common for all organisations involved in creating and delivering software \cite{sh_2018}.}

\textcolor{red}{Snakke om hvordan problemene til norsk offentlig sektor passer inn her? Skal jeg gidde? SE OVER}

\section{Practices}
A non-functional requirement often used as a measurement of quality in software, is security. The results show that security is of high importance to the agencies in the Norwegian public sector, as their software and data are of high value. This has lead to the agencies using multiple different methods and ways to organise, in order to ensure that their software are secure. It is also shown that having a strong focus on security before a piece of software is delivered can be beneficial to the Norwegian public sector. As weak security in software can have large consequences and costs to fix if discovered after it is delivered to the user \cite{csw_2011}. It could be argued that in the context of the Norwegian public sector, weak security can lead to mistrust by the Norwegian citizens. As their trust might be challenged if a Norwegian citizen discover their sensitive data to be lost and shared online.

Different code inspection methods is described to be used by the agencies to find faults in their software. One manual method, being code review, is described as taking long time and does not find any critical faults. While other automatic methods such as SonarQube, is said to have found critical fault in Mattilsynets system, "Mats". If the Norwegian public sector want to use code inspection methods to assure quality in their software, the specific method has to be chosen with care. As some methods might steal valuable time, while not returning to much value in software quality. It can therefore seem that if using code inspection, automation is recommended, as both Mattilsynet and scientific literature has show it to be in importance of increasing software quality \cite{smm_2018}.

Code inspection tools could also be avoided all together by implementing certain programming techniques, such as pair- and mob-programming. As these techniques also acts as code inspection methods, with several extra benefits, being increased ownership and knowledge in the software, described as increasing software quality. Implementing par- and mob-programming can also be of importance to the Norwegian public sector, as their software is said to be large and more complex than in the private sector. Making ownership and knowledge of the software being developed important.

Automatic testing is shown to be an important tool in the assurance of software quality in the agencies of the Norwegian public sector. As the agencies has moved away from projects with a set deadline, a change from acceptance tests to automatic tests were made. This is important as automatic testing is described as an important tool in validating the quality of a piece of software when it is subject to frequent mutations. High test coverage with automatic tests is therefore used as a safety net when performing frequent mutations to a software. It is also emphasised that automatic testing is needed, however manual tests are still important for a piece of software to achieve high quality \cite{dsc_2019}. Therefore the agencies in the Norwegian in public sector cannot fully rely on automatic tests as the primary testing method for achieving high software quality.

The ability to register technical debt is described as important, as technical debt was described as the definition of low software quality. It would seem like there is a high degree of technical debt in the Norwegian public sector. Such as with Mattilsynets system "Mats", where its technical debt was described to cost Mattilsynet about 10 Billion NOK over 15 years. The agencies might not have the resources the fix all technical debt, however by registering it, certain measures can be implemented in order to at least minimise the risks of the technical debt.

Measuring the systems in the Norwegian public sector in certain aspects, such as downtime and faults, can create valuable statistics that can help increase software quality. These statistics can be used to compare if the quality of a piece of software has increased or decreased over a longer period. Allowing the development teams to do a more qualitative analysis of the state of their software. Which could improve software quality by having data on when and where in their software is of low quality, making it easier for the development team to improve that particular point of low software quality.

Software which implement certain properties are described as being of higher quality. Such as Mattilsynets software using an API-first approach with \gls{open_api} and \gls{publish_subscribe}. With Mattilsynets explaining is going to be enforced in future software to ensure reuse and high software quality. It could therefore be of interested that the rest of the Norwegian public sector follow such enforcement in software properties to ensure high software quality. However certain enforcement could challenge the autonomy of the product teams, as the software is forced to include technology not fit for the teams domain of the software being developed. Then resulting in lower software quality, rather than higher as first intended.

%\textcolor{red}{en av de mest vanlige ikke-funksjonelle kravene som definerer kvalitet er sikkerhet -> sikkerhet er viktig blandt etatene i norsk offentlig sektor, programvaren og dataen den holder på er av stor verdi -> de bruker flere forskjellige metdoer og organiseringer for å sikre seg at programvaren er sikker -> This also implies to security, as weak security of software can have larger consequences and costs to fix if discovered after the software delivered to the customer \cite{csw_2011}}

%\textcolor{red}{etater har forskjellige metoder for å inspisere koden for å finne ut at den ikke har feil -> noen av metodene som code review, en manuell prosses, er skrevet som å ta lang tid, og ikke finner særlig kritiske feil, mens andre metdoer, som SonarQube er automatisk og har funnet kritiske feil i "Mats" hos mattilsynet -> viktig å velge verktøy for kode inspeksjon nøye, ettersom noen kan stjele mye tid utenom å faktisk gi så mye verdi tilabke. Automation is stated as an important factor in increasing software quality \cite{smm_2018}}

%\textcolor{red}{automatisk testing blir brukt for å sikre kvalitet blant etatene -> når man har gått borti ifra prosjekter med sluttdato, så har man byttet fra accpetance tester til automatiske tester -> automatiske tester er viktige for å validere at programvaren er av kvalitet når man gjør hyppige endringer -> derfor viktig at hvis eteter ønsker å gjøre mer hyppige endringer, så må de ha god test coverage som et sikkerhetsnett -> It is emphasised that continuous testing is not the same as test automation, and manual testing is still needed to achieve quality \cite{dsc_2019}.}

%\textcolor{red}{viktigheten av å kunne registrere teknisk gjeld er viktig -> teknisk gjeld er beskrevet som det motsatte av god kvalitet -> virker som det er en del teknisk gjeld i offentlig sektor, som man nødvendigvis ikke får gjort så mye mer, som med "Mats" som er beskrevet til å koste mattilsynet 10 milliarder NOK over 15 år -> man har ikke nødvendigvis tid å ressurser til å fikse det der og da, men siden det er registrert kan man gjøre visse tilltak for å minimere risikioen av teknisk gjeld}

%\textcolor{red}{viktig å kunne måle systeme sine, som downtime, feil osv -> da kan man sammenligne statistikken for å se om kvaliteten har økt gjennom en lengre periode -> lar teamene gjøre en mer kvantitativ måling på kvalitet i programvaren -> kan gjøre det lettere å fikse dårlig kvalitet siden man har mer data på hva og hvor som er dårlig med kvaliteten}

%\textcolor{red}{for å slippe å styre med CodeQL og code reviews for å forsikre at koden er av kvalitet, så kan man kjøre par- og mob-programmering istedenfor -> da får man også med alle godene som følger med, som økt eierskap og kunnskap til koden, som forhindrer feil -> kan være viktigere i offentlig sektor siden prosjektene kan være større og mer komplekse enn i privat sektor, som gjør eierskap og kunnskap om koden enda viktigere}

%\textcolor{red}{programvare med spesielle typer egenskaper er sagt til å være av høyere kvalitet, slik som mattylsynet gjør med OpenAPI, API-first utvikling, og publis-subscribe datastrømmer -> mattilsynet skal begynne å "kreve" at andre team må gjøre det samme for å sikre gjenbruk og at all programvare er av høy kvalitet -> kanskje dette er noe av de andre etatene også skal gjøre for å sikre høy kvalitet -> kan kanskje skape litt problemer, siden autonomitet er en viktig del av produkt team, og de blir kanskje presset på noe som ikke passer det domenet og det produkt teamet bygger, som da gir lavere kvalitet}

\section{Development Teams}
In the Norwegian public sector, it is described that the area of responsibility for software quality in a development team has changed. Changing from a single person, to the entire development sharing the responsibility. This could be due to the recent change of organising teams into autonomous product teams, and the removal of centralised check-lists. This could have lead to an increase in software quality due to the teams now being able to choose themselves what quality assurance methods to be used. Giving the team a higher feeling of ownership in the software.

Cross-functional teams are described as important in creating software of high quality. Something which could be important for the Norwegian public sector, as their projects are described to be large and complex. As well as having a user base which span the entire Norwegian demographic. Leading to the projects having a range of aspects involved in their success, needing multiple professionals in different fields. A lack of this has already is already visible in the Healthplatform, which did not include UX-designers in its development, leading to its usability being low. 

The product teams at Mattilsynet are describe as having a high degree of autonomy, however without alignment, the benefits of autonomous product teams can decrease. By not having alignment in areas like technology, a wide range of technology can lead to be used without much reasoning, creating technical debt. A lack of alignment can also lead to the wrong problem being solved by the product teams, which as been described as more important than good software quality. As there is no point in creating software of high quality, if it does not solve the correct problem, the problem has no reason in its function. Only wasting the already limited time and resources in the Norwegian public sector. 

%\textcolor{red}{det vises at hvem som har ansvaret for å sikre kvalitet ligger felles blant et helt team -> noe som kan ha endret seg siden teamene er organisert i produkt team i autonomitet -> gått bort i fra sentraliserte sjekklister -> dette kan gi økt kvalitet fordi hvert team får bestemme selv hvordan de får utikle, noe som kan gi mer eierskap til programvaren de lager}

%\textcolor{red}{Tverrfaglige team  for bedre kvalitet -> er viktig i offentlig sektor siden prosjektene kan være større og mer komplekse -> prosjektene har flere aspekter og dermed trenger flere ekspertiser på plass -> dette så man med Helseplatformen der man ikke hadde UX-designere, så man visste at usability kvaliteten ville være dårlig}

%\textcolor{red}{produkt teamene hos mattilsynet har full autonomitet, men det kan fortsatt være nødvendig å ha alignment i hvordan det ønskas at de skal jobbe -> ved å ikke ha alignment på teknologi så kan de bli veldig mye forskjellig som blir brukt, noe som skaper teknisk gjeld, det motsatte av god kvalitet -> uten alignment så kan også meningen med produkt team gå bort, at de løser det riktige problemet, noe som er beskrevet som enda viktigere en god kvalitet -> det er ikke vits å lage noe som har kjempehøy kvalitet, hvis det ikke løser riktig problem, da har du sløst bort tid og ressurser}

\section{Methodology}
The results from the study show that the agencies interviewed have a different perspective on software quality, and its assurance, than the traditional ISO standards which focus on non-functional requirements. Instead describing software quality as how easy it is to mutate a piece of software based on new feedback from the user. And if the user find that the software is of quality and of value to the user, the agencies think their software is of high quality. Software quality being user-centred, not measurable non-functional requirements.

As a method to gain better and more frequent feedback from it's users, DevOps has been implemented as a core software development methodology for the agencies in this study. This as the development teams are able to deliver small changes frequently to the user. This decreases the scope of which parts of the software the user has feedback about, and any eventual faults are limited. The development teams can therefore use the same methods to quickly fix their software based on the feedback, and quickly get new feedback from the user. The use of DevOps has been studied and by other researchers and found to assure quality due to its ability to continuously deliver new updates to the user \cite{am_2020}\cite{smm_2018}\cite{ml_2022}.

Sharing of knowledge and increasing the level of knowledge in an organisation has been described as being important in assuring quality in software. With recent research describing that the use of DevOps being an important method for assuring quality through increased knowledge sharing between employees \cite{smm_2018}\cite{mm_2021}. However none of the agencies described DevOps as a method for increasing knowledge in their organisation. Rather using more traditional methods, such as participating in conferences or leaders of the development teams meeting to share experiences. DevOps could therefore have unknown effects on knowledge sharing and it's increase of software quality.

It could therefore be seen as quite tempting to introduce DevOps as a measure due to its affect on software quality, and it's popularity among agencies in the Norwegian Public Sector. However recent research showing that DevOps can hinder software quality assurance if not implemented in the organisation correctly \cite{dsc_2019}. In order to DevOps to increase quality, the development team have to take a shared responsibility for quality assurance. Rather than a single person having all the responsibility for quality assurance \cite{dsc_2019}. This shared responsibility also extends through the software's life, from development to its phase out \cite{dsc_2019}. If more agencies in the Norwegian public sector want to introduce DevOps correctly, it needs long-lasting product teams which share responsibility, or else the software quality can suffer.

%\textcolor{red}{The most recent factor for assuring quality in software is the implementation of DevOps in an organisation. This due to the ability to combine continuous integration with automatic testing \cite{am_2020}\cite{smm_2018}\cite{ml_2022}.}

%\textcolor{red}{DevOps also help with assuring quality as it increases co-operation and knowledge sharing between employees, leading to an overall improvement of a "quality culture" in the organisation \cite{smm_2018}\cite{mm_2021}.}

%\textcolor{red}{det å introdusere DevOps kan virke fristende siden de fleste gjør det, men det kan være fallgruver --> It is also shown that introducing DevOps to an organisation can hinder the assurance of software quality if not addressed to properly \cite{dsc_2019}. The product teams have to rethink their roles and responsibilities towards quality assurance, by shifting from specific roles to raising the quality assurance competence of all members in the product team \cite{dsc_2019}. The product teams have to take an end-to-end responsibility for the content being produced, meaning that the entire product team must be involved in the quality assurance of the content in its entire life \cite{dsc_2019}.}

\textcolor{red}{det er folk som mener at effectene av DevOps ikke har blitt undersøkt fult -> However it was stated that the effects of testing in DevOps environments had not been studied systematical by scientific literature \cite{dsc_2019}\cite{ja_2016} -> kan være en ide å ta effektene som prates om av DevOps med en klype salt -> kan heller være et organisatorisk fokus på hyppighet, ikke nødvendigvis DevOps}

\section{Financing and Resources}
All agencies in this study described their main challenge when ensuring that their software is of high quality, is the lack of IT-resources available. It was described at NAV that it did not have the necessary resources to deal with the \gls{technical_debt} in their payment system. If this system fails due to this, it's negative consequences could go beyond the specific citizen not receiving their payment. But the Norwegian society and economy as a whole being negatively effected. 

It is estimated that in 2018, NAV constituted for 15.5\% of Norway's gross domestic product \cite{nav_ytelsene_frem_mot_2060_2019}, totalling to about 434 Billion NOK in 2018 \cite{faktaark_finansdepartementet_2020}. This means that the Norwegian economy have a heavy reliance on NAV's payment system, and any amount of downtime can have negative consequences for the Norwegian economy and in turn, the Norwegian society. It is therefore worrying that agencies in the Norwegian public sector are not given enough resources to fix nation critical systems. The situation for such systems might be worse than expected, due to the competency for fixing such system disappearing. This could mean that the whole system needs to be replaced with a new system, as it's faults were not fixed while the necessary competence were available.

An agency might have enough resources at their disposal to fix such faults, but the yearly budgeting in the Norwegian public sector might prevent the resources being used. However yearly budgeting in the Norwegian public sector is also a challenge for new software being developed. It is described that the yearly budged is limited, hence software quality being prioritised lower. Which could be the reason why agencies such as NAV end up with software that has issues later on in its lifetime. An issue which could be prevented by the agencies receiving enough fund to implement high software quality as the software is being developed.

It is described that some agencies want to have the same IT-capabilities like NAV and Skatteetaten, with a large in-house development environment and product-teams. Yet do not have enough staff or resources to achieve this vision. Thus resulting in requirement specifications being sent to external vendors to create the software. Using external vendors or consultants is described as something which could cost up to three times more than creating it in-house. Removing even more resources which could be used on quality assurance. And since the agency already have a limited budged, might mean lowering it's expectations on software quality, such as usability.

The use of external vendors who lack the domain knowledge that is present in an in-house development environment, could also lead to lower quality in the software delivered. The external vendor might be able to deliver a piece of software that delivers it's legal requirements based on the requirement specification. But without the domain knowledge on how the agencies users will interact with the software and how to will function in the agency. This could be the case with the health platform, or the Photo app for Mattilsynet. The external vendors delivered on what was obligatory in the contract, but did not deliver well in regards to the domain the software would exist in. Leading to delivering software of lower quality than expected, or software which is outdated when delivered, such as NAV's system for delivering pension benefits. Simply because NAV's domain had changed during it's long development process.

With yearly budgets described as being limited and having software created by expensive external vendors. It has forced most agencies to request funding for their software development projects, from the project wizard delivered by the Norwegian Digitisation agency. The project wizard is described by the Norwegian government to ensuring that the Norwegian tax payer's money is spent correctly and leads to something of high quality being produced. However the perception from the participants, was unanimous on the project wizard leading to the opposite. Software which were of low quality and costing more for the Norwegian tax payers, due to it's requirements for receiving the funding.

Being able to mutate and easily maintain a piece of software was described as important for software to be of high quality. The main criticism with the project wizard is described as being structured similarly to the waterfall model. And therefore making it more difficult to mutate a piece of software later in the development project. This challenge in mutability is also prohibiting new knowledge to be implemented in the software. As the project wizard is described to be hostile towards scope creep, when in reality it is new knowledge about the problem being solved. This leading to faults in the system needing to be fixed after it has been delivered, resulting in the faults being more expensive to fix than in the development phase \cite{sh_2018}\cite{csw_2011}.

 Challenges with mutability when using the project wizard increases in combination with changing domains for the agencies. This as using the project wizard can mean it takes longer before any value is delivered to the end-user, as with NAV's system for delivering pension benefits. And as the scale of the projects accepted by the project wizard are larger than usual software projects, as described by one of the expert participants. Their importance for the Norwegian society could be argued as being larger than other development projects in the Norwegian public sector. Thus the project wizard hurting itself, by resulting in software that are not fully relevant for the Norwegian tax payers, as the tax payers money not being spent correctly.

The project wizard are also in contrast to how the different agencies in the study has organised their development teams in order to create software of high quality. It is described as a piece of software never is complete, and if not constantly maintained its functionality will whither due to \gls{technical_debt}. Resulting in the agencies organising their teams as product teams, where the same teams developing the software, are it's owner and maintains it in perpetuity. This to ensure that the people who have the most knowledge in the software, are responsible for it operating and delivering value to the end-user as expected. However when developing with funds from the project wizard, and development is complete, the project is complete, and funding for further operations is reduced or not existing. This reduction of funds for operations leads to functionality withering, as with Mattilsynets systems "Mats", leading to a reduction in quality in relation to user expectations.

These challenges relating to the use of the project wizard due to insufficient funding has also been described as a challenge of prioritisation, not insufficient funding. It being a challenge to priorities either improving the quality of existing software, or creating new features by either extending existing software, or creating new software. And the agencies who do not have the resources to sufficiently prioritise both, should improve at choosing a single prioritisation, instead of multiple. This to prevent having to use the project wizard, which might create software of lower quality. Continuing the issue of not being able to prioritise correctly, creating something which can be described as an "evil circle" for the development and maintenance of software in the agencies.

With all the challenges described by using the project wizard, it will be interesting to follow Mattilsynets progress in developing and maintaining their IT-systems. As Mattilsynet explained that they have stopped using the project wizard for funding their projects, and instead moving to product teams. This should lead to Mattilsynet being able to create new software of higher quality than with the project wizard, if believed by the participants of this study.

It is still important to recognise even with it's alleged challenges, the project wizard has provided opportunities for the agencies in the Norwegian public sector. It has given, to some extent, the necessary funding for the agencies to create new software that servers the Norwegian citizens. The heavy reliance NAV has had on the project wizard enabled them to build up a large in-house development environment, as the funding spent on expensive external consultants was replaced with an in-house development environment. By replacing a single expensive consultants, NAV was able to hire 2-3 in-house developers. Without this, NAV would might not have the necessary resources to implemented methods such as DevOps and organising into product teams. Which are described by NAV as important factors for assuring high quality in their software.

The presence of expensive consultants are still strong in the Norwegian public sector, due to the lack of staff with IT-knowledge. However these consultants have now been incorporated into the product teams owned by the agencies as knowledge resources. By making the consultants a part of the in-house development, instead of separating them from the organisation, has allowed the agencies to maintain the software quality standards of in-house development. In addition gaining the advantages of the increased IT-capabilities present from the consultants.

\section{Legal requirements}
No research mentioned legal requirements for a piece of software as an important factor to influence the quality of the software. However it was described as a challenge for delivering software of high quality to the Norwegian citizens, for multiple of the agencies in the study. These challenges were both describes as stemming from the legal requirements imposed by the Norwegian government, and other laws such as GDPR.

For a piece of software to be of good quality in the context of the Norwegian public sector, it is described as being able to adhere to legal requirements. This includes the ability to change its software to new or updated legal requirements. This is something which Skatteetaten and NAV is described as being able to, however the Norwegian Police Service is described as not being able to. The reasoning for why the Norwegian Police Service not being able to maintain their software in relation to legal requirements were not provided. But by the definition of high software quality in the Norwegian public sector, it could seem that the software of the Norwegian Police Service is lower than other agencies.

Skatteetaten seems to be the agency in this study, where its direction is software development are most driven by new policies and laws decided by the Norwegian government. And describing that Skatteetaten therefore only have the resources to implement these legal requirements, without much left for other quality aspects such as usability. And some processes at Skatteetaten not having enough resources to be automated by software. It would therefore seem that the Norwegian government are not allocating enough resources other than the minimum when deciding on new laws and policies. Ending up in the agencies themselves not being able to create software with the desired quality and fulfilling needs that the user would expect from that software.

Laws such as GDPR are described as preventing agencies to deliver high quality software to the Norwegian citizens. Such as the case with Skatteetaten and the Brønnoysund Register centre described in \autoref{sec:legal}. Where both agencies could combine the user data in one interface, providing software of better usability than existing solutions. However in this case laws on data-sharing blocked this, even tough the agency's would not use each others data about the user internally and the user already owning the data they were displaying. It was only displayed in a new format, in a single interface instead of two separate interfaces. It would therefore seem like the laws for data sharing which is set to defend the Norwegian citizens might end up hurting the citizens. Due to it preventing the development of higher quality software than existing software. And helping the user gain better information in the Norwegian public sector, due to increased usability.

It could therefore be seen that no matter how much the Norwegian government wish to improve the digital services provided by the agencies. The laws and policies which lay in their foundation have to be digitisation friendly. Thus making further efforts on digitising the laws and policies even more challenging and resource intensive.

These reflections on software quality from the participants can seem to contract somewhat. It is said that for a piece of software in the Norwegian public sector to be of high quality, it needs to adhere to legal requirements. However also stating that the same legal requirements are creating challenges to provide software which is of high quality. This contradicting views on high software quality can create false expectations of the software, leading to potential resources wasted, which could go into improving the quality of other software. Reducing the problem of not having enough resources to meed the unclear standards of the software.

\section{Cooperation Between Agencies}
\textcolor{red}{virker som etatene fortsat har mer enn nok å tenke på selv når det kommer til å levere tjenestene sine, som er en viktig definisjon for at noe er av kvalitet -> hvis de da i tillegg skal begynne å sammarbeide kan de fort miste fokuset på å levere kvalitet til sine egne brukere i sin egen etat}

\textcolor{red}{dette målet er også enda mer uåpnålig med tanke på at det knapt er ressurser til hva som er "legaly required" -> dermed hvis etatene faktisk skal opnå det de ønsker med et teknisk fellesarena for samarbeid -> enten så må de får mer ressurser, eller så må etatene få fritak fra andre legal requirements til å gjennomføre ønsket -> dette kan gi dårligere verdi til brukerne til etatene, som gir dårligere kvalitet}

\textcolor{red}{var verken forskning eller noen av de som ble intervjuet som nevnte at det at de øker samarbeidet gjennom tekniske løsninger nødvendigvis gir mer verdi for brukeren -> er dette et faktisk behov som har kommet fra reele problemstillinger, eller er det ønsketekning om hva man tror er bra, fra noen som ikke sitter nærme nok brukerne.}

\textcolor{red}{er det hensiksmessig at man skal sammarbeide tettere, vil det faktisk gi verdi til brukeren? -> etatene har gått bort fra store applikasjoner som dekker større behov, til mindre og mindre applikasjoner som dekker et spisset behov -> noe som de tror har økt verdien for brukeren, og dermed kvaliteten på det de lager -> det kan da bli sett på som å gå tilbake i tid ved å lage større fellesarenaer for sammarbeid, teknisk sett, uten den samme fleksibiliteten som man har i dag}

\textcolor{red}{en platform som er lik en slik plattform som kanskje regjeringen ønsker er NAIS -> men her er det viktig å huske at denne er laget spesifikt for NAV applikasjoner og NAV's driftsmiljø, altså NAV sitt drift domene -> hvis en slik felles platform som ønsker skal kunne være vellykket, og levere kvalitet, så må den kunne levere så spesifikt som NAIS gjør for NAV -> hvem skal passe på at dette skjer? hvem har god nok domene-kunnskap over alle Norges etater til at dette er mulig? og hvem skal vedlikeholde og drifte dette? -> DigDir kan jo være en kandidat, men spm'ålet er om de har nok kapasitet til å først og fremst bygge dette, samt vedlikeholde og endre det for alle endringene i Norsk offentlig sektor -> som f.eks endring i struktur av etater, etater opprettes og legges ned, nye lover for etetaene blir satt og fjernet -> samle sammen alt dette og en dose god gammeldags politikk så har du en fin suppe med problemer som må fikses for at det kan skje}

\textcolor{red}{når denne rapporten har blitt skrevet så er tidsrammen på digitaliseringsmålene fra regjeringen gått ut (2018-2023) -> ser ikke ut som de har klart det i løpet av den tiden -> blir spennende å se om regjeringen kommer til å fortsette mtp probleme nevnt over, eller om de kanskje jekker litt ned ambisjonsnivået og splitte opp problemene de prøver å løse -> det er det etatene prøver å gjøre for å skape programvare av kvalitet, splitte opp problemene og løse de iterativt, istedenfor alt i en gjafs}

\textcolor{red}{DevOps is expected to bring the different agencies in the public sector closer, increasing knowledge culture and collaborative work \cite{mm_2021}, which is shown to increase software quality \cite{smm_2018}.}


\section{Researchers Perspective}
\textcolor{red}{selv om "utviklings-muskelen" til NAV virker stor og sterk, så er det fortsatt vanskelig å få ressurser -> når man ikke da har nok ressurser i teamet så blir det mest fokus på det som er av "legal requirements" og kjernebehov for brukeren -> ender med at ting som tester og til en viss grad god kode er noe som man tenker litt mindre på}

\textcolor{red}{selv om NAV har kommet langt, så er det forstatt litt å gå -> hvis man ikke lager en spesiell "NAV applikasjon", som f.eks et saksbehandlingssytem, så finner man fort at ut en del av hjelpeverktøyene ikke hjelper så mye -> da blir du plutselig sittende der og må oppfinne hjulet på nytt selv og prøve å tilpasse applikasjonen din til hjelpeverktøyet -> dette tar mye tid og ressurser, og da har man som regel kuttet på programvarekvalitet som tester osv}

\textcolor{red}{hjelpeverktøyene kan gjøre det enda brattere for en junior å gi verdi for etaten -> NAIS er veldig bra, når man kan alt av DevOps og platform, og vet hvordan det funker, sidne det abstraherer bort alt det kjipe -> men, hvis du ikke kan det som er abstrahert bort, så vil du ikke skjønne hva du tjener på å bruke en slik platfor, og ikke minst, du vet ikke hvordan du skal bruke den for å gjøre det lettere å lage en "NAV applikasjon" -> dermed blir veldig mye tid og ressurser brukt på å lære seg noe som egt skal spare ressurser og gjøre utviklings-hverdagen smudere. Tok meg 1 år før jeg var komfertabel som nyutdannet med NAIS -> hjelpeverktøyene krever en vis forskunskap for at de skal hjelpe i å bygge ting av kvalitet, og den forkunskapen må etaten forsikre seg at alle som skal bruke de har}

\textcolor{red}{hjelpeverktøyene kan bli en hvilkepute for teamet i at de ikke tar godt nok ansvar i visse aspekter ved kvalitetssikring -> man tenker bare at hjelpeverktøyet fikser, også trenger man ikke å tenke på det lenger -> men så dekker kun hjelpeverktøyet deler av kvalitetssikringen, ikke alt}

\textcolor{red}{man sammenligner seg med de gamle applikasjonene, og tenker at alt som er bedre en de er av høy kvalitet, bare fordi man hører at "gammelt er dårlig, nytt bra" -> man blir blind på at enten: (I) det gammle fortsatt er av høy kvalitet, kanskje ikke nødvendig å bytte ut og (II) det nye som blir laget, er laget med dårlig kvalitet uten spesielle sjekker som bekrefter kvaliteten, så ender man med å bytte ut det gamle gode, med det nye dårligere -> slik som man har sett litt med helseplatformen}

\textcolor{red}{selv om etater som NAV har gjort en stor endring på å gjøre det smudere å utvikle programvare, så merken man fortsatt at man er et mål for Norsk politkk -> "alt" handler om årlige budsjetter, og det blir en kamp hvert år om å få nok ressurser -> hvis et større prosjekt har mindre framgang en det som ønsker, så blir det mye styr, gjerne utenifra -> da blir det gjerne eksterne revisjoner, som er beskrevet som egt ikke gjør noe med kvaliteten, bare tar tid}

\textcolor{red}{det å jobbe i Norsk offentlig sektor kan være noe som blir sett ned på -> gjør at man til tider ikke føler seg verdsatt for innsatsen man gjør for å forbedre hverdagen til innbyggerne av -> når man ikke blir verdsatt så gidder man ikke gi full innsats i jobben sin som utvikler -> fører som regel til at man ikke gidder å sette inn innsats i å kvalitetssikre det man lager, man gjør bare "bare minimum"}

\section{Limitations}
\textcolor{red}{\begin{itemize}
    \item Det er mange etater i Norsk offentlig sektor, det at jeg kun har vært hos 3 kan gi et skevt bilde av hele Norsk offentlig sektor, kan hende at jeg har generalisert litt mye
    \item Det at jeg har intervjuet relativt få i hver etat, kan også hende at jeg har generalisert litt mye
    \item Det at jeg er utvikler i Norsk offentlig sektor, kan slå begge veier for interview participant'sa
    \item Det at jeg er utvikler i Norsk offentlig sektor kan hende at jeg er biased, og lener meg mer mot å snakke etatene opp, spesielt NAV
    \item Lite forskning som baserer seg på programvarekvalitet i Norsk offentlig sektor (tror jeg, må søke litt mer) -> Validity issues
    \item De ansatte i Norsk offentlig sektor har skrevet under taushetsavklaring -> kan være info som de vil ha sagt, som de ikke kan si -> eller informasjon som jeg har latt være å ta med, for å opprettholde etikk
\end{itemize}}


