\chapter{Discussion}
As mentioned by multiple participants in the interviews, it is important to consider the context of the software, when assuring quality in that software. This could be seen as important for each agency in the Norwegian public sector, and the whole sector itself. Some aspects regarding software quality and its assurance such as DevOps and product teams in the Norwegian public sector are true for the whole software development industry. While other factors such as allocation of resources and it's effect on how software is developed and it's quality seems to be a somewhat unique for the Norwegian public sector.

\section{Practices}
\textcolor{red}{Snakke litt om de generelle metodene som blir brukt for å sikre programvarekvalitet}

\cite{smm_2018}

Security -> \cite{sh_2018}

\section{DevOps and Product-Teams}
\textcolor{red}{Snakke om hvordan etatene har rigget seg rundt devops og produktteam for å sikre kvalitet}

\cite{am_2020}\cite{smm_2018}\cite{ml_2022}\cite{mm_2021}\cite{dsc_2019}

\section{Financing and Resources}
\textcolor{red}{\begin{itemize}
    \item Hvordan satens prosjektmodell skaper dårlig kvalitet
    \item Mattilsynet har stopped å bruke statens prosjetmodell. Er det håp for andre etater?
    \item Måten budsjetter i offentlig sektor er satt opp kan føre til dårlig kvalitet
    \item Lite ressurser generelt fører til dårligere kvalitet
\end{itemize}}

\cite{sh_2018}\cite{csw_2011}

\section{Legal requirements}
No research mentioned legal requirements for a piece of software as an important factor to influence the quality of the software. However it was described as a challenge for delivering software of high quality to the Norwegian citizens, for multiple of the agencies in the study. These challenges were both describes as stemming from the legal requirements imposed by the Norwegian government, and other laws such as GDPR.

For a piece of software to be of good quality in the context of the Norwegian public sector, it is described as being able to adhere to legal requirements. This includes the ability to change its software to new or updated legal requirements. This is something which Skatteetaten and NAV is described as being able to, however the Norwegian Police Service is described as not being able to. The reasoning for why the Norwegian Police Service not being able to maintain their software in relation to legal requirements were not provided. But by the definition of high software quality in the Norwegian public sector, it could seem that the software of the Norwegian Police Service is lower than other agencies.

Skatteetaten seems to be the agency in this study, where its direction is software development are most driven by new policies and laws decided by the Norwegian government. And describing that Skatteetaten therefore only have the resources to implement these legal requirements, without much left for other quality aspects such as usability. And some processes at Skatteetaten not having enough resources to be automated by software. It would therefore seem that the Norwegian government are not allocating enough resources other than the minimum when deciding on new laws and policies. Ending up in the agencies themselves not being able to create software with the desired quality and fulfilling needs that the user would expect from that software.

Laws such as GDPR are described as preventing agencies to deliver high quality software to the Norwegian citizens. Such as the case with Skatteetaten and the Brønnoysund Register centre described in \autoref{sec:legal}. Where both agencies could combine the user data in one interface, providing software of better usability than existing solutions. However in this case laws on data-sharing blocked this, even tough the agency's would not use each others data about the user internally and the user already owning the data they were displaying. It was only displayed in a new format, in a single interface instead of two separate interfaces. It would therefore seem like the laws for data sharing which is set to defend the Norwegian citizens might end up hurting the citizens. Due to it preventing the development of higher quality software than existing software. And helping the user gain better information in the Norwegian public sector, due to increased usability.

It could therefore be seen that no matter how much the Norwegian government wish to improve the digital services provided by the agencies. The laws and policies which lay in their foundation have to be digitisation friendly. Thus making further efforts on digitising the laws and policies even more challenging and resource intensive.

These reflections on software quality from the participants can seem to contract somewhat. It is said that for a piece of software in the Norwegian public sector to be of high quality, it needs to adhere to legal requirements. However also stating that the same legal requirements are creating challenges to provide software which is of high quality. This contradicting views on high software quality can create false expectations of the software, leading to potential resources wasted, which could go into improving the quality of other software. Reducing the problem of not having enough resources to meed the unclear standards of the software.

\section{Cooperation between agencies}
\textcolor{red}{Her må jeg snakke om hva som driver utvikling i etatene, og hvordan dette ikke passer inn med regjeringens mål om å en "felles samarbeidsplatform på tvers av etater for å gjøre tjenstene til etatene bedre"}

\section{Researchers Perspective}
\textcolor{red}{Her må jeg snakke om min egen erfaring som utvikler i Norsk offentlig sektor}

\section{Limitations}
\textcolor{red}{\begin{itemize}
    \item Det at jeg bruker case study
    \item Det at jeg er utvikler i Norsk offentlig sektor, kan slå begge veier for interview participant'sa
    \item Lite forskning som baserer seg på programvarekvalitet i Norsk offentlig sektor (tror jeg, må søke litt mer) -> Validity issues
\end{itemize}}

\textcolor{red}{Skal jeg få inn litt om etikk her? Er jo litt fare at jeg er både utvikler i offentlig sektor og forsker. Kan gi "validity" problems. Er også problemer med at de andre ansatte kan lettere gi med sensitive dokumenter/opplysninger, som ikke er egnet å publiseres (større ansvar for det etiske pga min posisjon)}
