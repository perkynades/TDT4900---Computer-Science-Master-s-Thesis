\chapter{Discussion}
As mentioned by multiple participants in the interviews, it is important to consider the context of the software, when assuring quality in that software. This could be seen as important for each agency in the Norwegian public sector, and the whole sector itself. Some aspects regarding software quality and its assurance such as DevOps and product teams in the Norwegian public sector are true for the whole software development industry. While other factors such as allocation of resources and it's effect on how software is developed and it's quality seems to be a somewhat unique for the Norwegian public sector.

\section{Practices}
\textcolor{red}{Snakke litt om de generelle metodene som blir brukt for å sikre programvarekvalitet}

\cite{smm_2018}

Security -> \cite{sh_2018}

\section{Agility}
\textcolor{red}{Snakke om hvordan endringsenve er beskrevet som viktig i kvalitet}

\textcolor{red}{Snakke om hvordan etatene har rigget seg rundt devops og produktteam for å sikre kvalitet}

\cite{am_2020}\cite{smm_2018}\cite{ml_2022}\cite{mm_2021}\cite{dsc_2019}

\textcolor{red}{Snakke om hvordna par-programmering og mob-programmering øker kvalitet, samt at det øker hastigheten og det blir lettere å gjøre endringer}

\section{Financing and Resources}

\textcolor{red}{Få inn at konsulenter sitter sammen i produkt-teamene? Kanskje passer bedre under "DevOps and Product teams"?}

All agencies in this study described their main challenge when ensuring that their software is of high quality, is the lack of IT-resources available. It was described at NAV that it did not have the necessary resources to deal with the \gls{technical_debt} in their payment system. If this system fails due to this, it's negative consequences could go beyond the specific citizen not receiving their payment. But the Norwegian society and economy as a whole being negatively effected. 

It is estimated that in 2018, NAV constituted for 15.5\% of Norway's gross domestic product \cite{nav_ytelsene_frem_mot_2060_2019}, totalling to about 434 Billion NOK in 2018 \cite{faktaark_finansdepartementet_2020}. This means that the Norwegian economy have a heavy reliance on NAV's payment system, and any amount of downtime can have negative consequences for the Norwegian economy and in turn, the Norwegian society. It is therefore worrying that agencies in the Norwegian public sector are not given enough resources to fix nation critical systems. The situation for such systems might be worse than expected, due to the competency for fixing such system disappearing. This could mean that the whole system needs to be replaced with a new system, as it's faults were not fixed while the necessary competence were available.

An agency might have enough resources at their disposal to fix such faults, but the yearly budgeting in the Norwegian public sector might prevent the resources being used. However yearly budgeting in the Norwegian public sector is also a challenge for new software being developed. It is described that the yearly budged is limited, hence software quality being prioritised lower. Which could be the reason why agencies such as NAV end up with software that has issues later on in its lifetime. An issue which could be prevented by the agencies receiving enough fund to implement high software quality as the software is being developed.

It is described that some agencies want to have the same IT-capabilities like NAV and Skatteetaten, with a large in-house development environment and product-teams. Yet do not have enough staff or resources to achieve this vision. Thus resulting in requirement specifications being sent to external vendors to create the software. Using external vendors or consultants is described as something which could cost up to three times more than creating it in-house. Removing even more resources which could be used on quality assurance. And since the agency already have a limited budged, might mean lowering it's expectations on software quality, such as usability.

The use of external vendors who lack the domain knowledge that is present in an in-house development environment, could also lead to lower quality in the software delivered. The external vendor might be able to deliver a piece of software that delivers it's legal requirements based on the requirement specification. But without the domain knowledge on how the agencies users will interact with the software and how to will function in the agency. This could be the case with the health platform, or the Photo app for Mattilsynet. The external vendors delivered on what was obligatory in the contract, but did not deliver well in regards to the domain the software would exist in. Leading to delivering software of lower quality than expected, or software which is outdated when delivered, such as NAV's system for delivering pension benefits. Simply because NAV's domain had changed during it's long development process.

With yearly budgets described as being limited and having software created by expensive external vendors. It has forced most agencies to request funding for their software development projects, from the project wizard delivered by the Norwegian Digitisation agency. The project wizard is described by the Norwegian government to ensuring that the Norwegian tax payer's money is spent correctly and leads to something of high quality being produced. However the perception from the participants, was unanimous on the project wizard leading to the opposite. Software which were of low quality and costing more for the Norwegian tax payers, due to it's requirements for receiving the funding.

Being able to mutate and easily maintain a piece of software was described as important for software to be of high quality. The main criticism with the project wizard is described as being structured similarly to the waterfall model. And therefore making it more difficult to mutate a piece of software later in the development project. This challenge in mutability is also prohibiting new knowledge to be implemented in the software. As the project wizard is described to be hostile towards scope creep, when in reality it is new knowledge about the problem being solved. This leading to faults in the system needing to be fixed after it has been delivered, resulting in the faults being more expensive to fix than in the development phase \cite{sh_2018}\cite{csw_2011}.

 Challenges with mutability when using the project wizard increases in combination with changing domains for the agencies. This as using the project wizard can mean it takes longer before any value is delivered to the end-user, as with NAV's system for delivering pension benefits. And as the scale of the projects accepted by the project wizard are larger than usual software projects, as described by one of the expert participants. Their importance for the Norwegian society could be argued as being larger than other development projects in the Norwegian public sector. Thus the project wizard hurting itself, by resulting in software that are not fully relevant for the Norwegian tax payers, as the tax payers money not being spent correctly.

The project wizard are also in contrast to how the different agencies in the study has organised their development teams in order to create software of high quality. It is described as a piece of software never is complete, and if not constantly maintained its functionality will whither due to \gls{technical_debt}. Resulting in the agencies organising their teams as product teams, where the same teams developing the software, are it's owner and maintains it in perpetuity. This to ensure that the people who have the most knowledge in the software, are responsible for it operating and delivering value to the end-user as expected. However when developing with funds from the project wizard, and development is complete, the project is complete, and funding for further operations is reduced or not existing. This reduction of funds for operations leads to functionality withering, as with Mattilsynets systems "Mats", leading to a reduction in quality in relation to user expectations.

These challenges relating to the use of the project wizard due to insufficient funding has also been described as a challenge of prioritisation, not insufficient funding. It being a challenge to priorities either improving the quality of existing software, or creating new features by either extending existing software, or creating new software. And the agencies who do not have the resources to sufficiently prioritise both, should improve at choosing a single prioritisation, instead of multiple. This to prevent having to use the project wizard, which might create software of lower quality. Continuing the issue of not being able to prioritise correctly, creating something which can be described as an "evil circle" for the development and maintenance of software in the agencies.

With all the challenges described by using the project wizard, it will be interesting to follow Mattilsynets progress in developing and maintaining their IT-systems. As Mattilsynet explained that they have stopped using the project wizard for funding their projects, and instead moving to product teams. This should lead to Mattilsynet being able to create new software of higher quality than with the project wizard, if believed by the participants of this study.

It is still important to recognise even with it's alleged challenges, the project wizard has provided opportunities for the agencies in the Norwegian public sector. It has given, to some extent, the necessary funding for the agencies to create new software that servers the Norwegian citizens. The heavy reliance NAV has had on the project wizard enabled them to build up a large in-house development environment, as the funding spent on expensive external consultants was replaced with an in-house development environment. By replacing a single expensive consultants, NAV was able to hire 2-3 in-house developers. Without this, NAV would might not have the necessary resources to implemented methods such as DevOps and organising into product teams. Which are described by NAV as important factors for assuring high quality in their software.

\section{Legal requirements}
No research mentioned legal requirements for a piece of software as an important factor to influence the quality of the software. However it was described as a challenge for delivering software of high quality to the Norwegian citizens, for multiple of the agencies in the study. These challenges were both describes as stemming from the legal requirements imposed by the Norwegian government, and other laws such as GDPR.

For a piece of software to be of good quality in the context of the Norwegian public sector, it is described as being able to adhere to legal requirements. This includes the ability to change its software to new or updated legal requirements. This is something which Skatteetaten and NAV is described as being able to, however the Norwegian Police Service is described as not being able to. The reasoning for why the Norwegian Police Service not being able to maintain their software in relation to legal requirements were not provided. But by the definition of high software quality in the Norwegian public sector, it could seem that the software of the Norwegian Police Service is lower than other agencies.

Skatteetaten seems to be the agency in this study, where its direction is software development are most driven by new policies and laws decided by the Norwegian government. And describing that Skatteetaten therefore only have the resources to implement these legal requirements, without much left for other quality aspects such as usability. And some processes at Skatteetaten not having enough resources to be automated by software. It would therefore seem that the Norwegian government are not allocating enough resources other than the minimum when deciding on new laws and policies. Ending up in the agencies themselves not being able to create software with the desired quality and fulfilling needs that the user would expect from that software.

Laws such as GDPR are described as preventing agencies to deliver high quality software to the Norwegian citizens. Such as the case with Skatteetaten and the Brønnoysund Register centre described in \autoref{sec:legal}. Where both agencies could combine the user data in one interface, providing software of better usability than existing solutions. However in this case laws on data-sharing blocked this, even tough the agency's would not use each others data about the user internally and the user already owning the data they were displaying. It was only displayed in a new format, in a single interface instead of two separate interfaces. It would therefore seem like the laws for data sharing which is set to defend the Norwegian citizens might end up hurting the citizens. Due to it preventing the development of higher quality software than existing software. And helping the user gain better information in the Norwegian public sector, due to increased usability.

It could therefore be seen that no matter how much the Norwegian government wish to improve the digital services provided by the agencies. The laws and policies which lay in their foundation have to be digitisation friendly. Thus making further efforts on digitising the laws and policies even more challenging and resource intensive.

These reflections on software quality from the participants can seem to contract somewhat. It is said that for a piece of software in the Norwegian public sector to be of high quality, it needs to adhere to legal requirements. However also stating that the same legal requirements are creating challenges to provide software which is of high quality. This contradicting views on high software quality can create false expectations of the software, leading to potential resources wasted, which could go into improving the quality of other software. Reducing the problem of not having enough resources to meed the unclear standards of the software.

\section{Cooperation between agencies}
\textcolor{red}{Her må jeg snakke om hva som driver utvikling i etatene, og hvordan dette ikke passer inn med regjeringens mål om å en "felles samarbeidsplatform på tvers av etater for å gjøre tjenstene til etatene bedre"}

\textcolor{red}{Knapt ressurser til å utvikler hva som er "leagaly requirered" -> Må få mer ressurser hvis regjeringen ønsker samarbeid}

\textcolor{red}{Er samarbeid virkelig, teknisk sett, hensiktsmessig? Alle etatene har sine egne domener som former hvordan teknologien under ser ut}

\section{Researchers Perspective}
\textcolor{red}{Her må jeg snakke om min egen erfaring som utvikler i Norsk offentlig sektor}

\section{Limitations}
\textcolor{red}{\begin{itemize}
    \item Det at jeg bruker case study
    \item Det at jeg er utvikler i Norsk offentlig sektor, kan slå begge veier for interview participant'sa
    \item Lite forskning som baserer seg på programvarekvalitet i Norsk offentlig sektor (tror jeg, må søke litt mer) -> Validity issues
\end{itemize}}

\textcolor{red}{Skal jeg få inn litt om etikk her? Er jo litt fare at jeg er både utvikler i offentlig sektor og forsker. Kan gi "validity" problems. Er også problemer med at de andre ansatte kan lettere gi med sensitive dokumenter/opplysninger, som ikke er egnet å publiseres (større ansvar for det etiske pga min posisjon)}
