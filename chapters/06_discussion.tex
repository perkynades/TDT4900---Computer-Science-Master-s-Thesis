\chapter{Discussion}

\textcolor{red}{Ble nenvt av participant \#1 at metodene privat og offentlig sektor bruker er like -> programvarekvalitet i privat sektor kan være relevant forskning}

\section{Practices}
\textcolor{red}{Snakke litt om de generelle metodene som blir brukt for å sikre programvarekvalitet}

\section{DevOps and Product-Teams}
\textcolor{red}{Snakke om hvordan etatene har rigget seg rundt devops og produktteam for å sikre kvalitet}

\section{Financing and Resources}
\textcolor{red}{\begin{itemize}
    \item Hvordan satens prosjektmodell skaper dårlig kvalitet
    \item Måten budsjetter i offentlig sektor er satt opp kan føre til dårlig kvalitet
    \item Lite ressurser generelt fører til dårligere kvalitet
\end{itemize}}

\section{Legal requirements}
\textcolor{red}{Snakke om at lovene som skal digitaliseres i offentlig sektor er lite digitaliseringsvenlige, som gir utfordringer for kvalitet}

\section{Cooperation between agencies}
\textcolor{red}{Her må jeg snakke om hva som driver utvikling i etatene, og hvordan dette ikke passer inn med regjeringens mål om å en "felles samarbeidsplatform på tvers av etater for å gjøre tjenstene til etatene bedre"}

\section{Researchers Perspective}
\textcolor{red}{Her må jeg snakke om min egen erfaring som utvikler i Norsk offentlig sektor}

\section{Limitations}
\textcolor{red}{\begin{itemize}
    \item Det at jeg bruker case study
    \item Det at jeg er utvikler i Norsk offentlig sektor, kan slå begge veier for interview participant'sa
    \item Lite forskning som baserer seg på programvarekvalitet i Norsk offentlig sektor (tror jeg, må søke litt mer)
\end{itemize}}

\textcolor{red}{Skal jeg få inn litt om etikk her? Er jo litt fare at jeg er både utvikler i offentlig sektor og forsker. Kan gi "validity" problems. Er også problemer med at de andre ansatte kan lettere gi med sensitive dokumenter/opplysninger, som ikke er egnet å publiseres (større ansvar for det etiske pga min posisjon)}
