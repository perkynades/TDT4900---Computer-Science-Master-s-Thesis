\chapter{Discussion} \label{sec:discussion}
This section will discuss the results from \autoref{sec:results} in relation to the scientific literature presented in \autoref{sec:theoretical_background}. And the research questions and goals of this study presented in \autoref{sec:motivation} will be attempted to be answered.

\section{RQ1: What practices do agencies of the Norwegian public sector use to assure quality in its software?}
The practices relating to software quality assurance in the Norwegian public sector were described in \autoref{sec:results} as being in technical techniques and software development methodologies.

A non-functional requirement described in \autoref{sec:non_functional_requirments_case_bg} often used as a measurement of quality in software, is security. The results from \autoref{sec:security} show that this non-functional requirement is also of importance to the agencies in this study. Leading to multiple practices described in \autoref{sec:security} used to ensure secure software, such as security knowledge, audit-logging and security analyses. It is shown that having a strong focus on security before a piece of software is delivered can be beneficial to the Norwegian public sector. Weak security in software can have large consequences and costs to fix if discovered after it is delivered to the user \cite{csw_2011}. It could therefore be argued that in the context of weak security can lead to mistrust by Norwegian citizens. As their trust might be challenged if a Norwegian citizen discovers their sensitive data to be lost and shared online. Leading to negative consequences due to the importance of Trust in the Norwegian public sector \cite{oecd_2022}.

%\autoref{sec:security} A non-functional requirement often used as a measurement of quality in software, is security. The results show that security is of high importance to the agencies in the Norwegian public sector, as their software and data are of high value. This has lead to the agencies using multiple different methods and ways to organise, in order to ensure that their software are secure. It is also shown that having a strong focus on security before a piece of software is delivered can be beneficial to the Norwegian public sector. As weak security in software can have large consequences and costs to fix if discovered after it is delivered to the user \cite{csw_2011}. It could be argued that in the context of the Norwegian public sector, weak security can lead to mistrust by the Norwegian citizens. As their trust might be challenged if a Norwegian citizen discover their sensitive data to be lost and shared online.

Different code inspection methods are described in \autoref{sec:inspection} to be used by the agencies of this study to detect faults in their software. One manual method, code review, is described as being time-consuming and does not find any critical faults. While other automatic methods such as SonarQube is said to have found critical fault in Mattilsynet's system, "Mats". This difference in methods and their respective benefits and drawbacks implies that the Norwegian public sector has to choose such methods with care. As some methods might steal valuable time, without returning to enough value in software quality. It could also seem that if choosing a code inspection method, an automatic method is recommended, as it has shown to be important in increasing software quality \cite{smm_2018}.

%\autoref{sec:inspection} Different code inspection methods is described to be used by the agencies to find faults in their software. One manual method, being code review, is described as taking long time and does not find any critical faults. While other automatic methods such as SonarQube, is said to have found critical fault in Mattilsynets system, "Mats". If the Norwegian public sector want to use code inspection methods to assure quality in their software, the specific method has to be chosen with care. As some methods might steal valuable time, without returning to enough value in software quality. It can therefore seem that if using code inspection, automation is recommended, as both Mattilsynet and scientific literature has show it to be in importance of increasing software quality \cite{smm_2018}.

Other techniques resembling the code inspection methods, as described in \autoref{sec:software_development_techniques} are used to assure software of quality in the Norwegian public sector. These techniques being \gls{pair_programming} and \gls{mob_programming}. While also acting as code inspection methods, these are described as increasing ownership and knowledge of the software, increasing its quality. Using pair- and mob programming in the Norwegian public sector is beneficial, as the context of a public sector is complex \cite{jc_2010}, leading to the software tasks being more complex, where techniques such as \gls{pair_programming} are said to beneficial \cite{jeh_2009}. 

%\autoref{sec:software_development_techniques} Code inspection tools could also be avoided all together by implementing certain programming techniques, such as pair- and mob-programming. As these techniques also acts as code inspection methods, with several extra benefits, being increased ownership and knowledge in the software, described as increasing software quality. Implementing pair- and mob-programming can also be of importance to the Norwegian public sector, as their software is said to be large and more complex than in the private sector. Making ownership and knowledge of the software being developed important. \cite{jeh_2009}

Specific testing methods used to assure quality in software have changed as shown in \autoref{sec:testing}, changing from acceptance tests to automatic tests. This change could have multiple reasons. As shown in \autoref{sec:development_projects}, the agencies are moving away from projects with a set deadline, usually funded by the project wizard described in \autoref{sec:project_wizard_case_bg}, requiring quality checks such as acceptance tests. Another reason could be due to the implementation of DevOps in the Norwegian public sector, shown in \autoref{sec:devops}, in order to be able to deliver frequent updates to the end user. And in order to deliver frequent updates with low risks, high test coverage with automatic tests is described in \autoref{sec:testing}. The agencies in the Norwegian public sector should still be careful about moving fully to automatic tests, as this can lead to lower quality, due to manual testing still being needed to achieve high quality \cite{dsc_2019}. 

%\autoref{sec:testing} Automatic testing is shown to be an important tool in the assurance of software quality in the agencies of the Norwegian public sector. As the agencies has moved away from projects with a set deadline, a change from acceptance tests to automatic tests were made. This is important as automatic testing is described as an important tool in validating the quality of a piece of software when it is subject to frequent mutations. High test coverage with automatic tests is therefore used as a safety net when performing frequent mutations to a software. It is also emphasised that automatic testing is needed, however manual tests are still important for a piece of software to achieve high quality \cite{dsc_2019}. Therefore the agencies in the Norwegian in public sector cannot fully rely on automatic tests as the primary testing method for achieving high software quality.

The practice of registering \gls{technical_debt}, as described in \autoref{sec:technical_debt} is used by Mattilsynet to help in software quality assurance. This practice is seen as important, as it enables organisations to have more optimal use of their resources \cite{mv_2022}, and enables development teams to identify and repay \gls{technical_debt} in a timely fashion \cite{mv_2022}. The practice of registering \gls{technical_debt} can therefore be seen as quite crucial in assuring software of quality. As it was shown in \autoref{sec:technical_debt} and \autoref{sec:resources} that the Norwegian public sector has a high degree of \gls{technical_debt}.

%\autoref{sec:technical_debt} The ability to register technical debt is described as important, as technical debt was described as the definition of low software quality. It would seem like there is a high degree of technical debt in the Norwegian public sector. Such as with Mattilsynets system "Mats", where its technical debt was described to cost Mattilsynet about 10 Billion NOK over 15 years. The agencies might not have the resources the fix all technical debt, however by registering it, certain measures can be implemented in order to at least minimise the risks of the technical debt. \cite{mv_2022}

Measurements are shown in \autoref{sec:measurements} as a method for agencies in the Norwegian public sector to measure the quality of software in runtime. Collecting software metrics is valuable, as it helps in creating meaningful estimates and guides development teams in taking decisions in their software \cite{jkc_2010}. The measurements could be used to compare if the quality of a piece of software has increased or decreased over a longer period. Helping to understand where in a piece of software faults or low quality are present, and guide on what decisions should be made to fix the faults or improve the quality.

%\autoref{sec:measurements} Measuring the systems in the Norwegian public sector in certain aspects, such as downtime and faults, can create valuable statistics that can help increase software quality. These statistics can be used to compare if the quality of a piece of software has increased or decreased over a longer period. Allowing the development teams to do a more qualitative analysis of the state of their software. Which could improve software quality by having data on when and where in their software is of low quality, making it easier for the development team to improve that particular point of low software quality.

Implementing certain properties to software is described in \autoref{sec:software_properties} to be used for increasing the quality of software. And it is being explained how Mattilsynet is going to enforce such implementations soon. However little research suggested the implementation of such properties to increase quality, and the properties in general lacking research. Therefore enforcing such implementations without proper research to evaluate the enforcement, could lead to negative consequences. It is also shown in \autoref{sec:team} that the Norwegian public sector is implementing autonomous product teams. Enforcing such implementations could lead to the autonomy of the product teams being reduced, which is described as important in creating software of high quality.

%\autoref{sec:software_properties} Software which implement certain properties are described as being of higher quality. Such as Mattilsynets software using an API-first approach with \gls{open_api} and the \gls{publish_subscribe} pattern. With Mattilsynets explaining such software properties is going to be enforced in future software to ensure reuse and high software quality. It could therefore be of interested that the rest of the Norwegian public sector follow such enforcement in software properties to ensure high software quality. However certain enforcement could challenge the autonomy of the product teams, as the software is forced to include technology not fit for the teams domain of the software being developed. Then resulting in lower software quality, opposite of what was intended.

Agencies in the Norwegian public sector are described in \autoref{sec:team} to organise their development teams as cross-functional in order to help in the development and maintenance of software of high quality. This is an important practice, as cooperation in IT teams, such as between operations and development is crucial to enhance the quality of software \cite{aw_2019}. This could be the reason why the Healthplatform of mid-Norway is lacking in quality, as described in \autoref{sec:knowledge}, it was made without knowledge of usability in their development teams.

The practice of cross-functional teams is described in \autoref{sec:team} to be in addition to the development teams being organised as autonomous product teams. And that this organisation is important in developing software of high quality. However, little research was found on the effects of product teams on software quality assurance, or in product teams as described in \autoref{sec:product_teams}. These teams being autonomous could be important for assuring software of quality, as team autonomy is described as being important in achieving agility, \cite{gl_2010}, and agility could be important in software quality \cite{mh_2004}.

%\autoref{sec:team} The product teams at Mattilsynet are describe as having a high degree of autonomy, however without alignment, the benefits of autonomous product teams can decrease. By not having alignment in areas like technology, a wide range of technology can lead to be used without much reasoning, creating technical debt. A lack of alignment can also lead to the wrong problem being solved by the product teams, which as been described as more important than good software quality. As there is no point in creating software of high quality, if it does not solve the correct problem, the problem has no reason in its function. Only wasting the already limited time and resources in the Norwegian public sector. 

DevOps as a practice to gain feedback more frequently and increase software quality is described in \autoref{sec:devops} as being implemented in agencies of the Norwegian public sector. This is because the development teams are able to deliver small changes frequently to the users. This decreases the scope of which parts of the software the users have feedback about and the area affected by eventual faults are limited. The development teams can therefore use the same methods to quickly fix their software based on the feedback, and quickly get new feedback from the user. This use of DevOps has been found to assure quality due to its ability to continuously deliver new updates to the user \cite{am_2020}\cite{smm_2018}\cite{ml_2022}.

%\autoref{sec:devops} As a method to gain better and more frequent feedback from it's users, DevOps has been implemented as a core software development methodology for the agencies in this study. This as the development teams are able to deliver small changes frequently to the user. This decreases the scope of which parts of the software the user has feedback about, and any eventual faults are limited. The development teams can therefore use the same methods to quickly fix their software based on the feedback, and quickly get new feedback from the user. The use of DevOps has been studied and by other researchers and found to assure quality due to its ability to continuously deliver new updates to the user \cite{am_2020}\cite{smm_2018}\cite{ml_2022}.

Sharing of knowledge and increasing the level of knowledge in their agency was described in \autoref{sec:knowledge} as important in assuring the quality of software in the Norwegian public sector. And the use of DevOps is an important method for assuring quality through increased knowledge sharing between employees \cite{smm_2018}\cite{mm_2021}. However, none of the agencies described DevOps as a method for increasing knowledge in their agency. Rather as described in \autoref{sec:knowledge}, using more traditional methods, such as participating in conferences or leaders of the development teams meeting to share experiences. It could therefore be the case that DevOps have effects on knowledge sharing and its effect on software quality which the Norwegian public sector are unaware of.

%\autoref{sec:devops} \autoref{sec:knowledge} Sharing of knowledge and increasing the level of knowledge in an organisation has been described as being important in assuring quality in software. With recent research describing that the use of DevOps being an important method for assuring quality through increased knowledge sharing between employees \cite{smm_2018}\cite{mm_2021}. However none of the agencies described DevOps as a method for increasing knowledge in their organisation. Rather using more traditional methods, such as participating in conferences or leaders of the development teams meeting to share experiences. DevOps could therefore have unknown effects on knowledge sharing and it's increase of software quality.

With all the seemingly positive effects of DevOps on software quality described in \autoref{sec:devops}, it could be quite tempting for other agencies in the Norwegian public sector to implement it. However, if not implemented correctly, DevOps can hinder software quality assurance \cite{dsc_2019}. In order for DevOps to increase quality, the development team have to take shared responsibility for quality assurance. Rather than a single person having all the responsibility for quality assurance \cite{dsc_2019}. This shared responsibility also extends through the software's life, from development to its phase-out \cite{dsc_2019}. If more agencies in the Norwegian public sector want to introduce DevOps correctly, it needs long-lasting product teams which share responsibility, or else the software quality can suffer.

%\autoref{sec:devops} \autoref{sec:team} It could therefore be seen as quite tempting to introduce DevOps as a measure due to its affect on software quality, and it's popularity among agencies in the Norwegian Public Sector. However recent research showing that DevOps can hinder software quality assurance if not implemented in the organisation correctly \cite{dsc_2019}. In order to DevOps to increase quality, the development team have to take a shared responsibility for quality assurance. Rather than a single person having all the responsibility for quality assurance \cite{dsc_2019}. This shared responsibility also extends through the software's life, from development to its phase out \cite{dsc_2019}. If more agencies in the Norwegian public sector want to introduce DevOps correctly, it needs long-lasting product teams which share responsibility, or else the software quality can suffer.

This temptation combined with it being stated that the effects of testing in DevOps environments had not yet been studied systematically by scientific literature \cite{dsc_2019}\cite{ja_2016}. This could mean that the agencies of the Norwegian public sector should be even more considerate if implementing DevOps is going to increase software quality. There could be hidden risks or benefits which are not uncovered by scientific literature.


\section{RQ2: What challenges are agencies in the Norwegian public sector encountering in the quality assurance of its software?}
The challenges relating to software quality assurance in the Norwegian public sector were described in \autoref{sec:results} as being in organisational aspects of software development.

The main challenge for software quality assurance in the Norwegian public sector, as described in \autoref{sec:resources}, is the lack of IT resources available. As it was described NAV lacks sufficient resources to deal with the \gls{technical_debt} in its payment system. If such a system fails, it could have negative consequences reaching beyond the specific citizen not receiving their payment, but the Norwegian society and economy being negatively effect. In 2018, it is estimated that NAV constituted for 15.5\% of Norway's gross domestic product \cite{nav_ytelsene_frem_mot_2060_2019}, totalling to about 434 Billion NOK in 2018 \cite{faktaark_finansdepartementet_2020}. It could therefore be argued that the Norwegian economy has a heavy reliance on NAV's payment system, and any amount of downtime can have negative consequences for the Norwegian economy and in turn, the Norwegian society.

%\autoref{sec:resources} All agencies in this study described their main challenge when ensuring that their software is of high quality, is the lack of IT-resources available. It was described at NAV that it did not have the necessary resources to deal with the \gls{technical_debt} in their payment system. If this system fails due to this, it's negative consequences could go beyond the specific citizen not receiving their payment. But the Norwegian society and economy as a whole being negatively effected.

%\autoref{sec:resources} It is estimated that in 2018, NAV constituted for 15.5\% of Norway's gross domestic product \cite{nav_ytelsene_frem_mot_2060_2019}, totalling to about 434 Billion NOK in 2018 \cite{faktaark_finansdepartementet_2020}. This means that the Norwegian economy have a heavy reliance on NAV's payment system, and any amount of downtime can have negative consequences for the Norwegian economy and in turn, the Norwegian society. It is therefore worrying that agencies in the Norwegian public sector are not given enough resources to fix nation critical systems. The situation for such systems might be worse than expected, due to the competency for fixing such system disappearing. This could mean that the whole system needs to be replaced with a new system, as it's faults were not fixed while the necessary competence were available.

The practice of yearly budgeting in the Norwegian public sector is described in \autoref{sec:budgeting} as a challenge for performing software quality assurance. An agency might have enough resources at its disposal to improve software quality of existing software, however, the rules about how the yearly budgets should be used prevent it. Yearly budgeting is also described in \autoref{sec:budgeting} to be a challenge for the creation of new software, as the yearly budget is usually limited. This could lead to software quality being prioritised lower than functional requirements, resulting in the same challenges described with Norwegian public sector software today in \autoref{sec:technical_debt}. \hyperref[sec:budgeting]{Section~\ref*{sec:budgeting}} also describes that software quality is not a good enough reason to change the budget. Even if software quality assurance efforts are shown to be good investments in the public sector, budgetary constraints can hinder its consideration \cite{jc_2010}.  

%\autoref{sec:budgeting} An agency might have enough resources at their disposal to fix such faults, but the yearly budgeting in the Norwegian public sector might prevent the resources being used. And yearly budgeting in the Norwegian public sector is also a challenge for new software being developed. It is described that the yearly budged is limited, hence software quality being prioritised lower. Which could be the reason why agencies such as NAV end up with software that has issues later on in its lifetime. An issue which could be prevented by the agencies receiving enough fund to implement high software quality as the software is being developed.

Using external vendors is described in \autoref{sec:contracting_and_tenders} as a challenge in the Norwegian public sector, with it being described as resulting in lower quality. This lower quality was described for three reasons: (I) distance between the user and the developers, (II) lack of operational knowledge of the external vendors, and (III) hard transition between development and operation at project handover. However, with these described challenges in external vendors, it can still seem that this practice is used in a large part of the Norwegian public sector as described in \autoref{sec:contracting_and_tenders}. Something that is described as being the case for other public sectors due to shortages in in-house resources \cite{jc_2010}, and described in \autoref{sec:project_wizard} to be true for the development environments in the Norwegian public sector.

%\autoref{sec:contracting_and_tenders} The use of external vendors who lack the domain knowledge that is present in an in-house development environment, could also lead to lower quality in the software delivered. The external vendor might be able to deliver a piece of software that delivers it's legal requirements based on the requirement specification. But without the domain knowledge on how the agencies users will interact with the software and how it will function in the agency. This could be the case with the health platform, or the Photo app for Mattilsynet. The external vendors delivered on what was obligatory in the contract, but did not deliver well in regards to the domain the software would exist in. Leading to delivering software of lower quality than expected, or software which is outdated when delivered, such as NAV's system for delivering pension benefits. Simply because NAV's domain had changed during it's long development process. "A lack of in-house shortages has led some public sectors to hire contract staff and/or outsource IT functions \cite{jc_2010}."

The lack of resources and budgeting has led to agencies in the Norwegian public sector requesting funding through the Norwegian digitations agency's project wizard, as described in \autoref{sec:project_wizard}. However this necessity is described to have several challenges: (I) challenges in mutability, (II) hostility towards new knowledge, (III) hard transition between development and operation at project handover, (IV) lack of funding when a project is complete, and (V) faulty quality control regime. Such public processes are said to have the possibility to reduce implementation success \cite{jc_2010}, which could lead to faulty software that is expensive to resolve after the software is delivered \cite{csw_2011}. With its challenges, it could seem like the project wizard is not assuring high quality and reducing costs as described in \autoref{sec:project_wizard_case_bg}, instead reducing quality and increasing costs.

%\autoref{sec:budgeting} \autoref{sec:project_wizard} With yearly budgets described as being limited and having software created by expensive external vendors. It has forced most agencies to request funding for their software development projects, from the project wizard delivered by the Norwegian Digitisation agency. The project wizard is described by the Norwegian government to ensure that the Norwegian tax payer's money is spent correctly, and leads to something of high quality being produced. However the perception from the participants, was almost unanimous that the project wizard leading to the opposite. Software which were of low quality and costing more for the Norwegian tax payers, due to it's requirements for receiving the funding.

%\autoref{sec:project_wizard} Being able to easily mutate and maintain a piece of software was described as important for software to be of high quality. The main criticism with the project wizard is described as being structured similarly to the waterfall model. And therefore making it more difficult to mutate a piece of software later in the development project. This challenge in mutability is also prohibiting new knowledge to be implemented in the software. As the project wizard is described to be hostile towards scope creep, when in reality it is new knowledge about the problem being solved. This leading to faults in the system needing to be resolved after it has been delivered, resulting in the faults being more expensive to resolve than in the development phase \cite{csw_2011}.

 %\autoref{sec:project_wizard} Challenges with mutability when using the project wizard increases in combination with changing domains for the agencies. This as using the project wizard can mean it takes longer before any value is delivered to the end-user, as with NAV's system for delivering pension benefits. And as the scale of the projects accepted by the project wizard are larger than usual software projects, as described by one of the expert participants. Their importance for the Norwegian society could be argued as being larger than other development projects in the Norwegian public sector. Thus the project wizard hurting itself, by resulting in software that are not fully relevant for the Norwegian tax payers, as the tax payers money not being spent correctly.

%\autoref{sec:project_wizard} The project wizard are also in contrast to how the different agencies in the study has organised their development teams in order to create software of high quality. It is described as a piece of software never is complete, and if not constantly maintained, its functionality will whither due to \gls{technical_debt}. Resulting in the agencies organising their teams as product teams, where the same teams developing the software, are it's owner and maintains it in perpetuity. This to ensure that the people who have the most knowledge in the software, are responsible for it operating and delivering value to the end-user as expected. However when developing with funds from the project wizard, and development is complete, the project is complete, and funding for further operations is reduced or not existing. This reduction of funds for operations leads to functionality withering, as with Mattilsynets systems "Mats", leading to a reduction in quality in relation to user expectations.

%\autoref{sec:project_wizard} These challenges relating to the use of the project wizard due to insufficient funding has also been described as a challenge of prioritisation, not insufficient funding. It being a challenge to priorities either improving the quality of existing software, or creating new features by either extending existing software, or creating new software. And the agencies who do not have the resources to sufficiently prioritise both, should improve at choosing a single prioritisation, instead of multiple. This to prevent having to use the project wizard, which might create software of lower quality. Continuing the issue of not being able to prioritise correctly, creating something which can be described as an "evil circle" for the development and maintenance of software in the agencies.

%\autoref{sec:project_wizard} With all the challenges described by using the project wizard, it will be interesting to follow Mattilsynets progress in developing and maintaining their IT-systems. As Mattilsynet explained that they have stopped using the project wizard for funding their projects, and instead moving to product teams. This should lead to Mattilsynet being able to create new software of higher quality than with the project wizard, if believed by the participants of this study.

%\autoref{sec:project_wizard} It is still important to recognise even with it's alleged challenges, the project wizard has provided opportunities for the agencies in the Norwegian public sector. It has given, to some extent, the necessary funding for the agencies to create new software that servers the Norwegian citizens. The heavy reliance NAV has had on the project wizard enabled them to build up a large in-house development environment, as the funding spent on expensive external consultants was replaced with an in-house development environment. By replacing a single expensive consultants, NAV was able to hire 2-3 in-house developers. Without this, NAV would might not have the necessary resources to implemented methods such as DevOps and organising into product teams. Which are described by NAV as important factors for assuring high quality in their software.

%\autoref{sec:resources} The presence of expensive consultants are still strong in the Norwegian public sector, due to the lack of staff with IT-knowledge. However these consultants have now been incorporated into the product teams owned by the agencies, as knowledge resources. By making the consultants a part of the in-house development, instead of separating them from the organisation, has allowed the agencies to maintain the software quality standards of in-house development. In addition gaining the advantages of the increased IT-capabilities present from the consultants.

%\autoref{sec:legal} No research mentioned legal requirements for a piece of software as an important factor to influence the quality of the software. However it was described as a challenge for delivering software of high quality to the Norwegian citizens, by multiple of the agencies in the study. These challenges were both describes as stemming from the legal requirements imposed by the Norwegian government, and other laws such as GDPR.

Legal requirements either requested by the Norwegian government or required by Norwegian law are described in \autoref{sec:legal} to create challenges in the assurance of quality in the Norwegian public sector. The challenges are described as (I) resources only covering legal requirements, (II) GDPR, and (III) laws unfit for digitisation. It can be argued that these challenges are changing in magnitude for software quality assurance, as the public sector is obliged to address these legal requirements \cite{jc_2010}. As the legal requirements are susceptible to changes from political change and cycles, the priorities in legal requirements changing significantly with each new administration \cite{jc_2010}.

%\autoref{sec:legal} Skatteetaten seems to be the agency in this study, where its direction is software development are most driven by new policies and laws decided by the Norwegian government. Describing that Skatteetaten only have the resources to implement these legal requirements, without much left for other quality aspects such as usability. And some processes at Skatteetaten not having enough resources to be automated by software. It would therefore seem that the Norwegian government are not allocating enough resources other than the minimum when deciding on new laws and policies. Ending up in the agencies themselves not being able to create software with the desired quality and fulfilling needs that the user would expect from that software.

%\autoref{sec:legal}Laws such as GDPR are described as preventing agencies to deliver high quality software to the Norwegian citizens. Such as the case with Skatteetaten and the Brønnoysund Register centre described in \autoref{sec:legal}. Where both agencies could combine the user data in one interface, providing software of better usability than existing solutions. However in this case laws on data-sharing blocked this, even tough the agency's would not use each others data about the user internally and the user already owning the data they were displaying. It was only displayed in a new format, in a single interface instead of two separate interfaces. It would therefore seem like the laws for data sharing which is set to defend the Norwegian citizens might end up hurting the citizens. Due to it preventing the development of higher quality software than existing software. And preventing the user to gain better information in the Norwegian public sector, due to better usability being prevented.

%\autoref{sec:legal}It could therefore be seen that no matter how much the Norwegian government wish to improve the digital services provided by the agencies. The laws and policies which lay in their foundation have to be digitisation friendly. Thus making further efforts on digitising the laws and policies even more challenging and resource intensive.

%\autoref{sec:legal}These reflections on software quality from the participants can seem to contradict somewhat. It is said that for a piece of software in the Norwegian public sector to be of high quality, it needs to adhere to legal requirements. However also stating that the same legal requirements are creating challenges to provide software which is of high quality. This contradicting views on high software quality can create false expectations of the software, leading to potential resources wasted, which could instead have been used on improving the quality of other software. Reducing the problem of not having enough resources to meet the unclear standards of the software.

\section{Cooperation Between Agencies}
One of the current digitisation goals of the Norwegian public sector, set by the Norwegian government, is to increase cooperation between agencies through a shared digital platform \cite{r_2019}. However, it could seem like the agencies are lacking in resources to provide software of high quality to their own domain. To then use the already precious resources on co-operation, could lead to providing software of lower quality in their own domain. 

Lack of resources was described as an issue, as agencies are only able to achieve what is legally required. It would therefore seem that if the Norwegian government want more cooperation between the agencies, more funding and resources is required. Or obtaining exemptions from certain legal requirements to work on digital cooperation. However, obtaining exemptions could be argued as less desirable, as it can decrease the value the agencies provide to Norwegian citizens.

None of the participants mentioned that cooperation through a shared digital platform would increase the value delivered to Norwegian citizens by the agencies. It could therefore be worth to investigate if these goals by the Norwegian government stem from a real problem. Or if the digitisation goals of the Norwegian government are the result of wishful thinking by leaders who are positioned far away from the agency's users. 

It could also be worth investigating if digital cooperation is something which will yield any value to Norwegian citizens. It is the perception from the interviews in this study that the agencies have moved away from large applications which cover a wide range of needs. To smaller applications which cover more specialised needs. Something described as giving higher value to the Norwegian citizens, and by the participant's definition, increasing quality. It could therefore seem that developing large technical solutions that cover a wide range of needs mean stepping back in time.

The digital platform that the Norwegian government might wish to possess, is something resembling the NAIS platform, used by NAV. However it is important to remember that the NAIS platform is built specifically for NAV applications and NAV's domain. It could therefore be argued that if a digital platform across the Norwegian public sector wishes to be successful and deliver services of quality to Norwegian citizens, it needs to be as specialised as the NAIS platform. 

And what organisation with enough domain knowledge of all the agencies in the Norwegian public sector are going to be responsible for the development and maintenance of the digital platform? The Norwegian digitisation agency might be a potential candidate, as they already develop shared solutions for the Norwegian public sector. Yet might not have enough resources and IT staff to develop and maintain a digital platform of the scale needed to cover the Norwegian public sector. 

All this criticism could also be somewhat unjustified, as recent literature has explained that DevOps is expected to bring the different agencies in the public sector closer, increasing knowledge culture and collaborative work \cite{mm_2021}, which is shown to increase software quality \cite{smm_2018}. The use of DevOps in the Norwegian public sector seems to be quite recent, and might not have time to be fully established in all the agencies using it. Therefore it might be the case that the Norwegian government preemptively released their strategy, without the Norwegian public sector being ready for its implementation and use.

At the time of this study, a new digitisation strategy is beginning to be planned. It will therefore be interesting to follow if the Norwegian government follow the same strategy, with the mentioned challenges. Or choose a different strategy with a lower level of ambition and a more detailed description of the problem trying to be solved. Leaving the agencies of the Norwegian public sector with a better understanding of what is to be expected from the Norwegian government, other than legal requirements.

%\textcolor{red}{virker som etatene fortsat har mer enn nok å tenke på selv når det kommer til å levere tjenestene sine, som er en viktig definisjon for at noe er av kvalitet -> hvis de da i tillegg skal begynne å sammarbeide kan de fort miste fokuset på å levere kvalitet til sine egne brukere i sin egen etat}

%\textcolor{red}{dette målet er også enda mer uåpnålig med tanke på at det knapt er ressurser til hva som er "legaly required" -> dermed hvis etatene faktisk skal opnå det de ønsker med et teknisk fellesarena for samarbeid -> enten så må de får mer ressurser, eller så må etatene få fritak fra andre legal requirements til å gjennomføre ønsket -> dette kan gi dårligere verdi til brukerne til etatene, som gir dårligere kvalitet}

%\textcolor{red}{var verken forskning eller noen av de som ble intervjuet som nevnte at det at de øker samarbeidet gjennom tekniske løsninger nødvendigvis gir mer verdi for brukeren -> er dette et faktisk behov som har kommet fra reele problemstillinger, eller er det ønsketekning om hva man tror er bra, fra noen som ikke sitter nærme nok brukerne.}

%\textcolor{red}{er det hensiksmessig at man skal sammarbeide tettere, vil det faktisk gi verdi til brukeren? -> etatene har gått bort fra store applikasjoner som dekker større behov, til mindre og mindre applikasjoner som dekker et spisset behov -> noe som de tror har økt verdien for brukeren, og dermed kvaliteten på det de lager -> det kan da bli sett på som å gå tilbake i tid ved å lage større fellesarenaer for sammarbeid, teknisk sett, uten den samme fleksibiliteten som man har i dag}

%\textcolor{red}{en platform som er lik en slik plattform som kanskje regjeringen ønsker er NAIS -> men her er det viktig å huske at denne er laget spesifikt for NAV applikasjoner og NAV's driftsmiljø, altså NAV sitt drift domene -> hvis en slik felles platform som ønsker skal kunne være vellykket, og levere kvalitet, så må den kunne levere så spesifikt som NAIS gjør for NAV -> hvem skal passe på at dette skjer? hvem har god nok domene-kunnskap over alle Norges etater til at dette er mulig? og hvem skal vedlikeholde og drifte dette? -> DigDir kan jo være en kandidat, men spm'ålet er om de har nok kapasitet til å først og fremst bygge dette, samt vedlikeholde og endre det for alle endringene i Norsk offentlig sektor -> som f.eks endring i struktur av etater, etater opprettes og legges ned, nye lover for etetaene blir satt og fjernet -> samle sammen alt dette og en dose god gammeldags politikk så har du en fin suppe med problemer som må fikses for at det kan skje}

%\textcolor{red}{når denne rapporten har blitt skrevet så er tidsrammen på digitaliseringsmålene fra regjeringen gått ut (2018-2023) -> ser ikke ut som de har klart det i løpet av den tiden -> blir spennende å se om regjeringen kommer til å fortsette mtp probleme nevnt over, eller om de kanskje jekker litt ned ambisjonsnivået og splitte opp problemene de prøver å løse -> det er det etatene prøver å gjøre for å skape programvare av kvalitet, splitte opp problemene og løse de iterativt, istedenfor alt i en gjafs}

%\textcolor{red}{DevOps is expected to bring the different agencies in the public sector closer, increasing knowledge culture and collaborative work \cite{mm_2021}, which is shown to increase software quality \cite{smm_2018}.}


\section{Researchers Perspective}
Even though the development environment of NAV has grown significantly with in-house developers, it can still be a challenge to acquire the necessary development resources. And when a development team is not receiving enough resources, only the legal requirements and core needs of the users are met. Resulting in shortcuts in testing and code quality being made to meet the lowest criteria of the software.

NAV has also made efforts to make it easier to create NAV applications of high quality by creating tools specifically for the development of NAV applications. However, when not developing a typical NAV application, it can lead to the specialised assisting tools not being helpful. Resulting in the developers re-inventing the wheel to make the application fit their specific domain, using the already limited resources on tasks that do not directly benefit the Norwegian citizens.

These tools can increase the already steep learning curve for inexperienced developers at the agency. Platforms such as NAIS are simple to use when in possession of knowledge in DevOps and software development, as most time-consuming tasks have been automated and abstracted. However, without the knowledge of how such platforms function, it is difficult to understand what value is to be gained from using tools such as NAIS. Resulting in a large amount of resources being spent learning to use the platform, instead of the platform saving resources. The researcher described that he worked a year as a software developer at NAV, before having a good knowledge of how to use NAV's tools to create software of high quality. Due to the researchers not having the required knowledge in DevOps and software development. 

%\textcolor{red}{selv om "utviklings-muskelen" til NAV virker stor og sterk, så er det fortsatt vanskelig å få ressurser -> når man ikke da har nok ressurser i teamet så blir det mest fokus på det som er av "legal requirements" og kjernebehov for brukeren -> ender med at ting som tester og til en viss grad god kode er noe som man tenker litt mindre på}

%\textcolor{red}{selv om NAV har kommet langt, så er det forstatt litt å gå -> hvis man ikke lager en spesiell "NAV applikasjon", som f.eks et saksbehandlingssytem, så finner man fort at ut en del av hjelpeverktøyene ikke hjelper så mye -> da blir du plutselig sittende der og må oppfinne hjulet på nytt selv og prøve å tilpasse applikasjonen din til hjelpeverktøyet -> dette tar mye tid og ressurser, og da har man som regel kuttet på programvarekvalitet som tester osv}

%\textcolor{red}{hjelpeverktøyene kan gjøre det enda brattere for en junior å gi verdi for etaten -> NAIS er veldig bra, når man kan alt av DevOps og platform, og vet hvordan det funker, sidne det abstraherer bort alt det kjipe -> men, hvis du ikke kan det som er abstrahert bort, så vil du ikke skjønne hva du tjener på å bruke en slik platfor, og ikke minst, du vet ikke hvordan du skal bruke den for å gjøre det lettere å lage en "NAV applikasjon" -> dermed blir veldig mye tid og ressurser brukt på å lære seg noe som egt skal spare ressurser og gjøre utviklings-hverdagen smudere. Tok meg 1 år før jeg var komfertabel som nyutdannet med NAIS -> hjelpeverktøyene krever en vis forskunskap for at de skal hjelpe i å bygge ting av kvalitet, og den forkunskapen må etaten forsikre seg at alle som skal bruke de har}

%\textcolor{red}{hjelpeverktøyene kan bli en hvilkepute for teamet i at de ikke tar godt nok ansvar i visse aspekter ved kvalitetssikring -> man tenker bare at hjelpeverktøyet fikser, også trenger man ikke å tenke på det lenger -> men så dekker kun hjelpeverktøyet deler av kvalitetssikringen, ikke alt}

%\textcolor{red}{man sammenligner seg med de gamle applikasjonene, og tenker at alt som er bedre en de er av høy kvalitet, bare fordi man hører at "gammelt er dårlig, nytt bra" -> man blir blind på at enten: (I) det gammle fortsatt er av høy kvalitet, kanskje ikke nødvendig å bytte ut og (II) det nye som blir laget, er laget med dårlig kvalitet uten spesielle sjekker som bekrefter kvaliteten, så ender man med å bytte ut det gamle gode, med det nye dårligere -> slik som man har sett litt med helseplatformen}

%\textcolor{red}{selv om etater som NAV har gjort en stor endring på å gjøre det smudere å utvikle programvare, så merken man fortsatt at man er et mål for Norsk politkk -> "alt" handler om årlige budsjetter, og det blir en kamp hvert år om å få nok ressurser -> hvis et større prosjekt har mindre framgang en det som ønsker, så blir det mye styr, gjerne utenifra -> da blir det gjerne eksterne revisjoner, som er beskrevet som egt ikke gjør noe med kvaliteten, bare tar tid}

%\textcolor{red}{det å jobbe i Norsk offentlig sektor kan være noe som blir sett ned på -> gjør at man til tider ikke føler seg verdsatt for innsatsen man gjør for å forbedre hverdagen til innbyggerne av -> når man ikke blir verdsatt så gidder man ikke gi full innsats i jobben sin som utvikler -> fører som regel til at man ikke gidder å sette inn innsats i å kvalitetssikre det man lager, man gjør bare "bare minimum"}

\section{Limitations}
As this study is a case study with interviews as the main data generation method, the study could risk generalising the results and its discussion to the entire Norwegian public sector. This as only 3 of the agencies in the Norwegian public sector were included, and only a few employees of these agencies being interviewed.

The participants in this study have also signed non-disclosure agreement's, not to share data about the users of the services being provided by the particular agency. Leading to either the employees not sharing data relevant for the study, or the researcher not revealing data which could be considered sensitive to the particular user or employee. 

The researchers role as a software developer in the Norwegian public sector, could be both positive and negative for the validity in the data collected in the study. It could be that the participants would be more open to share sensitive data. While at the same time being afraid to share the correct data, due to being scared of offending the researcher by criticism through the sharing of data.

Bias by the researcher, due to their position as a software developer in the Norwegian public sector can challenge the validity of the results presented. Due to the researcher having a conflict of interest in how the results is presented and discussed, in a way that could favour the agency the researcher is employed at, or the Norwegian public sector at large.

The lack of scientific literature of software quality assurance in the context of the Norwegian public sector could lead to validity challenges of the results presented in this study. As the results and arguments conducted have limited external data to be validated with. It could be that there is scientific literature relevant, however the limited time and resources for this study has halted any further searching of relevant scientific literature.

%\textcolor{red}{
%\begin{itemize}
    %\item Det er mange etater i Norsk offentlig sektor, det at jeg kun har vært hos 3 kan gi et skevt bilde av hele Norsk offentlig sektor, kan hende at jeg har generalisert litt mye
    %\item Det at jeg har intervjuet relativt få i hver etat, kan også hende at jeg har generalisert litt mye
    %\item Det at jeg er utvikler i Norsk offentlig sektor, kan slå begge veier for interview participant'sa. De kan bli redde for å si det de mener siden jeg er i samme bransje, eller at de kan bli mer åpne om å snakke med meg.
    %\item Det at jeg er utvikler i Norsk offentlig sektor kan hende at jeg er biased, og lener meg mer mot å snakke etatene opp, spesielt NAV
    %\item Lite forskning som baserer seg på programvarekvalitet i Norsk offentlig sektor (tror jeg, må søke litt mer) -> Validity issues
    %\item De ansatte i Norsk offentlig sektor har skrevet under taushetsavklaring -> kan være info som de vil ha sagt, som de ikke kan si -> eller informasjon som jeg har latt være å ta med, for å opprettholde etikk
%\end{itemize}}


