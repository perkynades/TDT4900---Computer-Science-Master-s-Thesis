\chapter{Discussion}
As mentioned by multiple participants in the interviews, it is important to consider the context of the software, when assuring quality in that software. This could be seen as important for each agency in the Norwegian public sector, and the whole sector itself. Some aspects regarding software quality and its assurance such as DevOps and product teams in the Norwegian public sector are true for the whole software development industry. While other factors such as allocation of resources and it's effect on how software is developed and it's quality seems to be a somewhat unique for the Norwegian public sector.

\section{Practices}
\textcolor{red}{Snakke litt om de generelle metodene som blir brukt for å sikre programvarekvalitet}

\cite{smm_2018}

Security -> \cite{sh_2018}

\section{DevOps and Product-Teams}
\textcolor{red}{Snakke om hvordan etatene har rigget seg rundt devops og produktteam for å sikre kvalitet}

\cite{am_2020}\cite{smm_2018}\cite{ml_2022}\cite{mm_2021}\cite{dsc_2019}

\section{Financing and Resources}
\textcolor{red}{\begin{itemize}
    \item Hvordan satens prosjektmodell skaper dårlig kvalitet
    \item Mattilsynet har stopped å bruke statens prosjetmodell. Er det håp for andre etater?
    \item Måten budsjetter i offentlig sektor er satt opp kan føre til dårlig kvalitet
    \item Lite ressurser generelt fører til dårligere kvalitet
\end{itemize}}

\cite{sh_2018}\cite{csw_2011}

\section{Legal requirements}
No research mentioned legal requirements for a piece of software as an important factor to influence the quality of the software. However it was described as a challenge for delivering software of high quality to the Norwegian citizens, for multiple of the agencies in the study.

\textcolor{red}{Snakke om at lovene som skal digitaliseres i offentlig sektor er lite digitaliseringsvenlige, som gir utfordringer for kvalitet}

\section{Cooperation between agencies}
\textcolor{red}{Her må jeg snakke om hva som driver utvikling i etatene, og hvordan dette ikke passer inn med regjeringens mål om å en "felles samarbeidsplatform på tvers av etater for å gjøre tjenstene til etatene bedre"}

\section{Researchers Perspective}
\textcolor{red}{Her må jeg snakke om min egen erfaring som utvikler i Norsk offentlig sektor}

\section{Limitations}
\textcolor{red}{\begin{itemize}
    \item Det at jeg bruker case study
    \item Det at jeg er utvikler i Norsk offentlig sektor, kan slå begge veier for interview participant'sa
    \item Lite forskning som baserer seg på programvarekvalitet i Norsk offentlig sektor (tror jeg, må søke litt mer) -> Validity issues
\end{itemize}}

\textcolor{red}{Skal jeg få inn litt om etikk her? Er jo litt fare at jeg er både utvikler i offentlig sektor og forsker. Kan gi "validity" problems. Er også problemer med at de andre ansatte kan lettere gi med sensitive dokumenter/opplysninger, som ikke er egnet å publiseres (større ansvar for det etiske pga min posisjon)}
