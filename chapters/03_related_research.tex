\chapter{Related Research}

\textcolor{red}{Snakke her om at det ikke dukket opp mye forskning angående kvalitetssikring av programvare i offentlig sektor?}

\cite{nbm_2015} Snakker om problemer rundt det å ha et dedikert testing team (langt unna i Kina) → Kan være interessant å finne ut om hvordan dette er i offentlig sektor for å så senere kunne unngå samme problemer. Kanskje når systemene til offentlig sektor “samles” så trengs det et felles testing team for de forskjellige systeme, siden de skal teste de samme “kvalitets atributtene”, dermed kan man lære av dette paperet?

\cite{dsc_2019} Snakker om kvalitetssikring i DevOps og hvilke “concenrs” som er for testere → Blir disse “concerne” tatt hånd om i norsk offentlig sektor når programvarene går nærmere hverandre.

\cite{ja_2016} Lite forskning på kvalitetssikring i utviklingen av programvare i DevOps miljøer → Vi trenger mer forskning siden etatene i offentlig sektor pusher mot DevOps.

\cite{mk_2017} Lite forskning og data på hvordan utviklere kvalitetssikrer sin software → Trenger mer forskning.

\cite{am_2020} Det vises til at DevOps fører til bedre kvalitet i programvare → Brukes DevOps i offentlig sektor? Dette må vi finne ut. Sier også at det må forskes mer på DevOps sin effekt på kvalitetssikring av programvare → Kanskje kvalitetssikringen er forskjellig i offentlig sektor.

\cite{smm_2018} Viser til at DevOps fører til økt kvalitet i programvare → Bruker offentlig sektor DevOps, har DevOps faktisk gitt bedre kvalitet i software for offentlig sektor. Kultur er viktig for å bygge programvare med høy kvalitet → Har offentlig sektor kultur for dette?

\cite{sh_2018} Lister opp mange problemer som kan forekomme i kvalitetssikringene av software → har offentlig sektor de samme problemene? Gjør de noe for å unngå disse problemene?

\cite{mm_2021} DevOps i offentlig sektor øker kunnskap og samarbeid mellom ansatte → Annet paper sier at økt kunnskap og bedre kultur øker kvalitet i software

\cite{ml_2022} Ved å bry seg om prosseser som øker kvaliteten av programvare, så minker man også kostnadene i å utvikle ny programvare, som ved å bruke DevOps. (Men dette er kanskje ikke rett på hvorfor leseren skal bry seg of kvalitetssikring av programvare spesifikt)

\cite{csw_2011} Å ikke teste førere til høyere kostnader og lavere sikkerhet → Mulighet til å spare skattepenger som kan bli brukt på bedre ting. Dataen til innbyggerne er verdifult, trenger bra sikkerhet.