\chapter{Related Research} \label{sec:related_research}
Previous research has sought out to understand how the quality of software is assured, and the relationship between different software development methodologies in organisations and the quality of their software and what benefits/disadvantages this brings. However, not a lot of this research is done with the Norwegian public sector as a context, nor any public sector. Therefore this paper intends to compare this previous research with the results of this paper to understand how the Norwegian public sector assures quality in it's software. 

\textcolor{red}{Få inn \cite{mb_2020} og \cite{ad_2013}? Ja.}

%\textcolor{red}{
%\begin{itemize}
    %\item Continious software testing. (Skal denne med?) \cite{nbm_2015}
    %\item DevOps software quality assurance \cite{dsc_2019} \cite{ja_2016} \cite{am_2020} \cite{smm_2018} \cite{mm_2021} \cite{ml_2022}
    %\item Expected results of DevOps in the public sector \cite{mm_2021}
    %\item Quality assurance by testers/develoeprs in DevOps \cite{dsc_2019}
    %\item General Testing \cite{mk_2017}
    %\item Factors for software quality \cite{smm_2018} \cite{ml_2022}
    %\item Problems in software quality assurance \cite{sh_2018}
    %\item Benefits of Software Quality Assurance \cite{csw_2011} \cite{sh_2018}
    %\item Testing teams \cite{nbm_2015} \cite{dsc_2019}
%\end{itemize}
%}

The most recent factor for assuring quality in software is the implementation of DevOps in an organisation. This due to the ability to combine continuous integration with automatic testing \cite{am_2020}\cite{smm_2018}\cite{ml_2022}. DevOps also help with assuring quality as it increases co-operation and knowledge sharing between employees, leading to an overall improvement of a "quality culture" in the organisation \cite{smm_2018}\cite{mm_2021}. However it was stated that the effects of testing in DevOps environments had not been studied systematical by scientific literature \cite{dsc_2019}\cite{ja_2016}.

No papers mentioned any specific effects on software quality by introducing DevOps to the public sector, however it was mentioned the expected benefits of introducing DevOps to the public sector \cite{mm_2021}. DevOps is expected to bring the different agencies in the public sector closer, increasing knowledge culture and collaborative work \cite{mm_2021}, which is shown to increase software quality \cite{smm_2018}.

It is also shown that introducing DevOps to an organisation can hinder the assurance of software quality if not addressed to properly \cite{dsc_2019}. The product teams have to rethink their roles and responsibilities towards quality assurance, by shifting from specific roles to raising the quality assurance competence of all members in the product team \cite{dsc_2019}. The product teams have to take an end-to-end responsibility for the content being produced, meaning that the entire product team must be involved in the quality assurance of the content in its entire life \cite{dsc_2019}. It is emphasised that continuous testing is not the same as test automation, and manual testing is still needed to achieve quality \cite{dsc_2019}.

Several other factors which are related to DevOps, but not necessarily directly connected is stated to be important in improving software quality. Automation is stated as an important factor in increasing software quality \cite{smm_2018}. Culture is also stated as important, due to it boosting software quality through integration, evaluation and sharing of knowledge \cite{smm_2018}.

When striving for high software quality, there are several challenges in the process of assuring software quality. (I) Software requirement challenges, (II) Stakeholders perspective challenges, and (III) General challenges \cite{sh_2018}. Software requirement challenges concerns challenges related to the project itself and effects of software quality which are not visible until after the projects implementation or maintenance period \cite{sh_2018}. Stakeholder perspectives challenges concerns the challenges which the different stakeholders might have when asserting quality in software \cite{sh_2018}. General challenges concerns challenges in assuring quality in software which are common for all organisations involved in creating and delivering software \cite{sh_2018}.

It is shown that assuring quality in software reduces costs and increases security \cite{csw_2011}\cite{sh_2018}. Continuous focus on quality and testing in the development of software reduces costs, as the cost of fixing bugs after the software is delivered to the customers is higher than before \cite{csw_2011}\cite{sh_2018}. This also implies to security, as weak security of software can have larger consequences and costs to fix if discovered after the software delivered to the customer \cite{csw_2011}. 

%Software quality assurance in virtual teams can be difficult to implement and can hinder the development velocity of software. A case where the development team were based in Norway and testing team based in China showed that such a setup can have issues in software quality

%\cite{nbm_2015} Snakker om problemer rundt det å ha et dedikert testing team (langt unna i Kina) → Kan være interessant å finne ut om hvordan dette er i offentlig sektor for å så senere kunne unngå samme problemer. Kanskje når systemene til offentlig sektor “samles” så trengs det et felles testing team for de forskjellige systeme, siden de skal teste de samme “kvalitets atributtene”, dermed kan man lære av dette paperet?

%\cite{dsc_2019} Snakker om kvalitetssikring i DevOps og hvilke “concenrs” som er for testere → Blir disse “concerne” tatt hånd om i norsk offentlig sektor når programvarene går nærmere hverandre.

%\cite{ja_2016} Lite forskning på kvalitetssikring i utviklingen av programvare i DevOps miljøer → Vi trenger mer forskning siden etatene i offentlig sektor pusher mot DevOps.

%\cite{mk_2017} Lite forskning og data på hvordan utviklere kvalitetssikrer sin software → Trenger mer forskning.

%\cite{am_2020} Det vises til at DevOps fører til bedre kvalitet i programvare → Brukes DevOps i offentlig sektor? Dette må vi finne ut. Sier også at det må forskes mer på DevOps sin effekt på kvalitetssikring av programvare → Kanskje kvalitetssikringen er forskjellig i offentlig sektor.

%\cite{smm_2018} Viser til at DevOps fører til økt kvalitet i programvare → Bruker offentlig sektor DevOps, har DevOps faktisk gitt bedre kvalitet i software for offentlig sektor. Kultur er viktig for å bygge programvare med høy kvalitet → Har offentlig sektor kultur for dette?

%\cite{sh_2018} Lister opp mange problemer som kan forekomme i kvalitetssikringene av software → har offentlig sektor de samme problemene? Gjør de noe for å unngå disse problemene?

%\cite{mm_2021} DevOps i offentlig sektor øker kunnskap og samarbeid mellom ansatte → Annet paper sier at økt kunnskap og bedre kultur øker kvalitet i software

%\cite{ml_2022} Ved å bry seg om prosseser som øker kvaliteten av programvare, så minker man også kostnadene i å utvikle ny programvare, som ved å bruke DevOps. (Men dette er kanskje ikke rett på hvorfor leseren skal bry seg of kvalitetssikring av programvare spesifikt)

%\cite{csw_2011} Å ikke teste førere til høyere kostnader og lavere sikkerhet → Mulighet til å spare skattepenger som kan bli brukt på bedre ting. Dataen til innbyggerne er verdifult, trenger bra sikkerhet.