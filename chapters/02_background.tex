\chapter{Background}

\section{Norwegian Public Sector}
The Norwegian public sector is a collective term for the state administration and the municipal administration in Norway. It is an entity which include public goods and government services, such as military, infrastructure, public education, and health care. The public sector differs from the private sector, as it does not concern privately owned organisation, nonprofit organisations or households \cite{os_snl_2022}\cite{ps_wiki_2023}.

\textcolor{red}{Skal jeg også skrive litt her om de forskjellige prosjektene som jeg snakker om i resultat biten}

\textcolor{red}{Forklare litt forskjellige prosseser her? Som f.eks ROS analyse?}

\subsection{Norwegian Labour and Welfare Organization (NAV)}
The Norwegian Labour and Welfare Organisation, or NAV, is a central agency in the Norwegian public sector which provides social and economic security to the citizens of Norway, while encouraging a transition to activity and employment \cite{nav_r_2023}. NAV is partnered with the municipalities in Norway, and has about 22 000 employees, where about 15 500 is employed to the Norwegian state, and 6 500 employed in the municipalities \cite{org_nav_2023}. 

\subsection{Norwegian Tax Administration (Skatteetaten)}
The Norwegian Tax Administration, or Skatteetaten, are responsible for ensuring that taxes and other claims in Norway are correctly assessed and paid. It also ensures that the Norwegian National Registry is up to date \cite{skatt_r_2023}. As of 2019, Skattetaten has about 6 300 employees \cite{skatt_r_2023}. 

\subsection{Norwegian Food Safety Authority (Mattilsynet)}
The Norwegian Food Safety Authority, or Mattilsynet, are responsible for promoting animal and human health in Norway by supervising the different food chains in Norway, all from the farms or fisheries, to the food Norwegians eat from said farms and fisheries \cite{mat_r_2023}. As of 2023, Mattilsynet has about 1 300 employees, stationed in offices all regions of Norway \cite{org_mat_2023}.

\textcolor{red}{
\begin{itemize}
    \item Få inn statens pensjonskasse her?
    \item Få inn norsk tipping her?
    \item Få inn Entur her?
\end{itemize}
}

\textcolor{red}{
\begin{itemize}
    \item Få snakket om NAIS? Kanskje under NAV?
\end{itemize}
}

\subsection{Digitalisation in the Norwegian public sector}
The government of Norway has in the last decades has an increased focus on digitalisation of the Norwegian public sector, with the most recent digitalisation strategy from 2019-2015 \cite{r_2019}. The main goal of the digitalisation strategy is for different actors in the Norwegian public sector to co-operate by the development of shared digital eco-systems. The goal of this development project is to aid the development of user-centred system development for a more effective and coordinated exploitation of the Norwegian public sectors IT-systems \cite{r_2019}.

\section{Software Quality}
As stated in \autoref{sec:motivation}, software quality is defined as how well a piece of software is measured against a chosen set of non-functional requirements, chosen based on the software's functional requirements \cite{iso_25010:2011}. 

\subsection{Non-functional requirements} \label{sec:non_functional_requirments}
Non-functional requirements are defined to when it can be determined that a piece of software delivers value, hence the non-functional requirements are chooses based on what value the piece of software is supposed to generate \cite{iso_25010:2011}. These are some examples of the most common non-functional requirements:

\begin{itemize}
    \item \textbf{Performance} - The amount of useful work accomplished by a piece of software \cite{performance_wiki_2023}.
    \item \textbf{Scalability} - How a piece of software can handle a growing and shrinking amount of work by adding and reducing resources \cite{scalability_wiki_2023}. 
    \item \textbf{Portability} - The effort required for a piece of software to run on different devices \cite{portability_wiki_2023}.
    \item \textbf{Reliability} - How long a piece of software can function without failure \cite{reliability_wiki_2023}.
    \item \textbf{Maintainability} - How easy is a piece of software to repair or update \cite{maintainability_wiki_2023}. 
    \item \textbf{Security} - How protected is a piece of software from attacks by malicious actors \cite{security_wiki_2023}.
    \item \textbf{Usability} - How a piece of software provides conditions for its user to perform tasks safely, effectively, and efficiently \cite{usability_wiki_2023}.
\end{itemize}

\subsection{Software Quality Assurance}
As stated in \autoref{sec:motivation}, software quality assurance is the planning, controlling, and executing of processes which measure non-functional requirements or other defined quality standards, such as the ones above, which are defined by a product team or at the organisational level \cite{ieee_730_2014}\cite{sqa_wiki_2023}.

\section{Software Testing}

\section{DevOps}

\textcolor{red}{Disse må sees over om de er riktige/skal være her}
\subsection{Continuous Integration}

\subsection{Continuous Testing}

\subsubsection{Automatic Testing}


\subsection{Continuous Deployment}