\chapter{Results}

\textcolor{red}{
\begin{enumerate}
    \item Se over hva som må forklares i Background
    \item Se over hva som faktisk passer inn med promplemstillingen
\end{enumerate}
}

\section{Perspectives of Software Quality}
\subsection{Definition}
The results from the interviews in the case study show that the participants have different perspectives on software quality and software quality assurance.

The word \textit{"quality"} is a word that is difficult to describe, as it is as word which people think they understand and know what is, however nobody quite know how to define it. People thing that something of high quality is something which will last long, something which will give value, something which is good. The words \textit{"software quality"} is something which is not often used in the Norwegian Public Sector.

\textit{"Software quality"} has also been described as a word which is difficult to give a single definition of. One definition of software quality is the ability for a piece of software to be easily modified or adapted as time passes. NAV's system for disbursing social securities to the Norwegian citizens and is nation-bearing for the Norwegian society. It is a system which is difficult to modify and has a limited test coverage. Yet it as solved a problem for over 40 years. One definition would say that this software is of low quality, while another definition would say it is of high quality due to it being a robust system for over 40 years.

%\subsection{Definition}
%\textit{Participant \#2} explaining that the word "quality" is a difficult word to describe. It is a word which everyone think they understand and what it is, but nobody really know how to define it. Many think that something of high quality is something which will last long, something which give value, something which is good.

%\textit{Participant \#2} also explaining that he does not hear many use the words "software quality" in the Norwegian public sector.

%\textit{Participant \#6} explaining that software quality is the ability for software to change over time. One of NAV's systems for disbursing social securities is about 40 years old and is nation-bearing for Norway. However it is very difficult to change, and has a limited amount of tests. Yet it has solved a problem for over 40 years, so its of good quality in that sense, but yet it so bad in other ways.

\subsection{Context}
What context the software is situated in, is important for how it's quality should be assured. If creating software for medical machines, which can endanger humans, no faults can be allowed. Hence methods needed to assure quality could be manual and time consuming testing with a high degree of coordination. For software with less consequences of faults, automatic testing as a software quality assurance method can be acceptable.

The context deciding what software quality assurance methods should be used are also important for other situations. If a piece of software is only intended to run once, mutability is not important, the most important property being that it solves the particular problem. In this situation a heavy focus on software quality by the development team can lead to unwanted complexity in the software, which do not help the software in solving it's intended problem.

At NAV each piece of software decides how the development teams assures quality. The system for disbursing social securities and the system for disbursing sick pay to not use the same quality assurance methods. Some systems use manual and time consuming test-processes, while others use automatic testing. Some systems are combining both manual and automatic testing. Some systems are modified and deployed every third month, while other systems are modified and deployed 20 times a day.

%\subsection{Contextual}
%\textit{Participant \#3} explaining that what quality assurance method used is based on the context. Testing could be done manually, with a lot of coordination, or it could be done automatically with computers. Performing automatic tests will be much quicker, but some faults may not be detected. This can be acceptable for creating some software, however creating medical devices such as eye-lasers might need more rigorous testing, as you cannot let faults get through. So it depends on the context of what is being created.

%\textit{Participant \#6} explaining that software quality is defined based on the context. If a piece software is only meant to be used once, then ability to change the software is not important, then the most important thing is that it just works. A development team can be more focused on quality than the solution itself. As only focusing on quality can lead to a lot of complexity which is not necessarily needed for the software to solve the problem it is intended to solve.

%\textit{Participant \#6} also explaining that the systems such as the system disbursing social securities and sick pay at NAV do not use the same methods to assure software quality. Some have long manual test-processes, while others have automatic testing, and some are in-between with manual testing which are trying to move to automatic tests. Some system have changes which are deployed every third month, while other deploy changes 20 times a day. 

\subsection{Non-functional Requirements}
Associations to non-functional requirements are described when it comes to the definition of software quality, that it relates to usability, maintainability, performance, security, or the mutability of the software's architecture. Associations to manageability of the software being created is important for the software quality. These non-functional requirements are also described as being the most important in the technical quality of a piece of software.

At NAV, non-functional requirements has been described as quality requirements or quality properties. In another part of the Norwegian public sector a team of software once was delivered a list of 250 such requirements or properties which the system had to maintain. Controlling that all 250 requirements are being followed to be difficult, and actually implementing all 250 requirements into the system being even more difficult.

%\subsection{Non-functional requirements}
%\textit{Participant \#1} describing that software quality is related to non-functional requirements, such as usability, maintainability, performance, security, or the mutability of the software's architecture. 

%\textit{Participant \#1} explained that a team of software consultant once was delivered a list of over 250 non-functional requirements which the intended system had to maintain. Such a long list of requirements being difficult to control that the requirements are actually followed, being even more difficult to actually implement the requirements.

%\textit{Participant \#2} describing software quality as a piece of software being made in such a way that it is robust. And that it is made so that it is easy for other programmers to modify, maintain and repair if it is needed.

%\textit{Participant \#4} has similar reflections on non-functional requirements. However also focus on the manageability of the software being created, that it should be long lasting and be able to be modified over a longer span of time. Also describing that a piece of software is not dependant on 1 person, but the whole developing it. 

%\textit{Participant \#5} describing that NAV has called non-functional requirements as quality requirements or quality properties. Something which was done in the past. Also explaining quality as testable or maintainable.

%\textit{Participant \#7} explaining that in the technical quality of software, some non-functional requirements are more important than others, such as manageability, functionality, usability, maintainability being more important to focus on than other non-functional requirements. 

%\textit{Participant \#8} relating to the reflections of \textit{Participant \#2} in that a piece of software should be easy to modify, and that the software's code should be readable and quick to understand how it functions by just reading it.

%\textit{Participant \#9} explained that he had worked with software quality, but measuring it up against the ISO standards. Including measuring it up against non-functional requirements such as usability, security, re-usability etc. 

%\textit{Participant \#11} describing good usability as important in software quality, that the user should feel that a piece of software is of quality, the visual aspect is perceived as attractive. Also describing software quality is that piece of software is stable, secure and robust.

\section{Technical}
\subsection{Security}
In the development process of NAV's new systems for distributing parental benefits, how security has been assured in the system has changed. Initially only specific persons had the responsibility for security, however when the project was complete in 2019, each team had a specific person with knowledge in security. Other projects such as "Perform Project" at the Norwegian Public Service Pension Fund had a more central control of security in its requirements, which the development teams had to take into account.

The requirements for security at Skatteetaten are and has always been strict, due to Skatteetaten's software  being constantly exposed to cyber attacks. This and a general lack of development resources has resulted in Skatteetaten prioritising security over other requirements and functionalities. Meaning that Skatteetaten's users might have to accept software which lacks in usability and functionality, due to Skatteetatens high requirements for security. This focus on strict security requirements are described as being the same for NAV, as it is for Skatteetaten.

One of the methods that Skatteetaten use to ensure security in their systems, is by all events in their systems being logged. Whenever a network-call is made, someone does a direct look-up in a database etc it is audit-logged. This so that if someone both with authorised and unauthorised permissions will be logged and exposed.

\textcolor{red}{Har "ROS-analysis" et mer engelsk navn}

Analysis such ROS-analysis is done to ensure that quality requirements such as security is met at Mattilsynet. However ROS-analysis is also done to ensure that other quality requirements than just security is also met.

%\subsection{Security}
%\textit{Participant \#1} explaining that at the development of the new system for distributing parental benefits at NAV, in the beginning specific persons had the responsibility for security. The project was done in 2019, and at that point each team had a specific person with knowledge in security. 

%\textit{Participant \#1} describing that at the "Perform project" at the Norwegian Public Service Pension fund had central in it project organisation requirements for security, which the development teams had to take into account.

%\textit{Participant \#7} explaining that Skatteetaten takes requirements related to security seriously. Skatteetaten is constantly exposed to cyber attacks, so their requirements on quality related to secure software has always been strict. Thus due to a lack of development resources, means that Skatteetaten has to prioritise security over functionalities for its users, meaning that they will have to make do with something a little simpler or which includes more manual work.

%\textit{Participant \#7} describing that the focus on security is the same at NAV, as at Skatteetaten.

%\textit{Participant \#8} explaining that at Skatteetaten, everything is logged. If someone looks up something, does a network call, we log that person doing it. Audit-logging is also done for direct lookups to a database, so that if you have the correct right's, then you will be logged and exposed.

%\textit{Participant \#9} explaining that at Mattilsynet a lot of ROS analasys is done to ensure quality in secruity, the ROS does also work as a more general ensuring of quality in the software being analysed.

\subsection{Inspection}
It is the belief that many development teams in the Norwegian public sector use code inspection as a tool to find faults in their software. As well as new code have to be inspected before it is allowed to be merge into the code repository. This was done at the "Perform project" at the Norwegian Public Service Pension fund.

When developing a new feature for software at Skatteetaten, at least 2 developers have to be included in a code review process for the developed feature before it is accepted in order to assure that the feature is of quality.

Some are hesitant to the use of code inspection tools to increase software quality, such as a developer code review, due to it being time consuming. When asked to perform a code review, a developer has to stop what they are doing, review the code, give their toughs, send the review and repeat the process until the code is of satisfactory quality. The developer then has to get back to the task he/she was doing before the review. This can lead to a code review taking multiple days. The developers could instead have performed pair-programming, and the code would be reviewed then and there. 

Recently the developers at Mattilsynet has increased their use of static code analysis tools such as SonarQube. A static code analysis of an large old system at Mattilsynet is said to find 145 critical faults. Which enables Mattilsynet to know that these faults exists and can stat processes to fix these faults. However this is in contrast with an alleged study from Microsoft, which is said to conclude that such code inspection tools only find a limited amount of critical faults. Mostly finding faults in code style and not much else.

It is also described that inspection through monitoring and monitoring metrics is a bit weak at Mattilsynet, that it might be to a better standard at other organisations such as Norsk Tipping.

%\textit{Participant \#1} describing that at the "Perform project" at the Norwegian Public Service Pension fund, they used code inspection as a tool to ensure quality.

%\textit{Participant \#1} explaining that some are critical to some inspection techniques, due to it being time consuming. A developer has to stop what they are doing to look at a other developers code, review it, give their thoughts, send in a review, and repeat the process. After the code is approved, the developer then has to spend time on returning to what he/she was working on before, which leaves the whole process to take up to multiple days. If the developers instead had been pair-programming, the code would be inspected then and there.

%When developing a new function, at least 2 people have to look at the developed function, which introduces a QA-process. 

%\textit{Participant \#1} explaining that a study from Microsoft found that there were only a few critical faults that were found in code inspections. Mostly faults relating to code style were found, not so much critical faults.

%\textit{Participant \#1} describing that it is their belief that many teams does code inspections, and in some organisations the code has to be inspected before it is allowed to be merged into the code repository.

%\textit{Participant \#10} explaining that they have increased the use of static code analysis at Mattilsynet. And had recently scanned a large old system by using SonarQube which he believes to have found about 145 critical faults. This enables Mattilsynet to know that these faults exists and that they can start fixing them, which is a huge step from not knowing that they exist. 

%\textit{Participant \#10} explaining that when it comes to inspection through monitoring and monitoring metrics, it is a bit weak at Mattilsynet, the participant further explaining that he is used to a higher standard at places such as Norsk Tipping.

\subsection{Testing}
Allegedly most agencies in the Norwegian public sector use regression testing to ensure software quality in their software. As well as trying to automate as much of the testing as possible. However some of the testing activities are still done manually, such as exploitative testing and involving the customer in the testing activities. 

At the "Perform project" at the Norwegian Public Service Pension fund supporting systems for testing their solution with simulated data were created. While at the development of the new system for distributing parental benefits at NAV, more traditional methods were used to test the system with simulated data.

Writing tests is important when trying to include proper continuous delivery in a software development project. These tests are written to ensure that there is a web of security around the software being create, so that the chance of something faulty being delivered is as low as possible. The chance of faults will never be zero, but it its important to get it as low as possible withing reason. Writing these tests are also important for the ability to more securely delivering smaller and more frequent changes to the software being created.

Skatteetaten have a high degree of automated tests, and there are years since they stopped performing acceptance tests. Skatteetaten has also had to experiment with writing the correct amount of automated tests. If there for instance are too many automatic tests, it becomes a large task to maintain all the tests. Therefore there has been a challenge for Skatteetaten to find the balancing point on what is the right amount of automated tests.

It is described that to be able to modify a piece of software, the software needs to have a high test coverage. A low-level method is first tested using unit-tests, then the whole software unit or application which the method is a part of is tested. If using a microservice architecture where there is interactions and values between the applications, then testing is needed at all levels. This to ensure to ensure that all works as intended and modifications in a piece of software do not lead to any faults. High test coverage is also described as the most important measure to ensure that a piece of software can be modified over a longer period of time.

%\textit{Participant \#1} explaining that most agencies in the Norwegian public sector use regression testing to test their software, and are trying to automate as much as possible in testing. However some testing is still done manually, such as exploitative testing and having the customer involved in the testing activities.

%\textit{Participant \#1} describing that in the "Perform project" at the Norwegian Public Service Pension fund extra systems for testing their solution with simulated data. While at the development of the new system for distributing parental benefits at NAV, more traditional methods were used to test the system with simulated data.

%\textit{Participant \#3} explaining that continuous delivery is something which takes a lot of effort to get working properly, which includes writing tests. Tests are being written to make sure that there is a web of security around the software being created, so that the chance of something faulty being delivered is a low as possible. The chance of faults will never be 0, but trying to get it as low as possible within reason. 

%\textit{Participant \#4} explaining that in order to deploying smaller and more frequent changes to a piece of software, automated quality checks are needed.

%\textit{Participant \#6} explaining that in order to create software that is able to change a lot has to be in place, such as testing. Testing being the most important measure to make sure that a piece of software can change over time.

%\textit{Participant \#7} explaining that Skatteetaten have a high degree of automated tests, and do perform any accepts tests, which were years since last were done. Skatteetaten has also experimented with the use of automated testing, such as the amount of automated tests being written is the correct amount. If to many are written it becomes a large task to maintain all the automated tests. So it has been a challenge for Skatteetaten to find the balancing point on what is the right amount of automated tests.

%\textit{Participant \#8} explaining that to be able to change a piece of software, the software needs to have a high test coverage. A method is tested on the lower level with unit-tests, then the whole unit or application which the method is a part of is tested. If using a microservice architecture where there is interaction and values between the application, then testing is needed at all levels. This to ensure that everything works as intended and changes in a piece of software does not lead to any faults which leads to the software crashing.

\subsection{Technical Debt}
An important measure to increase software quality is the ability to register technical debt. This is important as it enables knowledge in what should be improved in a piece of software, due to it uncovering potential faults and vulnerabilities in a piece of software.

NAV has a large variety of software in their organisation to aid them in their operations. Some of the software is over 40 years old, some are being created from the ground up right now and some in-between. This means that NAV use different methods to modify their software based on the particular piece of software. If the software is for instance 20 years old, a full testing environment with manual tests are needed, creating a lot of overhead prohibiting securing quality in the software. This leads to the need for NAV to modernise their old software, which usually means creating new software to replace the old software, and turning off the old software.

\textcolor{red}{Se om quoten til participant 3 er riktig her}

However NAV is also trying to modernise their old software without creating new software and turning the old software off, instead modernising the old software while still in service. NAV's system for delivering pension benefits was initially create by consultants from 2007 to 2011, and which in 2017 was described as being technical bankrupt. It was in such a bad state as "It took half a year to change a comma," described participant \#3. However by working on and improving the pension benefits system for five year, it is now at the point where changes are deployed daily. Creating modifiable software is important, it does not matter how much testing has been done upfront, the requirements of the system will change from when it was ordered to when it is delivered, and after the users start using it.

The ability for software developing teams to understand the necessity of maintenance and handling of technical debt is an issue. Development teams should use 20-25\% of their time to improve technical debt, as technical debt is the opposite of software quality. The challenge is to make the development teams dedicate that amount of time to technical debt.

There is a challenge at NAV, that there is old systems without automated tests, requiring more manual quality assurance methods. These systems also have a lower rate of deploy, such as the system called ARENA, NAV's system for delivering daily allowance benefits. It's deploy frequency has changed from 4 deploys to 12 deploys a year. The deploy frequency having increased somewhat, but still being in-frequent due to the need for manual quality assurance methods.

Skatteetaten also have old software, and in the re-writing and renewing of this old software is a challenge due to bad quality in the old software. Whereas the new software that Skatteetaten is creating is of good quality. The old software of Skatteetaten is being modernised slowly, and is turned off as they are being modernised by newer software.

Mattilsynet do also have some older systems, however with the new product-teams organisation recently introduced, it has given better focus to these older systems. The new systems created at Mattilsynet are created with better quality, yet that are still issues with bad quality of their older systems when it comes to usability and technical debt.

A Gartner-report on Mattilsynets system "Mats" in 2012 allegedly described that "Mats" will cost Mattilsynet 10 Billion NOK over the next 15 years from 2012. Those 10 Billion NOK does not exist, so Mattilsynet do not have the opportunity to bear the cost of this. The consequences of this is that "Mats" will loose functionality as time goes on just from the world around it changing. "Mats" also has integration's with other agencies in the Norwegian public sector. When a data format from the Norwegian Directorate of Agriculture changed, "Mats" lost functionality, due to Mattilsynet being unable to modify "Mats" to accept the new data format.

%\textit{Participant \#1} explaining that being able to register technical debt can be important for software quality as it enables knowledge in what should be done in the future and having knowledge on potential vulnerabilities in a system.

%\textit{Participant \#3} explaining that NAV has a large variety of software in their organisation. Some software is over 40 years old, while some is being created from the ground up now and all software in-between. And this variety in age of software means that NAV has to have different methods for how changes in its software is done, based on that piece of software. If the software is old, maybe 20 years old, manual testing is needed and a full test environment is needed, thus creating a lot of overhead which prohibits securing quality in the software. This leads to the need for modernisation of the old software, which for NAV means creating new software and turning off the old.

%\textit{Participant \#3} also explaining that NAV is trying to save old systems without turning them off, instead modernising them while still in service. The system for delivering pension benefits was created by consultants from 2007 to 2011, and which in 2017 was described as technical bankrupt. "It took half a year to change a comma". However by working at it and improving for five years and is now able to have changes to the system deployed daily. Creating software with the ability to change is important, it does not matter how much testing has been done upfront, the needs of the system will change from when it was ordered to when it goes live, and after it has gone live.

%\textit{Participant \#4} explaining that maintaining software is an issue. The development teams should use 20-25\% of their time on improve technical debt, as technical debt is the opposite of software quality. The challenge is to make the development teams spend so much time on technical debt.

%\textit{Participant \#5} explaining that NAV has a challenge with old systems without automated tests, requiring more manual quality assurance methods. These systems also have a lower rate of deploys, such as the system for delivering daily allowance benefits, called ARENA. It has gone from 4 deploys a year to 12 deploys. The frequency of deploys has increased some, but are still in-frequent due to the need for more manual quality checks.

%\textit{Participant \#8} explaining that Skatteetaen has some old software, and it is the re-writing and renewing of these systems that is a challenge in bad quality. The new systems that Skatteetaten is creating is of good quality, but the old has some quality issues. So the old software is being modernised bit by bit, and is turned of as they are being modernised.  

%\textit{Participant \#9} explaining that Mattilsynet has some older systems, however with the new product-teams organisation, it has given a better focus on this old software. The new systems are made with better quality, however there are still issues with quality in the older systems.

%\textit{Participant \#10} explaining that a Gartner-report on Mattilsynet's system "Mats" i 2012 explained that the system will cost Mattilsynet 10 Billion NOK over the next 15 years from 2012. That money does not exist, so Mattilsynet cannot let that happen. However the consequences of this is that the system looses its functionality as time goes on, just from the world around it changing. "Mats" has integration with other systems in the Norwegian public sector. A data format from the Norwegian Directorate of Agriculture changed led to "Mats" loosing functionality, due to Mattilsynet being unable to update "Mats" to accept the new data format.

%\textit{Participant \#11} explaining that Mattilsynet has older software which are good in quality when it comes to downtime and other operational requirements. However they are lacking in quality when it comes to technical debt and usability. The new applications of Mattilsynet has better usability and has overall better quality.

\subsection{Measurements}
The system NAV uses in their operations have a strong focus on measurements, being that they are measured in how they behave, where in the system are there downtime and where are there not downtime, are there any response problems, etc. It is up to each team at NAV how they measure their systems when in operations, while it is encourage that they do, to have a good understanding of what state their system is at all times.

It is the belief that the software at Skatteetaten are of few faults. For instance if looking at the process for calculating how much taxes is owed in Norway. First Skatteetaten calculate how much each citizen in Norway owe them, or how much Skatteetaen owe the citizen. Then this amount is sent out to each citizen of Norway, which only takes place once a year. Each time this takes place, Skatteetaten measure faults in the systems responsible for this process. The faults measured then get fixed, and the next year none of these faults appear again. New faults may arise the next year, but the system is still descibed as of high quality due to their measurements.

%\textit{Participant \#5} explaining that there is a strong focus on measurements, that systems are being measured on how they behave, where are there downtime, where are there not downtime, are there any response-problems. It is up to each team at NAV how they measure their software, but it is encourage that they do, to understand what state their software is in and have a good understanding of it.

%\textit{Participant \#7} explaining that his impressions that that there are little faults in the software at Skatteetaten. If looking that the processing for calculating how much taxes is owed in Norway and the amount is sent out to the citizens of Norway, which only happens once a year. During each time, the faults is being measured and fixed. The next year the same faults are not measured, new faults arise. However it is still of high quality.

\subsection{Software Properties}
A property described as being an indicator for a piece of software to be of good quality, is that the data it serves is easily accessible. If a systems quality at Mattilsynet should be assessed today, it would be based on how its data interfaces with other applications. Today it is expected for an application to have accessible API's so that it is predictable and easy to retrieve its data. This is something which is not possible to achieve without following all aspects of good software quality, such as correct architecture and mindset in building a solution.

This description of software quality has also led to new initiatives being started at Mattilsynet, where when a new piece of software is create, it's API is created first. This is done by creating API-contracts using OpenAPI, and designing the API from these contracts. This form of API-contract creating strategy is something that the product-teams at Mattilsynet is going to start doing. Thus enabling the product-teams to discover in an API has already been created by another team, preventing duplicate work.

Mattilsynet also using other data-sharing patters to increase the quality of their services, such as the publish-subscribe pattern. This enables Mattilsynet to share sets of data between applications through data-streams, both internally and externally of Mattilsynet. For instance when Nortura wanted data from Mattilsynet on what restrictions an animal has, such as what medication the animal is using. Instead of constantly asking Mattilsynet for this data, Nortura can simply retrieve one of Mattilsynets data-streams on the API-level. This also has the effect of creating a loose coupling between Mattilsynets and Norturas software, meaning that Norturas software will not fault if Mattilsynets software are unavailable.

%\textit{Participant \#9} explaining that easily accessible data is important for software to be of good quality. How data is store and retrieved in a system is important. If a systems quality should be assessed today it would be based on how its data interfaces with other applications. Today it is expected for an application to have accessible API so that its easy to get its data, and its not possible with everything that follows in good software quality such as the correct architecture and mindset on building a solution.

%\textit{Participant \#10} explaining that an initiative has started at Mattilsynet when a new piece of software is created, the API is created first. This by creating API-contracts using OpenAPI, and designing the API from this. This form of API-contract creating strategy is something that the product-teams at Mattilsynet is going to start doing. This enables the product-teams to discover if an API has already been created by another team, preventing duplicate work.

%\textit{Participant \#10} also explaining that the publish-subscribe pattern is used to share sets of data between applications through data-streams, both inside and outside Mattilsynet. There was a case where Nortura wanted data from Mattilsynet on what restrictions an animal has, such as what medication an animal is using. Instead of asking Mattilsynet for this data, Nortura can just retrieve one of Mattilsynets data-streams on the API-level. This also gives a loose coupling between Mattilsynet and Norturas software, meaning that Norturas systems will not crash if Mattilsynets systems are down.

\section{Methodological}
\subsection{Agile}
The Norwegian public sector is not concerned with assuring software quality through strict regimes, through projects which were quite common before. However not is is based upon getting feedback from its users. If the user do not experience any faults with the software, then it is of high quality. If the user find that the software is of value, then it is of high quality.

Retrieving continuous feedback on piece of software is a method which is used to assess the quality of the piece of software. A method for feedback is to demonstrate the software being developed, and at each demonstration retrieve feedback on the quality of the software. Such demonstrations were also performed frequently at the development in the "Perform project" at the Norwegian Public Service Pension fund.

The principle of retrieving continuous feedback in the Norwegian public sector can be a challenge, as the citizens of Norway usually are not interested in giving feedback, only being interested in that the software functions correctly. This challenge increases as handling proper feedback in a public sector from its citizens is important due to the democratic principle of the public sector having power over their users. Also for the users to be willing to give their feedback in a public sector, the users have to trust that the particular public sector will not use the data against their user. This is not seen as a challenge in the private sector, due to them not having the same power over their users.

In NAV's project to develop a new system for distributing parental benefits, most agile practices where followed in order to obtain good quality in the systems software.

At Skatteetaten, agile methods are used to ensure that their software is of high quality, including delivering MVP's of the software to ensure that their software delivers the value as expected. Including in these agile methods are development speed, what is happening in the development of a piece of software and how each new release is published. 

It is described that Mattilsynet could have more established routines for assuring quality in its software. However Mattilsynet is a young organisation when it comes to software development, and is currently most focused on increasing the development speed for its new product-teams and increasing productivity. Once this is at a satisfactory level, the focus will be shifted from development speed to quality, testing and management of Mattilsynets software.

%\textit{Participant \#1} explaining that by using agile methods you demonstrate the software being developed, and at each demonstration feedback will be given about the software quality.

%\textit{Participant \#1} explaining that at the perform project at the Norwegian pension fund, they had frequent demonstrations of the system they were creating.

%\textit{Participant \#1} explaining that in agile methods it is primarily a customer which is receiving a product, and that it is primarily the product owner which follows the development team, and must be able to give feedback on the most important aspects in software quality, and if it is good enough.

%\textit{Participant \#1} explaining that in agile methods, it is meant to be used as a method to maintain high software quality throughout the lifetime of the software. That you should not compromise on quality to quickly finish a feature, instead compromising on the amount of user stories or features of the software.

%\textit{Participant \#1} explaining that in NAV's project to develop an new system for distributing parental benefits, most agile practices were followed in order to obtain good quality in the systems software.

%\textit{Participant \#1} explaining that in agile methods there should not be a head of architecture who controls everything. Rather the head of architect should have dialogue with each team and should with the teams agree on good architectural decisions. 

%\textit{Participant \#2} explaining that in the democratic principle where a country has power over their users, the country has to make sure they treat their feedback from their users properly. However this can be a challenge, as not many users in the Norwegian public sector wants to give their feedback, they just want the software to work.

%\textit{Participant \#2} also explaining that the Norwegian public sector is not concerned about assuring quality through strict regimes, like through projects which were very common before. However now it is based upon getting feedback from the users. If software do not crash on the user, the quality is high, if the user find it valuable and good, it is of good quality.

%\textit{Participant \#2} also explaining that a challenge in the Norwegian public sector is to gather data about their users. The users must be willing to tell if they are satisfied, and since the Norwegian public sector have power over their users, its important that the user trusts the public sector to give away their data, and their feedback will not be used against them. This is easier in the private sector as they do not have the same power over their users.

%\textit{Participant \#3} describing that software quality does not have so much to do with ISO standards, but being able for a team to be able to change in their development of a system. Setting ability to change as similarities to quality in the software NAV creates. This because the development teams might not know what is correct or best when developing software in a world which is always changing. The development teams does not want any faults in their software, however since the world around the software is always changing, the teams have to guard against eventual faults and ensure that the consequences are as low as possible.

%\textit{Participant \#8} describing that software quality is described quite simply, how simple is it to modify the software. Also describing that in Skatteetaten agile methods to ensure quality, including delivering MVP's of the software and ensuring that their software delivers as expected.

%\textit{Participant \#9} explaining that an important measure to ensuring software quality is development speed. What is happening in the development of a piece of software and how is new releases published.

%\textit{Participant \#9} explaining that Mattilsynet could have better routines for assuring quality in its software. However Mattilsynet is a young organisation when it comes to software development and is currently most focused on development speed for its new product teams and focus on productivity and releasing solutions, instead of quality and testing. It will then become a challenge to change from this focus on development speed to quality, testing and managing its software.


\subsection{DevOps}
There are allegedly studies on the difference of the review of software through demonstration or deployed software. And such studies concluding with the customer and users finding more faults when the software is deployed, rather than in an demonstration. This because when the software is presented in an demonstration, the context is artificial. The user reviewing the software does not get to experience how it works in their own organisation or in their day to day life. Hence other aspects of quality is reviewed when the software is deployed and available to the user in their normal context.

For the ensuring of developing robust software, continuous deployment of that software will need to be performed. To ensure that other quality aspects of the software is high when being developed, high development speed and good development flow is needed for the development teams. Also if the software is not maintained, it will naturally decrease in quality. Any eventual fault in their software needs to be fixed quickly as to ensure that the software constantly is of quality. It is for these reasons that the development teams at NAV follow the DevOps Principles.

Helping the development teams of NAV to follow the DevOps principles, NAV have created their own application platform, NAIS. NAIS is designed in a way which gives strong recommendations on how software should be developed and deployed to NAV's infrastructure, including standardised services for typical NAV applications. The sum of NAIS, is that it makes it easier to create software at NAV, which increases software quality. The teams at NAV are not forced to followed the recommendations of NAIS, however are encouraged to make good decisions. These are important incentives for the development teams themselves to operate the software they have developed, resulting in increased software quality.

NAV uses continuous deployment to gain feedback about quality in their software. In the development project for the disbursement of parental benefits, it was initially planned to be delivered through three large deployments, however for the last deployment the delivery plan was changed. For the last deployment, the teams were combined with both software development and management resources, and continuous deployment was performed instead of a single deployment. What NAV experienced from this change in the project, is that it gave the teams more control, due to the feedback from the users being more authentic on a wider range of quality aspects. 

During the development of software, the negative consequences has to be as low as possible, which needs to be handled with continuous delivery. In the past, NAV had 4 large releases every year for their new software, where the largest release included about 116 000 development hours. The only way to ensure that 116 000 development hours do not cause negative consequence, is to perform a large amount of manual tests and a large amount of check lists. However NAV believes that more frequent and smaller releases decreases the consequences of an eventual fault. This is why NAV has gone from 4 releases a year to 1500 releases a week.

The teams themselves at Skatteetaten have the responsibility of delivering production ready software, where from 2013 Skatteetaten have worked with continuously deploying production ready software. And focusing on deploying code that is at user-story level, not at the delivery level. This to be able to deliver production ready software as early as possible. If there are any faults in the software, it is most likely of limited scope, and the development team is able to quickly roll back the software to an earlier versions and fix the fault. This is in contrast to large and infrequent releases, where large acceptance tests are needed, resulting in large faults.

New software being developed at Mattilsynet is created using the micro-service or service-oriented architecture. Which is then deployed to Mattilsynets own Kubernetes cluster, which gives a declarative way of deploying software. Mattilsynet also have a set of requirements for their software to be deployed to their cluster. The code has to be stored in a GitHub repository which is a part of Mattilsynets GitHub organisation. This for Mattilsynets platform team to know what the code is, what it does, and are able to scan the code for different reasons.

%\textit{Participant \#1} explaining that there has been done studies on the difference of demonstration of deployment of software. And the studies concluding with the customer and users finding more faults when the software is deployed, instead of in demonstrations. This because when the piece of software is being presented in an demonstration, the context is artificial. The user testing the software does not get to experience if it works in their own organisation or how it works with other tools used in their normal life. Hence a other kind of quality is tested when the software is deployed and available to the user in their normal context.

%\textit{Participant \#1} further explaining that NAV uses continuous deployment to gain feedback on its software. And in the project for parental benefits, the project was initially planned to be three large deployments, however for the last deployment the project switched tactics. For the last deployment, where teams were combined with both development and management resources, where continuously deployment was performed. And thinking that this gave the teams more control, due to the feedback from the users being more authentic on a wider range of quality aspects.

%\textit{Participant \#3} explaining that in the development of software, the negative consequences has to be as low as possible. This needs have to be handled with continuous delivery. In the past, NAV had 4 big releases every year, where the largest release had about 116 000 development hours. The only way to ensure that 116 000 development hours does not cause negative consequences is to do a lot of manual testing, a lot of check lists. However NAV believes that doing more frequent releases will decrease the consequences of an eventual fault. This is why NAV has gone from 4 releases a year to 1500 releases a week.

%\textit{Participant \#4} explaining that to ensure the development of robust software it needs to be continuously developed. If not maintained it will naturally decrease in quality. Also explaining that in order to create software of quality, speed and good flow is needed for the development teams. The teams at NAV follows the DevOps principles.

%\textit{Participant \#4} explaining that NAV has their own application platform, NAIS. NAIS is designed in a way which gives strong recommendations on how software should be developed and deployed to NAV's infrastructure. The sum of NAIS is that it makes it easier to create software at NAV, which increases quality. The teams at NAV are not forced to follow the recommendations of NAIS, however are encourage to make good decisions. These are important incentives for the teams to operate the software they develop, as this increases quality.

%\textit{Participant \#4} also explaining that to increase quality it is important to be able to fix faults quickly. Faults will always happen, so it is important to be able to fix these faults quickly.

%\textit{Participant \#5} explaining that NAV has a whole range of software, from COBOL software which is deployed to mainframes, to software deployed to the NAIS platform. NAV works the most with software that is deployed to the NAIS platform, where the platform itself maintains a higher level of quality by introducing standardised services and expectations on how software should be deployed.

%\textit{Participant \#7} explaining that at Skatteetaten, the teams themselves has the responsibility of delivering production ready software, where from 2013 they have worked with continuously deploying production ready software. And deploying code that is at user-story level, not at the delivery level. All of this to deliver production ready software as early as possible. If there are any faults, it is most likely of limited scope, and the team is able to quickly roll back to an earlier version of the software and can fix the fault quickly. This in contrast to large and infrequent releases, where large acceptance tests are needed, which gives large faults.

\textit{Participant \#10} explaning that if not looking at its older software, Mattilsynets newer software is build using the micro-services or service-oriented architecture. The software is deployed to a Kubernetes cluster, which gives a declarative way of deploying software. Mattilsynet has a set of requirements for their software to be deployed to their cluster. The code has to be on GitHub and part of Mattilsynet's GitHub organisation, so that their platform team knows what it is, what it does and then can scan it for different reasons.

\subsection{Method}
\textit{Participant \#1} explaining that there is little difference in the methods that the Norwegian public sector and the Norwegian private sector use in the development of their software.

\textit{Participant \#1} also explaining that at the "Perform project" at the Norwegian pension fund, pair-programming was used as a software development method. However not all development teams in the Norwegian public sector perform pair-programming, but pair-programming is done for the more complex tasks. The development teams do make sure that all team members has pair-programmed in an development iteration. This has the effect of increasing ownership in the code the team is developing, which prevents future faults and the development team has a better understanding of how all the code function.

\textit{Participant \#1} also explaining that at NAV software teaming or mob-programming is used, meaning that the whole development team sit together and write code. This enables more discussion about the code than by using pair-programming. This also has the effect of increasing knowledge in the whole development team.

\textit{Participant \#2} explaining that software developers are using agile methods to ensure quality in their software by releasing it to the user, and seeing if the user gains any value from the software. These principles are used at NAV and Entur, which focuses MVP's and let the user test the MVP's. If the user do not care about a feature, NAV and Entur do not care either. By repeating this with their users, NAV and Entur ensure high quality in their software.

\textit{Participant \#5} explaining that through NAV's technical direction, which describes what methods are desired in software development. This describes that pair-programming is preferred and it recommends that the development teams to it as often as possible. It also recommends focus on test-driven development and focus on test coverage in the development teams.

\textit{Participant \#5} also explaining that some of the processes for software quality assurance at NAV are not fully documented. In a revision situation of a project, it would be challenging to get a overview of the quality assurance methods, the development teams would have to be asked to document their processes. This has benefits, which is that the teams at NAV do not have to spend much time on documenting their processes just for documentation sake.

\textit{Participant \#7} explaining that he thinks that the traditional ISO-mindset controls the quality of software after it has been created. For the last 15 years at Skattetaten the mindset has been that the control of software quality should be done from the start, instead of doing it after the fact.

\textit{Participant \#7} also explaining that Skatteetaten has a method-framework which explains the best practice in software development and describes what working patterns should be followed. This is then added as part of the developments teams methods and is communicated to share experience on it. It is not seen as a method just for the methods sake, but is used as a way to share knowledge.

\textit{Participant \#7} also explaining that part of Skatteetatens development patterns is doing things in the right order and having a good work flow, that competency is used in a good way. As a software developer there might be things to get stuck on, such as not understanding logic, organisation rules, architectural decisions. In these sceniarios good quality is being preventative in making the right decisions.

\textit{Participant \#7} also explaining that to achieve good quality, cross-functional teams is needed. In software development there are a lot of aspects to think about, and it is not possible for a single person to handle all these aspects alone. Pair-programming is also recommended in assuring quality, instead of discovering faults and bad quality in tests.


\textit{Participant \#8} explaining that Skatteetaten follows certain software development methods, such as KISS (Keep it simple stupid), Clean Code and Separation of Concern. These are all used to assure good software quality.

\textit{Participant \#8} also explaining that there are several mechanisms at Skatteetaten to ensure good software quality. Such as code being through a test-phase is required for anything to be deployed to the production cluster. The developers never take any shortcuts. 

\textit{Participant \#10} explaining that at NRK there was still a lot projects, with the focus on external revisions or performing quality assurance of the projects. However this did little to actually assure quality in the project.

\section{Organisational}
\subsection{Team}
\textit{Participant \#1} explaining that it is the environment who is creating the solution who should have opinions on what is good software quality in the context of the solution they are creating.

\textit{Participant \#1} also explaining that in the larger project, there has been sub-projects which has focused on software quality. With people with knowledge in testing techniques and testing tools working both with the team and outside the team.

\textit{Participant \#2} explaining that at Entur is based on that their development teams should be autonomous and decide what should be done at what point in a project. Assuming that this includes the assurance of software quality is part of the team, and the team can therefore decide what methods they should use to assure quality.

\textit{Participant \#2} also explaining it is the teams themselves that have the responsibility on what methods should be used to assuring quality dependant on the context which they are in.

\textit{Participant \#2} also explaining that assurance of quality might have been a more formal part of software development, when the software development process was more defined, like in the waterfall model. Now it has changed to development teams choose themselves what should be prioritised and what is important. Further suspecting that this change has led to software quality assurance being prioritised lower and is seen as less important. The development teams instead thinking that if the users are happy, then the software is of high quality.

\textit{Participant \#2} also explaining that he does not know if any teams have a person which is responsible for quality assurance. However the teams has a for example a product owner or designers which have the user in mind, and that the user should feel that the software is usable and valuable. Further stating that it could be a challenge that the responsibility is not placed on a single person, however there is a responsibility throughout the team.

\textit{Participant \#5} explaining that NAV has changed from centralised requirements and check-lists to where there is no centralised management. It is up to each team to understand how the software they develop function. Management trusts that the teams has the best knowledge of their product and can create good products.

\textit{Participant \#6} explaining that if a piece of software is large enough to need more than one person, the need for software quality assurance increases. Thus the quality assurance needs to be adapted to the people who own the software and how the team developing it co-operate. 

\textit{Participant \#6} also explaining that NAV is making efforts to make it easier for the development teams to change, test and own the software they are developing or maintaining.

\textit{Participant \#8} explaining that at Skatteetaten the teams are cross-functional. Each team has a member responsible for the architecture and that it follows the guidelines and patterns which are approved in Skatteetaten. There is a also a member responsible for security which makes sure that the has good security in their product. These is a also a member which have the responsibility of DevOps who makes sure that the CI/CD pipeline is properly configured. The teams does also have a product owner, which is responsible for choosing what should be prioritised and what should be done at what times. There are also members who are representing the organisational and making sure the organisational requirements are met. There is also a member who has an extra focus on testing and testing related activities. Roles such as security and testing responsible does not have the full responsibility for this, they are only supposed to make sure that it happens, the whole development team is responsible for these activities. 

\textit{Participant \#10} explaining that Mattilsynet has recently done a big job of moving away from project based development to product-teams. This is making Mattilsynet solve their strategic problems by making sure that in the highest degree possible, the correct problems are being solved. This helping solving the issue of having software that is high in quality, but does not solve the correct problems, which are not useful. Thus switching to product-teams has been an important decisions and making sure the product-teams are well looked after.

\textit{Participant \#10} also explaining that the teams at Mattilsynet has about 10-11 product-teams with great autonomy, with no centralised authority for architecture. This is something which Mattilsynet might implement very shortly, as it is becoming to be a problem that there is no centralised authority for architecture, and the problem is only going to grow. However the strategy for now is simple increasing the velocity of the product-teams before giving the product-teams the direction, and instead iterating the project-teams as new problems arise.

\textit{Participant \#10} also explaining that there are no boundaries in the choice of technology etc by the product-teams. However when recruiting new developers, a skill in a certain set of programming languages are preferred, such as Kotlin/TypeScript. The product-teams will not get any pushback from choosing a language which is not widely deployed at Mattilsynet, but the team must be prepared to have good arguments for why they made the particular choice. This has been a deliberate choice, as it should be easy to follow the "main path" at Mattilsynet, but not making any hard borders on what is allowed, so that the product-teams can experiment, as hindering this slows town the development velocity. However going beyond the "main path" is going to increase costs, so the product-teams needs to bear this increased costs, this motivating the product teams not to just experiment without having a good justification.

\textit{Participant \#10} also explaining that the product-teams at Mattilsynet has a mix of consultants and permanent employees, no longer having teams purely consisting of consultants. This ensures that both consultants and permanent employees understands the costs of creating long-lasting software. This due to Mattilsynet having experiences with consultants not taking into account the costs of maintaining a software when creating it.

\textit{Participant \#10} also explaining that the product-teams at Mattilsynet are set up to be long lasting and there is no plan for these teams to be dissolved, which is a view shared by the leadership of the IT-department of Mattilsynet. This ensures that the teams most familiar with the software have a long-term ownership and responsibility for that software. Thus ensuring that any technical debts are decreased.

\textit{Participant \#11} explaining that Mattilsynet develop their software on different levels, that each product-team has a tech lead which has the main responsibility for the technical quality. Mattilsynet also have a tech lead forum for all tech leads at Mattilsynet to increase knowledge sharing between all the tech leads.

\textit{Participant \#11} also explaining that Mattilsynet has discussed a range of governance-principles which each team have to deal with. These principles include things such as security, archive-law and GDPR. However the product teams are quite new, and has a high percentage of newly employed members, so Mattilsynet is currently only focusing on governance-principles which are required by Norwegian public sector agencies by Norwegian law, these being GDPR and security.
To help with this, Mattilsynet has a central team for security which works closely with the platform team to create a good security foundation for the product teams. 

\subsection{Knowledge}
\textit{Participant \#1} explaining that the Norwegian public sector participates in conferences to discuss competence development for their agencies. It is not only the larger agencies such as NAV and Skattetaten who attend such conferences, but other agencies in the Norwegian public sector.

\textit{Participant \#4} explaining that the sharing of knowledge is important for assuring quality in software, as well as high competence throughout a organisation is important for software quality.  

\textit{Participant \#6} explaining that competency can be a challenge. It is quite clear that the people with the least experience in software development has little knowledge in testing or continuous delivery, so these have to be trained in such things.

\textit{Participant \#7} explaining that the people maintaining Skatteetatens older IT-system does not have the competency needed to develop and maintaining their newer software. These people usually work with stuff like Oracle databases do not know how to create a user-story or create a Java component. Skatteetaten need a different competency than that as there is a large change in their modernisation plans.

\subsection{Project}
\textit{Participant \#1} explaining that a thing which is important for the Norwegian public sector is the Norwegian digitisation agencies project wizard, based upon PRINCE2. This project wizard is not suited for agile project, rather project following methods relating to the waterfall method. This means that you will see the methods such as the waterfall method which require more planning and defining of a project in the Norwegian public sector rather than in the Norwegian private sector.

\textit{Participant \#1} also explaining that the health platform developed for the counties in mid-Norway, was mainly created using the waterfall method, switching to an agile method later in the project. However most professionals would agree to not follow such waterfall methods, as it most of the times leads to the software quality suffering. In such projects it common to define a set of requirements, send them to multiple contractors, the contractors fight to deliver the cheapest system. Then its necessary to take some shortcuts, and if the requirements are written in a contract, it will be on the software quality. And this problem gets even larger the receiver of the project is not good at evaluation software quality. That is the problem with sending software systems on tender, it distances the requirements from the user of the system.

\textit{Participant \#1} also explaining that there are different contract models which can be used in the Norwegian public sector. Where competency is hired into an already existing development project which agencies such as NAV and Skatteetaten is doing. Here the agency are the owner and responsible part of the project, but are hiring resources need to complete the project. However this requires a higher degree of maturity in the agencies than before.

\textit{Participant \#1} also explaining that if using the health platform as a case, it was known beforehand that it wouldn't be very user friendly. And special activities to get the usability better were not performed, as well as being technical faults in the platform. Including a testing rig which were to far away from the real environment which the platform should operate in. This could be due to a lack of IT-competence in the planning of the project.

\textit{Participant \#2} explaining that the yearly budgeting in the Norwegian public sector is an issue. In the private sector if you can prove that a development project will earn money, then its easier to spend money for that development project. However in the Norwegian public sector budgets are planned yearly, so it is not easy to change how its budged should be spent to different development projects underway. If a certain software in the public sector experiences any sudden problems which needs larger funding, it will be difficult to get the necessary resources. This problem is larger for software quality, as this is not a good enough reason for changing the funding underway in a budget year or retrieving extra funding.

\textit{Participant \#2} also explaining that the issue of yearly budgets also affects the creation of new software in the Norwegian public sector. As the money is funded by the Norwegian tax payers, a development project usually do not have enough funding, hence the software quality is where it is usually saved on to fulfil the requirements of the new software.

\textit{Participant \#2} also explaining that a challenge in the Norwegian public sector is to adjust off the shelf software as a way to save money. Instead of doing it like NAV which has in-house developers who create the software themselves, and have full control of the software quality. By trying to adjust off the self software might lead to lower quality or spend a lot of resources of adjust it to a point where the users are satisfied. So there might be a divide between the parts of the Norwegian public sector which have their own development environment, so they can create high quality software themselves, and those who cannot and have to adjust off the shelf software.  

\textit{Participant \#2} also explaining that his impression that if an agency has an in-house development environment, it know that agency and their challenges quite well, so the job of adjusting off the shelf software might be easier. However when using off the shelf software, the rules of the software are already set which has to be taken into account. When building something from scratch, these rules do not apply anymore, which means that the software can be highly specified for the users needs.

\textit{Participant \#3} explaining that a lot of organisations wants to deliver software to its users like NAV has achieved. It sounds quite attractive and is spoken about in conferences, however it requires quite a lot from the organisation. Mainly it requires to not have software delivered solely from a consultants, but software developed by in-sourcing at the organisation. As soon as software is created outside the organisation and is delivered, the quality can decrease. To do as NAV it requires the organisation to take ownership in their software, however not all agencies in the Norwegian public sector have the opportunity for this.

\textit{Participant \#3} also explaining that before 2016 NAV had little knowledge or resources do develop software themselves. All systems required for NAV's operations had to be put on tenders. However NAV has stopped with this, as it is not sustainable and has negative consequences. In particular when NAV got their original system for delivering pension benefits, NAV did not know how to maintain it themselves. This meaning that NAV had to contract the maintenance and further develop the systems which are responsible for their core business, which is bad for a public sector agency.

\textit{Participant \#3} also explaining that the Norwegian digitisation agency's project wizard is a challenge for NAV and the rest of the Norwegian public sector. The project wizard which is necessary for NAV to follow to get the correct funding for their larger software proposals, has a quality assurance regime that is more suited for building roads, rather than software. The project wizard assumes a perfect understanding of the software proposed before it is created, which NAV has never had, and never will. When proposing something which will replace 40 year old software, which as been altered to fit new laws and needs through 40 years. And actually start creating the replacement, you uncover new things which could never be planned ahead of time. The fact that the funding from the project wizard stops after the software put to use is an issue. NAV wish to create software wish will live for a long time, and this requires a steady funding for development, long after the software has been put to use. This because there will always be changes in the laws surrounding NAV's software and other needs for the software, and the project wizard is not suited for such changes. This is something which is hurting NAV the most, as it has gotten the furthest in its digitisation process of the Norwegian public sector agencies. And will increase to hurt NAV as becomes better at delivering digital solutions to the citizens of Norway.

\textit{Participant \#4} explaining that before the software quality assurance was centralised at NAV. When getting delivery of a new system there would be quality assurance of the delivered software. The quality assurance was disconnected from the operations of the system. A lot of resources was used to control quality, but faults were not hindered. Today at NAV, the quality assurance is connected to the development of the software.

\textit{Participant \#4} also explaining that there are agencies in the Norwegian public sector who are following NAV's methods for improving their digitisation transformation efforts. This by rigging themselves around DevOps, product-teams and small specialised applications. Even though the the Norwegian public sector is improving, they still have some way to go to be as good as the private sector. It is worth mentioning that the agencies are not follow NAV just because NAV is successful, but because the methods NAV are using are smart trends.

\textit{Participant \#4} also explaining that there are challenges in how the development of software at NAV is financed and has a to strong dependence on project funds from the Norwegian digitisation agency project wizard. The assumption that you can plan, build and be done with the software is a faulty assumption. NAV wants to build small applications which can be iterated, and the developers can learn as they go, since you learn more about application this way than in the planning phase. The funding from the project wizard does not fit this mindset, there should instead be continuous funding.

\textit{Participant \#4} also explaining that the IT budget of NAV is set wrong. The funding should instead be put to the problem, not the technology. For NAV's case, this means increased funding for a benefit, so that the benefit can create software that it needs, rather than all the funding going under the same "IT-fund". This way the management of NAV can be more flexible with their funds on what software should be created for the benefit that needs it the most. However now it is all "unreachable" for the management and not prioritised in a way that is beneficial for NAV's users.

\textit{Participant \#5} explaining that the issue of financing might rather be an issue of prioritisation. It being a prioritisation between improving technical quality and new functionality. Thinking that its usually a discussion about financing, when in reality its an prioritisation issue, that there is to much to do, independent of the financing. Financing might not always mean that to little money, but the expectations of what should be delivered compared to the given budget being to much. And over-promising can create issues which are expensive to fix, the cleanup being expensive. So by not prioritising correctly and the wish to create as much as possible as quickly as possible and not prioritising to clean up can be an issue.

\textit{Participant \#5} explaining that there has been an external evaluation for the development of the system for sick pay at NAV. However this is not usually done, as this system had a lot of attention, and the progress was not satisfactory. So there has been external quality assurance of a lot of aspects of this system, including the software quality, which is not often done.

\textit{Participant \#6} explaining that NAV has gotten ahead on their digital transformation because of their focus on ownership. This being the solely use of consultants who does not have the long perspective when creating NAV's software. That the consultants think that they should create a piece of software for NAV, and then be finished. This being the opposite of good quality. As time goes on and a piece of software at NAV is not maintained, it looses quality, so you constantly have to fight against quality weathering.

\textit{Participant \#6} also explaining that NAV has a financial model which push then towards projects with a set deadline and a reduction of resources after the deadline has been reached. Both of which is bad for quality.

\textit{Participant \#6} also explaining that firmer boundaries and more trust in the financing of NAVs software would be good. Thus being able to more flexibly prioritise. It is hard to calculate money saved by using the project wizard, the rewards gained becomes an illusion. However firmer boundaries and more trust might not be smart for the Norwegian public sector, as it might lead to all agencies asking for to much funding.

\textit{Participant \#6} also explaining that the Norwegian digitisation agency's project wizard is a challenge of software quality. However it is the only way to get funding for the larger software initiatives at NAV. If not for the funding from the project wizard, NAV might had to simplify their services, leading to a reduction in what NAV can deliver for the citizens of Norway.

\textit{Participant \#7} explaining that the Norwegian digitisation agency's project wizard gives opportunities, but it also gives problems. Before NAV had a financial model which were heavily dependant on funds from the project wizard, and buying software from external vendors. This has Skatteetaten never done, Skatteetaten has always developed and managed their own software. This means that NAV has had an opportunity to save money as consultant might cost 2.5-3 million NOK, while a full time employee might cost 1 million NOK. That means that NAV has had good opportunity to build up a sizeable development environment of over 800 employees (\textcolor{red}{sleng in detsombetyrnoe.no sitering}). That opportunity Skatteetaten has never had, we rely upon the funding from the project wizard for for their larger initiatives and being able to delivered as required by the Norwegian government. The IT department of Skatteetaten has a total of 1000 personnel, where about 300 is consultants, these 300 consultants would never be possible without the funding from the project wizard.

\textit{Participant \#7} also explaining that when working with funds from the project wizard, only the most necessary and what is legally required should be implemented. This means that it might not be possible to prioritise things such as user needs, which is an disadvantage.

\textit{Participant \#7} also explaining that since 2003 the operational and IT has worked together in product-teams for IT-related project. However for other projects in the Norwegian public sector, it is common to put something out on tender, a contractor builds it, and delivers it. Skatteetaten has never done this, it has always been the case that operational and IT has worked together on creating software. This because when there is a drastic change in operations at Skatteetaten, it cannot just be given to a single team. It requires a change in organisation, how things are solved, how people cooperate both inside and outside the agency so that the IT-systems is created in new way.

\textit{Participant \#7} also explaining that the research by Torgeir Dingsrøyr from NTNU about NAV talks a lot about coordinating teams and steering of development direction. The politicians in Norway will always give Skatteetaten new things to do or existing things to change. Thus creating the need for coordination between product areas, multiple teams and multiple parts of the business operations at Skateetaten.

\textit{Participant \#7} also explaining that many thinks of product teams as something quite stable. However this is something which can be unwanted from the perspective of the organisation, wanting for the product teams to be more flexible. However when the organisation tries to change the staffing in an product team or development are, a lot of resistance is encountered. This due to the staff enjoying the team or area to which they belong, with the people they already work with. So organising around product teams can create stiffness in the organisation.

\textit{Participant \#7} explaining that Skatteetaten as a different strategy and a higher architecture than NAV has. Skatteetaten has an increased focus on common components or the re-use of components. When having a high degree of reuse, it lets Skatteetaten get more quickly done with certain tasks.

\textit{Participant \#8} explaining that in the process of making software, 20\% of the time is used to create the software. While the last 80\% is used to maintaining the software.

\textit{Participant \#8} also explaining that a more predictable financing model would be better for Skatteetaten. If all the money for a project is used, which is a risk with an un-predictable financing model, that the financial model does not take into account unforeseen events. This can have effect on the quality of what is being made.

\textit{Participant \#9} explaining that when it comes to the software Mattilsynet develops themselves, they have a lot to do to assure quality in their suppliers. It comes down to routines and questions are asked for potential suppliers, which might need to be deeper ingrained into Mattilsynets procurement process.

\textit{Participant \#9} also explaining that Mattilsynet does not work project based anymore. Recently having moved from two large consultant contracts in development and operations to having their own product teams. However this change has lead to a loss of routines which were in place.

\textit{Participant \#10} explaining that Mattilsynet has a multiple group which can define their software. There are third-party off the self software which have been acquired through tenants and a customer supplier model. The software is made by smart people, but the supplier model do not reflect well on them. The off the shelf software tends to be outdated, and their dependencies can be old. The supplier of the off the self software rarely say "no" to a feature request from a customer, so there grows complexity in the software.

\textit{Participant \#10} also explaining that Mattilsynet has an old system called "Mats". It is not an off the shelf software, but system tailored for Mattilsynet by a contractor. This has the problem of Mattilsynet wanting functionality, without the contractor being motivated to say no. So the system has grown complexity organically. The customer in this matter being Mattilsynet, has not understood over a longer period of time the consequences of own orders. Mattilsynet has ordered functionality while ignore technical debt, and thinking that someone else will fix the technical debt, but nobody does.

\textit{Participant \#10} also explaining that Mattilsynet has had quite weak requirement specifications for their contractors. There was not enough internal resources at Mattilsynet to create proper requirements specifications for the contractors, and the contractors themselves almost created the whole requirement specification for Mattilsynet. This was the case with Mattilsynets photo-app, where Mattilsynets inspectors can take pictures and store them in their databases, without having to use iCloud. However since the specification was weak, the ones making it did not think that someone had to maintain it for over 5 years after it was made. Thus making the decision making for technology and solution easier than it might should be, technical debt being created from the start.

\textit{Participant \#10} also explaining that the assurance of software quality is quite variable in the rest of the Norwegian public sector. Areas such as healthcare, municipalities and national defence are still heavily in putting projects on tender and assigning contractors. However there are others such as NAV, Skatteetaten, NRK and Norsk Tipping which create the software themselves. The big difference is that the consultants are hired as competency as part of an already existing development team. 

\textit{Participant \#10} also explaining the Norwegian digitisation agency's project wizard is something which can become a resting pillow for not moving to product teams. Before the new IT leadership at Mattilsynet, this was the case, that Mattilsynet should be better at projects suited for the project wizard. Mattilsynet had the need to stop with such projects, hence it was unfortunate that Mattilsynet was spending resources on becoming better at such projects. Such as actively recruiting staff that specialises in such projects, which creates slowness in the organisation. Then suddenly receiving leadership that does a 180 and switches to product teams instead. So it might have created a challenge for those with specialisation with project, it might not be big, but some challenge.

\textit{Participant \#10} also explaining that it is better to be good at projects and bad at projects. There is a project at Mattilsynet which was delivered as a service-oriented modern architecture, the project for Meat controlling, which was a well managed project. It was so well that it became the template on how projects should be done at Mattilsynet. However it had no real strategy on how it should be maintained and how further operations should be conducted after it is delivered. So even if the project was well done, it still left a big problem for Mattilsynet. The needs for the system has also changed before it was delivered. Now this system has been has been split up between two product teams which has not been problem free, as both product teams now have to develop components which they do not understand, with technology they do not know. Therefore functional changes in such systems do not happen, as there are lots of technical debts.

\textit{Participant \#11} explaining that Mattilsynet has moved away from projects with central functions, to a model with distributed product teams. Where the responsibility for software quality is up to each team. 

\textit{Participant \#11} also explaining that the project wizard from the Norwegian digitisation agency is not a challenge for Mattilsynet, as they have stopped using it.

\textit{Participant \#11} also explaining that in projects scope creep is a bad thing, something to be avoid. However scope creep only mean that new insights have been discovered about the problem being solved by the software. Thus having the possibility to give more value to the end user. Scope creep can also mean that the original scope of the project has changed, so not changing the scope can mean solving the wrong problem. This is the issue with things like the project wizard, which do not take changing scope into account and the dynamic approach which can be achieved by product teams.

\subsection{Resources}
\textit{Participant \#1} explaining that the main challenge for Norwegian public sector is the lack av people with IT-knowledge.

\textit{Participant \#1} also explaining that more agencies in the Norwegian public sector are doing like NAV. That what is needed to create good software is continuous development and maintenance. However this requires the agency to have their own IT resources.

\textit{Participant \#6} explaining that NAV has a old systems with low quality, which NAV does not have the resources to handle at this moment. An example of this is their payment system, which each year pays out 500 billion NOK to the Norwegian citizen, a third of the Norway's national budget. It is something which needs fixing, but nobody can fix it. The people who are able to fix it are over 60 years old, as it are written in COBOL on a mainframe. NAV is loosing competency on the older systems.

\textit{Participant \#7} explaining that Skatteetaen have a limited amount of resources and time available. So in creating new software prioritisation have to be made. Thus meaning that Skatteetaten's projects usually end up with a backlog of tasks which have been have de-prioritised, because there is a limited amount of time. This backlog is also due to lack of capacity to meet to meet the large amount of wishes and needs.

\textit{Participant \#10} explaining that Mattilsynet has a very large domain-complexity, while also having relaxed non-functional requirements. The domain-complexity is quite large, due to it being a mini-healthcare, only for more than just humans, but a wide range of animal species. This means that Mattilsynet does not have the capacity to maintain their operational logic, which is their largest challenge. In a year over 2000 new animal husbandry's are registered, which may not sound like a lot, but each of these are unique.

\textit{Participant \#11} explaining that it is important to have central resources in an organisation for the product teams. Such as security experts and legal advisers. Then teams can use this special expertise in their projects. There are not enough resources for the teams themselves to have this expertise, but this allows them to be somewhat available for the teams, which is important for software quality.

\subsection{Legal}
\textit{Participant \#2} explaining that it is a challenge for the Norwegian public sector to gain feedback from their users, as there is a big focus on GDPR. And its bad for the public sector if they use data about their citizens, which they are not allowed to, due to the power they have over their users.

\textit{Participant \#7} explaining that software quality is that it the software solves what it should solve, that it covers needs. Not just user needs, but operation need, such as legal requirements. The needs of Skatteetaten are heavily driven by legal requirements, as the legal requirements say how thing should be and how the software should function. The software should have good usability, however the software at Skatteetaten usually are created from legal requirements, not user needs.

\textit{Participant \#7} also explaining that if looking at the legal requirements of a piece of software, then Skatteetaten and NAV are able to change their software due to new Norwegian laws. However the Norwegian police are not always able to change their software due to changes in the Norwegian laws.

\textit{Participant \#7} also explaining that he was talking with the Head of Architecture at NAV, and heard that only 20\% of their IT operations were bound to legal requirements. However at Skatteetaten it is the opposite, 80\% of their IT operations bound to legal requirements. Due to such a large percentage being bound to legal requirements, means that Skatteetaten has to de-prioritise user needs in order to fulfil the legal requirements, even tough Skatteetaten always has their users in mind.

\textit{Participant \#7} also explaining that Skatteetaten is struggling with legal requirements which are not digitisation friendly, and making these simpler for their operations. In a project in cooperation with The Brønnøysund Register centre, Skatteetaten and Brønnøysund wanted to create a panel where busuiness owners could see information about requirements from both Skatteetaten and Brønnøysund at the same place, which was a wish from the Norwegian business owners. Skatteetaten did not get any data from Brønnøysund, and Brønnøysund did not get any data from Skatteetaen. However the laws did not allow this, even tough the user owned the data, they were not allowed to combine it and show it to the user. On the other side, Skatteetaten are allowed to share data to other parts of the Norwegian public sector, such as the health sector. Overall the legal requirements set hard limitations to deliver good user experiences.

\textit{Participant \#7} also explaining that Skatteetaten shall create trust in the Norwegian populous so that each citizen is motivated to pay their taxes, to pay for the Norwegian welfare state. The sharing of data and information is also part of Skatteetatens social missions for Norway. This is not only about taxes, but Skatteetaten know alot about the Norwegian society in general and the Norwegian private sector, and a lot of this data is used to improve the Norwegian society. However this sharing is difficult, due to legal requirements which are putting limitations on this data sharing.

\textit{Participant \#10} explaining that Mattilsynets core system, Mats, are affected by a wide range of legal requirements. When the system got a bug, a wide range of legal requirements was affected.

\textit{Participant \#11} explaining that in the Norwegian public sector there are legal requirements related to archive laws, which states that specific documents should be archived or recorded in journals, which are things that should be handled automatically.

\textit{Participant \#11} also explaining that to help their development teams with legal requirements and GDPR, a legal adviser has recently been hired, which will be of great help in raising knowledge on GDPR in the different product teams.