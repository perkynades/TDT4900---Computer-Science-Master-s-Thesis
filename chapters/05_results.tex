\chapter{Results}

\textcolor{red}{Når jeg skriver disse kan jeg kanskje sette disse i større grupper, for å gi mer kontekst rundt hver og hva de betyr for hver etat.}

\textcolor{red}{Format: Dette er hva forsker sa. Dette er hva personene i etat 1 sa. Dette er hva personenen i etat 2 sa.}

\textcolor{red}{
Se på hva som må endres på i denne listen:
\begin{itemize}
    \item Er det noe overlapp på "Non-functional requirements" og "definition"
\end{itemize}
}

\section{Non-functional requirements}
\textit{Participant \#1} describing that software quality is related to non-functional requirements, such as usability, maintainability, performance, security, or the mutability of the software's architecture. 

\textit{Participant \#1} explained that a team of software consultant once was delivered a list of over 250 non-functional requirements which the intended system had to maintain. Such a long list of requirements being difficult to control that the requirements are actually followed, being even more difficult to actually implement the requirements.

\textit{Participant \#2} describing software quality as a piece of software being made in such a way that it is robust. And that it is made so that it is easy for other programmers to modify, maintain and repair if it is needed.

\textit{Participant \#4} has similar reflections on non-functional requirements. However also focus on the manageability of the software being created, that it should be long lasting and be able to be modified over a longer span of time. Also describing that a piece of software is not dependant on 1 person, but the whole developing it. 

\textit{Participant \#5} describing that NAV has called non-functional requirements as quality requirements or quality properties. Something which was done in the past. Also explaining quality as testable or maintainable.

\textit{Participant \#7} explaining that in the technical quality of software, some non-functional requirements are more important than others, such as manageability, functionality, usability, maintainability being more important to focus on than other non-functional requirements. 

\textit{Participant \#8} relating to the reflections of \textit{Participant \#2} in that a piece of software should be easy to modify, and that the software's code should be readable and quick to understand how it functions by just reading it.

\textit{Participant \#9} explained that he had worked with software quality, but measuring it up against the ISO standards. Including measuring it up against non-functional requirements such as usability, security, reusability etc. 

\textit{Participant \#11} describing good usability as important in software quality, that the user should feel that a piece of software is of quality, the visual aspect is perceived as attractive. Also describing software quality is that piece of software is stable, secure and robust.

\section{Agile}
d
\section{Simulation}
d
\section{Simulation}
d
\section{Inspection}
d
\section{DevOps}
d
\section{Architecture}
d
\section{Team}
d
\section{Testing}
d
\section{Knowledge}
d
\section{Method}
d
\section{Project}
d
\section{Resources}
d
\section{Feedback}
d
\section{Definition}
d
\section{User}
d
\section{Contextual}
d
\section{Measurements}
d
\section{Revision}
d
\section{Legal}
d
\section{Modelling}
d
\section{Languages}
d
\section{Domain}
d