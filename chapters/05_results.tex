\chapter{Results} \label{sec:results}
The results of the interviews conducted for the case study are divided into four primary categories:

\begin{enumerate}
    \item \textit{Perspectives} - What perspectives the participants had on "software quality" and "software quality assurance" in the Norwegian public sector
    \item \textit{Technical} - Aspects relating to the assurance of the technical quality of the software in the Norwegian public sector 
    \item \textit{Methodological} - Aspects relating to software development methods used to assure software quality in the Norwegian public sector
    \item \textit{Organisation} - Organisational aspects influencing the quality assurance of software in the Norwegian public sector
\end{enumerate}

\section{Perspectives} \label{sec:perspectives}
The results from the interviews in the case study show that participants \#1, \#2, \#3, \#4, \#6, \#7, \#8, and \#9 conveyed their perspectives on software quality and software quality assurance.

\subsection{Definition} \label{sec:definition}
The word \textit{"quality"} is described by participant \#2 as a word that is difficult to describe: "The word "quality" is a very difficult word. It's a word that everyone feels they understand and know what it is, but no one can really define. Although everyone has some sort of idea that they know what quality is". Further describing that \textit{"software quality"} are words which are not widely used in the Norwegian public sector: "I don't hear many people talking about software quality in the Norwegian public sector".

%The word \textit{"quality"} is a word that is difficult to describe, as it is as word which people think they understand and know what is, however nobody quite know how to define it. People think that something of high quality is something which will last long, something which will give value, something which is good. The words \textit{"software quality"} is something which is not often used in the Norwegian Public Sector.

Participant \#6 gives multiple definitions to the words \textit{"software quality"}. Firstly describing it as the ability to modify a piece of software: "I think that software quality is changeability". Secondly describing \textit{"software quality"} as a piece of software solving a particular problem for a long period: "NAV, for example, has "Infotrygd" which is about 40 years old, and which is nation-supporting. And if that is not quality, then I don't know. It has little ability to change. And as far as I know, it has very few tests. But it has solved a problem over 40 years".

%\textit{"Software quality"} has also been described as a word which is difficult to give a single definition of. One definition of software quality is the ability for a piece of software to be easily modified or adapted as time passes. NAV's system for disbursing social securities to the Norwegian citizens is nation-critical for the Norwegian society. It is a system which is difficult to modify and has a limited test coverage. Yet it as solved a problem for over 40 years. One definition would say that this software is of low quality, while another definition would say it is of high quality due to it being a robust system for over 40 years.

%\subsection{Definition}
%\textit{Participant \#2} explaining that the word "quality" is a difficult word to describe. It is a word which everyone think they understand and what it is, but nobody really know how to define it. Many think that something of high quality is something which will last long, something which give value, something which is good.

%\textit{Participant \#2} also explaining that he does not hear many use the words "software quality" in the Norwegian public sector.

%\textit{Participant \#6} explaining that software quality is the ability for software to change over time. One of NAV's systems for disbursing social securities is about 40 years old and is nation-bearing for Norway. However it is very difficult to change, and has a limited amount of tests. Yet it has solved a problem for over 40 years, so its of good quality in that sense, but yet it so bad in other ways.

\subsection{Context} \label{sec:context}
It is described by Participant \#3 that what context the software is situated in, is important for how its quality should be assured. In the context of deciding if software quality should be assured manually by humans, or automatically with computers. Participant \#3 describes both have pros and cons for their respective context: "If you're trying to do most of your quality assurance with computers, then you have the effect that you can do this much faster. However it will mean that something will slip through". The description of context deciding what software quality assurance methods should be used is shared with participant \#6. Describing that NAV's systems for disbursing social securities and sick pay do not use the same quality assurance methods due to their context being different.

%What context the software is situated in, is important for how it's quality should be assured. If creating software for medical machines, which can endanger humans, no faults can be allowed. Hence methods needed to assure quality could be manual and time consuming testing with a high degree of coordination. For software with less consequences of faults, automatic testing as a software quality assurance method can be acceptable.


%At NAV each piece of software decides how the development teams assures quality. The system for disbursing social securities and the system for disbursing sick pay do not use the same quality assurance methods. Some systems use manual and time consuming test-processes, while others use automatic testing. Some systems are combining both manual and automatic testing. Some systems are modified and deployed every third month, while other systems are modified and deployed 20 times a day.

Other situations are also important for the context of deciding what software quality assurance methods should be used. Participant \#6 explains that sometimes there is no need for quality assurance: "If you are going to develop something that will be used once, then changeability is not the most important thing. Then it is more important that it just works". Further explaining that in the described situation, a heavy focus on software quality by the development team can lead to unwanted complexity in the software, which does not help the software in solving its intended problem.

%The context deciding what software quality assurance methods should be used, are also important for other situations. If a piece of software is only intended to run once, mutability is not important, the most important property being that it solves the particular problem. In this situation a heavy focus on software quality by the development team can lead to unwanted complexity in the software, which do not help the software in solving it's intended problem.


%\subsection{Contextual}
%\textit{Participant \#3} explaining that what quality assurance method used is based on the context. Testing could be done manually, with a lot of coordination, or it could be done automatically with computers. Performing automatic tests will be much quicker, but some faults may not be detected. This can be acceptable for creating some software, however creating medical devices such as eye-lasers might need more rigorous testing, as you cannot let faults get through. So it depends on the context of what is being created.

%\textit{Participant \#6} explaining that software quality is defined based on the context. If a piece software is only meant to be used once, then ability to change the software is not important, then the most important thing is that it just works. A development team can be more focused on quality than the solution itself. As only focusing on quality can lead to a lot of complexity which is not necessarily needed for the software to solve the problem it is intended to solve.

%\textit{Participant \#6} also explaining that the systems such as the system disbursing social securities and sick pay at NAV do not use the same methods to assure software quality. Some have long manual test-processes, while others have automatic testing, and some are in-between with manual testing which are trying to move to automatic tests. Some system have changes which are deployed every third month, while other deploy changes 20 times a day. 

\subsection{Non-functional Requirements} \label{sec:non_functional_requirements_results}
As described by participants \#1, \#2, \#4, \#7, \#8, and \#9, the definition of software quality is associated with non-functional requirements. Software quality relates to usability, maintainability, performance, security, or the mutability of the software's architecture. As well as manageability, described by participant \#4 as important for software quality. 

%Associations to non-functional requirements are described when it comes to the definition of software quality, that it relates to usability, maintainability, performance, security, or the mutability of the software's architecture. Associations to manageability of the software being created is important for the software quality. These non-functional requirements are also described as being the most important in the technical quality of a piece of software.

At NAV, non-functional requirements have been described as quality requirements or quality properties by participant \#1. Also explaining that in another part of the Norwegian public sector, a team of software once delivered a list of 250 such requirements or properties which the system had to maintain. Further explaining that following such a list is challenging: "If you come with a very long list of things, then it is difficult to check that everything has been done. Not to mention actually doing it". 

%At NAV, non-functional requirements has been described as quality requirements or quality properties. In another part of the Norwegian public sector a team of software once was delivered a list of 250 such requirements or properties which the system had to maintain. Controlling that all 250 requirements are being followed to be difficult, and actually implementing all 250 requirements into the system being even more difficult.

%\subsection{Non-functional requirements}
%\textit{Participant \#1} describing that software quality is related to non-functional requirements, such as usability, maintainability, performance, security, or the mutability of the software's architecture. 

%\textit{Participant \#1} explained that a team of software consultant once was delivered a list of over 250 non-functional requirements which the intended system had to maintain. Such a long list of requirements being difficult to control that the requirements are actually followed, being even more difficult to actually implement the requirements.

%\textit{Participant \#2} describing software quality as a piece of software being made in such a way that it is robust. And that it is made so that it is easy for other programmers to modify, maintain and repair if it is needed.

%\textit{Participant \#4} has similar reflections on non-functional requirements. However also focus on the manageability of the software being created, that it should be long lasting and be able to be modified over a longer span of time. Also describing that a piece of software is not dependant on 1 person, but the whole developing it. 

%\textit{Participant \#5} describing that NAV has called non-functional requirements as quality requirements or quality properties. Something which was done in the past. Also explaining quality as testable or maintainable.

%\textit{Participant \#7} explaining that in the technical quality of software, some non-functional requirements are more important than others, such as manageability, functionality, usability, maintainability being more important to focus on than other non-functional requirements. 

%\textit{Participant \#8} relating to the reflections of \textit{Participant \#2} in that a piece of software should be easy to modify, and that the software's code should be readable and quick to understand how it functions by just reading it.

%\textit{Participant \#9} explained that he had worked with software quality, but measuring it up against the ISO standards. Including measuring it up against non-functional requirements such as usability, security, re-usability etc. 

%\textit{Participant \#11} describing good usability as important in software quality, that the user should feel that a piece of software is of quality, the visual aspect is perceived as attractive. Also describing software quality is that piece of software is stable, secure and robust.

\section{Technical} \label{sec:technical}
In the interviews of the case study, participants \#1, \#3, \#4, \#5, \#7, \#8, \#9, \#10 described aspects relating to the assurance of the technical quality of the software in the Norwegian public sector. 

\subsection{Security} \label{sec:security}
How quality in security is assured, is explained by participant \#1 to differ between agencies and projects. In the development process of NAV's new systems for distributing parental benefits, how security has been assured in the system has changed. Initially, only specific persons had the responsibility for security, however, when the project was completed in 2019, each team had a specific person with knowledge of security. Other projects such as the "Perform Project" at the Norwegian Public Service Pension Fund had more central control of security in its requirements, which the development teams had to take into account.

%In the development process of NAV's new systems for distributing parental benefits, how security has been assured in the system has changed. Initially only specific persons had the responsibility for security, however when the project was complete in 2019, each team had a specific person with knowledge in security. Other projects such as "Perform Project" at the Norwegian Public Service Pension Fund had a more central control of security in its requirements, which the development teams had to take into account.

The requirements for security at Skatteetaten are described by participant \#7 as being and has always been strict: "We are constantly exposed to attacks, so the requirements and quality of security solutions have always been very high". Explaining that the focus on security and a general lack of development resources has resulted in Skatteetaten prioritising security over other requirements and functionalities. And therefore Skatteetaten's users might have to accept software which lacks usability and functionality. Participant \#7 describes this focus on strict security are the same for NAV, as it is for Skatteetaten.

%The requirements for security at Skatteetaten are and has always been strict, due to Skatteetaten's software being constantly exposed to cyber attacks. This and a general lack of development resources has resulted in Skatteetaten prioritising security over other requirements and functionalities. Meaning that Skatteetaten's users might have to accept software which lacks in usability and functionality. This focus on strict security requirements are described as being the same for NAV, as it is for Skatteetaten.

For Skatteetaten to ensure their strict requirements for security, participant \#8 explains that all events in Skatteetaten's systems are logged. For example, whenever a network call is made or someone does a direct look-up in a database, it is audit logged. Explaining its reason being so that when either someone with authorised permissions is abusing their permissions or someone with unauthorised permissions will be logged and exposed.

%One of the methods that Skatteetaten use to ensure security in their systems, is by all events in their systems being logged. Whenever a network-call is made, someone does a direct look-up in a database etc. It is audit-logged. This so that either someone with authorised permissions is abusing their permissions, or someone with unauthorised permissions will be logged and exposed.

To ensure that quality requirements such as security are met at Mattilsynet, participant \#9 explains that analysis such as \gls{ros_analysis} is used. Also explaining that the \gls{ros_analysis} is used to ensure that other quality requirements than just security are met.

%Analysis such \gls{ros_analysis} is done to ensure that quality requirements such as security is met at Mattilsynet. However \gls{ros_analysis} is also done to ensure that other quality requirements than just security is also met.

%\subsection{Security}
%\textit{Participant \#1} explaining that at the development of the new system for distributing parental benefits at NAV, in the beginning specific persons had the responsibility for security. The project was done in 2019, and at that point each team had a specific person with knowledge in security. 

%\textit{Participant \#1} describing that at the "Perform project" at the Norwegian Public Service Pension fund had central in it project organisation requirements for security, which the development teams had to take into account.

%\textit{Participant \#7} explaining that Skatteetaten takes requirements related to security seriously. Skatteetaten is constantly exposed to cyber attacks, so their requirements on quality related to secure software has always been strict. Thus due to a lack of development resources, means that Skatteetaten has to prioritise security over functionalities for its users, meaning that they will have to make do with something a little simpler or which includes more manual work.

%\textit{Participant \#7} describing that the focus on security is the same at NAV, as at Skatteetaten.

%\textit{Participant \#8} explaining that at Skatteetaten, everything is logged. If someone looks up something, does a network call, we log that person doing it. Audit-logging is also done for direct lookups to a database, so that if you have the correct right's, then you will be logged and exposed.

%\textit{Participant \#9} explaining that at Mattilsynet a lot of ROS analasys is done to ensure quality in secruity, the ROS does also work as a more general ensuring of quality in the software being analysed.

\subsection{Inspection} \label{sec:inspection}
It is the belief by participant \#1, that many development teams in the Norwegian public sector use code inspection as a tool to find faults in their software. As well as new code has to be inspected before it is allowed to be merged into the code repository. Participant \#1 that this was done at the "Perform project" at the Norwegian Public Service Pension fund.

%It is the belief that many development teams in the Norwegian public sector use code inspection as a tool to find faults in their software. As well as new code having to be inspected before it is allowed to be merge into the code repository. This was done at the "Perform project" at the Norwegian Public Service Pension fund.

%When developing a new feature for software at Skatteetaten, at least 2 developers have to be included in a code review process for the developed feature before it is accepted in order to assure that the feature is of quality.

Some developers are hesitant certain code inspection tools to increase software quality, such as developer code review, was explained by participant \#1, due to it being time-consuming: "The whole thing can take a number of days, and the one has written must understand what you did again when you received the results of the inspections". Participant \#1 instead recommending to perform \gls{pair_programming}, as to code would have been inspected then and there.

%Some are hesitant to the use of code inspection tools to increase software quality, such as a developer code review, due to it being time consuming. When asked to perform a code review, a developer has to stop what they are doing, review the code, give their toughs, send the review and repeat the process until the code is of satisfactory quality. The developer then has to get back to the task he/she was doing before the review. This can lead to a code review taking multiple days. The developers could instead have performed \gls{pair_programming}, and the code would be reviewed then and there. 

Mattilsynet's developers recently started to use static code analysis tools. And participant \#10 explained how this is said to find 145 critical faults, enabling Mattilsynet to know that these faults exist, and can start processes to handle them. Something which is in contrast with an alleged study from Microsoft, described by participant \#1. This is said to conclude that such code inspection tools only find a limited amount of critical faults. Mostly finding faults in code style and not much else.

%Recently the developers at Mattilsynet has increased their use of static code analysis tools such as SonarQube. A static code analysis of an large old system at Mattilsynet is said to find 145 critical faults. Which enables Mattilsynet to know that these faults exists and can stat processes to fix these faults. However this is in contrast with an alleged study from Microsoft, which is said to conclude that such code inspection tools only find a limited amount of critical faults. Mostly finding faults in code style and not much else.

%It is also described that inspection through monitoring metrics is a bit weak at Mattilsynet, that it might be to a better standard at other organisations such as Norsk Tipping.

%\textit{Participant \#1} describing that at the "Perform project" at the Norwegian Public Service Pension fund, they used code inspection as a tool to ensure quality.

%\textit{Participant \#1} explaining that some are critical to some inspection techniques, due to it being time consuming. A developer has to stop what they are doing to look at a other developers code, review it, give their thoughts, send in a review, and repeat the process. After the code is approved, the developer then has to spend time on returning to what he/she was working on before, which leaves the whole process to take up to multiple days. If the developers instead had been \gls{pair_programming}, the code would be inspected then and there.

%When developing a new function, at least 2 people have to look at the developed function, which introduces a QA-process. 

%\textit{Participant \#1} explaining that a study from Microsoft found that there were only a few critical faults that were found in code inspections. Mostly faults relating to code style were found, not so much critical faults.

%\textit{Participant \#1} describing that it is their belief that many teams does code inspections, and in some organisations the code has to be inspected before it is allowed to be merged into the code repository.

%\textit{Participant \#10} explaining that they have increased the use of static code analysis at Mattilsynet. And had recently scanned a large old system by using SonarQube which he believes to have found about 145 critical faults. This enables Mattilsynet to know that these faults exists and that they can start fixing them, which is a huge step from not knowing that they exist. 

%\textit{Participant \#10} explaining that when it comes to inspection through monitoring and monitoring metrics, it is a bit weak at Mattilsynet, the participant further explaining that he is used to a higher standard at places such as Norsk Tipping.

\subsection{Testing} \label{sec:testing}
The use of regression testing is described by participant \#1 as used by most agencies in the Norwegian public sector to ensure quality in their software. As well as trying to automate as much of the testing as possible. Yet some of the testing activities described by participant \#1 are to still be done manually, such as exploitative testing and involving the customer in the testing activities.

%Allegedly most agencies in the Norwegian public sector use regression testing to ensure software quality in their software. As well as trying to automate as much of the testing as possible. However some of the testing activities are still done manually, such as exploitative testing and involving the customer in the testing activities. 

At the "Perform project" at the Norwegian Public Service Pension fund, it was described by participant \#1 that supporting systems for testing their solution with simulated data were created. And at the development of the new system for distributing parental benefits at NAV, more traditional methods were used to test the system with simulated data.

%At the "Perform project" at the Norwegian Public Service Pension fund, supporting systems for testing their solution with simulated data were created. While at the development of the new system for distributing parental benefits at NAV, more traditional methods were used to test the system with simulated data.

Writing tests is seen as important by participant \#3 when trying to include proper continuous delivery in a software development project: "It's about making sure that you've spun a good enough safety net around the code you write so that the probability of you letting something go wrong is as small as possible. It will never be zero, but you have to make it as good as possible". Further explaining that writing tests are also important for the ability to more securely deliver smaller and more frequent changes to the software being created.

The importance of writing tests is shared with participant \#8. Describing that to be able to modify a piece of software, the software needs to have high test coverage. A low-level method is first tested using unit-tests, then the whole software unit or application of which the method is a part is tested: "In a \gls{microservices} architecture where you have interaction and values going through, you have to tests at all levels. To make sure things work and nothing breaks if you go in and make a change".

%It is described by participant \#8, that to be able to modify a piece of software, the software needs to have a high test coverage. A low-level method is first tested using unit-tests, then the whole software unit or application which the method is a part of is tested. If using a \gls{microservices} where there is interactions and values between the applications, then testing is needed at all levels. This to ensure to ensure that all works as intended and modifications in a piece of software do not lead to any faults. High test coverage is also described as the most important measure to ensure that a piece of software can be modified over a longer period of time.

%Writing tests is important when trying to include proper continuous delivery in a software development project. These tests are written to ensure that there is a web of security around the software being create, so that the chance of something faulty being delivered is as low as possible. The chance of faults will never be zero, but it its important to get it as low as possible, within reason. Writing these tests are also important for the ability to more securely delivering smaller and more frequent changes to the software being created.

The degree of automated tests is described by participant \#7 to be high at Skatteetaten, and it is years since Skatteetaten stopped performing acceptance tests. Participant \#7 further explains that Skatteetaten has had to experiment with writing the correct amount of automated tests: "Automated testing is that you shouldn't have too much, but you shouldn't have too little either. If you have too much, it becomes a big management task. ... so it has been challenging to find the correct balance".

%Skatteetaten have a high degree of automated tests, and there are years since they stopped performing acceptance tests. Skatteetaten has also had to experiment with writing the correct amount of automated tests. If there for instance are too many automatic tests, it becomes a large task to maintain all the tests. Therefore there has been a challenge for Skatteetaten to find the balancing point on what is the right amount of automated tests.

%\textit{Participant \#1} explaining that most agencies in the Norwegian public sector use regression testing to test their software, and are trying to automate as much as possible in testing. However some testing is still done manually, such as exploitative testing and having the customer involved in the testing activities.

%\textit{Participant \#1} describing that in the "Perform project" at the Norwegian Public Service Pension fund extra systems for testing their solution with simulated data. While at the development of the new system for distributing parental benefits at NAV, more traditional methods were used to test the system with simulated data.

%\textit{Participant \#3} explaining that continuous delivery is something which takes a lot of effort to get working properly, which includes writing tests. Tests are being written to make sure that there is a web of security around the software being created, so that the chance of something faulty being delivered is a low as possible. The chance of faults will never be 0, but trying to get it as low as possible within reason. 

%\textit{Participant \#4} explaining that in order to deploying smaller and more frequent changes to a piece of software, automated quality checks are needed.

%\textit{Participant \#6} explaining that in order to create software that is able to change a lot has to be in place, such as testing. Testing being the most important measure to make sure that a piece of software can change over time.

%\textit{Participant \#7} explaining that Skatteetaten have a high degree of automated tests, and do perform any accepts tests, which were years since last were done. Skatteetaten has also experimented with the use of automated testing, such as the amount of automated tests being written is the correct amount. If to many are written it becomes a large task to maintain all the automated tests. So it has been a challenge for Skatteetaten to find the balancing point on what is the right amount of automated tests.

%\textit{Participant \#8} explaining that to be able to change a piece of software, the software needs to have a high test coverage. A method is tested on the lower level with unit-tests, then the whole unit or application which the method is a part of is tested. If using a \gls{microservices} where there is interaction and values between the application, then testing is needed at all levels. This to ensure that everything works as intended and changes in a piece of software does not lead to any faults which leads to the software crashing.

\subsection{Technical Debt} \label{sec:technical_debt}
The ability to register \gls{technical_debt} is described by participant \#1 as an important measure to increase software quality. This is important as it enables knowledge of what should be improved in a piece of software, due to it uncovering potential faults and vulnerabilities in a piece of software.

%An important measure to increase software quality is the ability to register \gls{technical_debt}. This is important as it enables knowledge in what should be improved in a piece of software, due to it uncovering potential faults and vulnerabilities in a piece of software.

NAV is described by participant \#3 as having a large variety of software in their organisation to aid them in their operations: "Some systems that are 40 years old, some that are being written for the first time now, and everything in-between". Further explaining that the age of the system decides how quality should be assured: "If you are going to make a change in a system that is 20 years old, then there is bound to be some manual testing in place. You have to have your own test environment ... It prevents us from achieving quality assurance". Participant \#3 explained that this has led to NAV modernising their old software, which usually means creating new software to replace the old software and switching off the old software.

%NAV has a large variety of software in their organisation to aid them in their operations. Some of the software is over 40 years old, some are being created from the ground up right now and some in-between. This means that NAV use different methods to modify their software based on the particular piece of software. If the software is for instance 20 years old, a full testing environment with manual tests are needed, creating a lot of overhead, prohibiting securing quality in the software. This leads to the need for NAV to modernise their old software, which usually means creating new software to replace the old software, and switching off the old software.

This process of replacing old software is explained by participant \#8 to be similar at Skatteetaten. As Skatteetaen have old software, the re-writing and renewing of this old software is a challenge due to bad quality in the old software. Whereas the new software that Skatteetaten is creating is of good quality. The old software of Skatteetaten is being modernised slowly and is switched off as they are being modernised by newer software.

%Skatteetaten also have old software, and in the re-writing and renewing of this old software is a challenge due to bad quality in the old software. Whereas the new software that Skatteetaten is creating is of good quality. The old software of Skatteetaten is being modernised slowly, and is switched off as they are being modernised by newer software.

NAV is also described by participant \#3 as trying to modernise their old software, without creating new software and switching the old software off. Instead modernising the old software while still in service. NAV's system for delivering pension benefits was initially created by consultants from 2007 to 2011, and in 2017 was described by Participant \#3 as being technical bankrupt: "We just have to throw that away, because we can't change it anymore. It takes half a year to change a comma". Yet by working on and improving the pension benefits system for five years, participant \#3 explained that it is not at the point where changes are deployed daily. And creating modifiable software is important, it does not matter how much testing has been done upfront, the requirements of the system will change from when it was ordered to when it is delivered, and after the users start using it.

%NAV is also trying to modernise their old software without creating new software and switching the old software off, instead modernising the old software while still in service. NAV's system for delivering pension benefits was initially create by consultants from 2007 to 2011, and which in 2017 was described as being technical bankrupt. Participant \#3 describing its state as "It took half a year to change a comma." However by working on and improving the pension benefits system for five year, it is now at the point where changes are deployed daily. Creating modifiable software is important, it does not matter how much testing has been done upfront, the requirements of the system will change from when it was ordered to when it is delivered, and after the users start using it.

Some older systems at NAV have been described as a challenge by participant \#5, as they do not have automated tests, requiring more manual quality assurance methods. Explaining that these systems also have a lower rate of deployment, such as the system called ARENA, NAV's system for delivering daily allowance benefits. Its deploy frequency has changed from 4 deploys to 12 deploys a year. The deployment frequency has increased somewhat, but still being in-frequent due to the need for manual quality assurance methods.

%There is a challenge at NAV, that there is old systems without automated tests, requiring more manual quality assurance methods. These systems also have a lower rate of deploy, such as the system called ARENA, NAV's system for delivering daily allowance benefits. It's deploy frequency has changed from 4 deploys to 12 deploys a year. The deploy frequency having increased somewhat, but still being in-frequent due to the need for manual quality assurance methods.

Mattilsynet is described by participant \#9 to have some older systems, however, with the new product teams organisation recently introduced, it has given better focus to these older systems. The new systems created at Mattilsynet are created with better quality, yet that are still issues with the bad quality of their older systems when it comes to usability and \gls{technical_debt}.

%Mattilsynet do also have some older systems, however with the new product teams organisation recently introduced, it has given better focus to these older systems. The new systems created at Mattilsynet are created with better quality, yet that are still issues with bad quality of their older systems when it comes to usability and \gls{technical_debt}.

The ability of software developing teams to understand the necessity of maintenance and handling of \gls{technical_debt} is described as an issue by participant \#4. Development teams should use 20-25\% of their time to improve \gls{technical_debt}, as \gls{technical_debt} is the opposite of software quality. The challenge is to make the development teams dedicate that amount of time to \gls{technical_debt}.

%The ability for software developing teams to understand the necessity of maintenance and handling of \gls{technical_debt} is an issue. Development teams should use 20-25\% of their time to improve \gls{technical_debt}, as \gls{technical_debt} is the opposite of software quality. The challenge is to make the development teams dedicate that amount of time to \gls{technical_debt}.

An alleged Gartner report on Mattilsynets system "Mats" in 2012 was mentioned by participant \#10 and described that "Mats" will cost Mattilsynet 10 Billion NOK over the next 15 years. And its consequences for Mattilsynet are described by Participant \#10 as being negative: "That money doesn't exist, so it can't happen. But the consequences are that the system loses functionality over time because the environment changes". Participant \#10 explained that functionality loss has already occurred, when a data format from the Norwegian Directorate of Agriculture changed, due to Mattilsynet being unable to modify "Mats" to accept the new data format.

%A Gartner-report on Mattilsynets system "Mats" in 2012 allegedly described that "Mats" will cost Mattilsynet 10 Billion NOK over the next 15 years. Those 10 Billion NOK do not exist, so Mattilsynet do not have the opportunity to bear the cost of this. The consequences of this is that "Mats" will loose functionality as time goes on just from the world around it changing. "Mats" also has integration's with other agencies in the Norwegian public sector. When a data format from the Norwegian Directorate of Agriculture changed, "Mats" lost functionality, due to Mattilsynet being unable to modify "Mats" to accept the new data format.

%\textit{Participant \#1} explaining that being able to register \gls{technical_debt} can be important for software quality as it enables knowledge in what should be done in the future and having knowledge on potential vulnerabilities in a system.

%\textit{Participant \#3} explaining that NAV has a large variety of software in their organisation. Some software is over 40 years old, while some is being created from the ground up now and all software in-between. And this variety in age of software means that NAV has to have different methods for how changes in its software is done, based on that piece of software. If the software is old, maybe 20 years old, manual testing is needed and a full test environment is needed, thus creating a lot of overhead which prohibits securing quality in the software. This leads to the need for modernisation of the old software, which for NAV means creating new software and turning off the old.

%\textit{Participant \#3} also explaining that NAV is trying to save old systems without turning them off, instead modernising them while still in service. The system for delivering pension benefits was created by consultants from 2007 to 2011, and which in 2017 was described as technical bankrupt. "It took half a year to change a comma". However by working at it and improving for five years and is now able to have changes to the system deployed daily. Creating software with the ability to change is important, it does not matter how much testing has been done upfront, the needs of the system will change from when it was ordered to when it goes live, and after it has gone live.

%\textit{Participant \#4} explaining that maintaining software is an issue. The development teams should use 20-25\% of their time on improve \gls{technical_debt}, as \gls{technical_debt} is the opposite of software quality. The challenge is to make the development teams spend so much time on \gls{technical_debt}.

%\textit{Participant \#5} explaining that NAV has a challenge with old systems without automated tests, requiring more manual quality assurance methods. These systems also have a lower rate of deploys, such as the system for delivering daily allowance benefits, called ARENA. It has gone from 4 deploys a year to 12 deploys. The frequency of deploys has increased some, but are still in-frequent due to the need for more manual quality checks.

%\textit{Participant \#8} explaining that Skatteetaen has some old software, and it is the re-writing and renewing of these systems that is a challenge in bad quality. The new systems that Skatteetaten is creating is of good quality, but the old has some quality issues. So the old software is being modernised bit by bit, and is turned of as they are being modernised.  

%\textit{Participant \#9} explaining that Mattilsynet has some older systems, however with the new product teams organisation, it has given a better focus on this old software. The new systems are made with better quality, however there are still issues with quality in the older systems.

%\textit{Participant \#10} explaining that a Gartner-report on Mattilsynet's system "Mats" i 2012 explained that the system will cost Mattilsynet 10 Billion NOK over the next 15 years from 2012. That money does not exist, so Mattilsynet cannot let that happen. However the consequences of this is that the system looses its functionality as time goes on, just from the world around it changing. "Mats" has integration with other systems in the Norwegian public sector. A data format from the Norwegian Directorate of Agriculture changed led to "Mats" loosing functionality, due to Mattilsynet being unable to update "Mats" to accept the new data format.

%\textit{Participant \#11} explaining that Mattilsynet has older software which are good in quality when it comes to downtime and other operational requirements. However they are lacking in quality when it comes to \gls{technical_debt} and usability. The new applications of Mattilsynet has better usability and has overall better quality.

\subsection{Measurements} \label{sec:measurements}
The systems NAV uses in their operations are described by participant \#5 as having a strong focus on measurements, being that they are measured in how they behave, wherein the system is there downtime and where there is no downtime, are there any response problems, etc. Participant \#5 also explains that it is up to each team at NAV how they measure their systems when in operations, while it is encouraged that they do, have a good understanding of what state their system is in at all times.

%The system NAV uses in their operations have a strong focus on measurements, being that they are measured in how they behave, where in the system are there downtime and where are there not downtime, are there any response problems, etc. It is up to each team at NAV how they measure their systems when in operations, while it is encourage that they do, to have a good understanding of what state their system is at all times.

It is the belief of Participant \#7 that the software at Skatteetaten is of few faults. Giving an example of the process for calculating how much taxes are owed in Norway. First Skatteetaten calculate how much each citizen in Norway owes them, or how much Skatteetaen owe the citizen. Then this amount is sent out to each citizen of Norway, which only takes place once a year. Each time this takes place, Skatteetaten measures faults in the systems responsible for this process. Participant \#7 explains that the faults measured then get fixed, and the next year none of these faults appear again: "Then we measure mistakes, and what we see is that the mistakes we make one year, we fix them, and we don't make the same mistakes the following year".

%It is the belief that the software at Skatteetaten are of few faults. For instance if looking at the process for calculating how much taxes is owed in Norway. First Skatteetaten calculate how much each citizen in Norway owe them, or how much Skatteetaen owe the citizen. Then this amount is sent out to each citizen of Norway, which only takes place once a year. Each time this takes place, Skatteetaten measure faults in the systems responsible for this process. The faults measured then get fixed, and the next year none of these faults appear again. New faults may arise the next year, but the system is still descibed as of high quality due to their measurements.

%\textit{Participant \#5} explaining that there is a strong focus on measurements, that systems are being measured on how they behave, where are there downtime, where are there not downtime, are there any response-problems. It is up to each team at NAV how they measure their software, but it is encourage that they do, to understand what state their software is in and have a good understanding of it.

%\textit{Participant \#7} explaining that his impressions that that there are little faults in the software at Skatteetaten. If looking that the processing for calculating how much taxes is owed in Norway and the amount is sent out to the citizens of Norway, which only happens once a year. During each time, the faults is being measured and fixed. The next year the same faults are not measured, new faults arise. However it is still of high quality.

\subsection{Software Properties} \label{sec:software_properties}
A property described by participant \#9 as being an indicator for a piece of software to be of good quality, is that the data it serves is easily accessible. Explaining that if systems quality at Mattilsynet should be assessed today, it would be based on how its data interfaces with other applications. And such quality assurance methods are described as important by participant \#9: "It is extremely important to ensure that you have quality in your solution". Participant \#9 explains that today it is expected for an application to have accessible APIs so that it is predictable and easy to retrieve its data. Something that is not possible to achieve without following all aspects of good software quality, such as correct architecture and mindset in building a solution. 

%A property described as being an indicator for a piece of software to be of good quality, is that the data it serves is easily accessible. If a systems quality at Mattilsynet should be assessed today, it would be based on how its data interfaces with other applications. Today it is expected for an application to have accessible API's so that it is predictable and easy to retrieve its data. This is something which is not possible to achieve without following all aspects of good software quality, such as correct architecture and mindset in building a solution.

This description of software quality has also led to new initiatives being started at Mattilsynet. Participant \#10 explains that when a new piece of software is created, its API is created first. This is done by creating API contracts using \gls{open_api} and designing the API from these contracts. Further explaining that this form of API-contract-creating strategy is something that the product teams at Mattilsynet are going to start doing. Enabling the product teams to discover an API that has already been created by another team, preventing duplicate work.

%This description of software quality has also led to new initiatives being started at Mattilsynet, where when a new piece of software is create, it's API is created first. This is done by creating API-contracts using \gls{open_api}, and designing the API from these contracts. This form of API-contract creating strategy is something that the product teams at Mattilsynet is going to start doing. Thus enabling the product teams to discover in an API has already been created by another team, preventing duplicate work.

The use of other data-sharing patterns, such as the \gls{publish_subscribe} pattern, is described by participant \#10 to increase the quality of Mattilsynets services. Described as enabling Mattilsynet to share sets of data between applications through data streams, both internally and externally of Mattilsynet. Participant \#10 provides an example with their partner Nortura: "Instead of asking Mattilsynet iterating over an API and asking for all these animals that they intend to slaughter have some restrictions. They can rather get the data stream from us, so then we don't have to ask". Creating a loose coupling between Mattilsynets and Norturas software, meaning that Norturas software will not fault if Mattilsynets software is unavailable.

%Mattilsynet also using other data-sharing patters to increase the quality of their services, such as the \gls{publish_subscribe} pattern. This enables Mattilsynet to share sets of data between applications through data-streams, both internally and externally of Mattilsynet. For instance when Nortura wanted data from Mattilsynet on what restrictions an animal has, such as what medication the animal is using. Instead of constantly asking Mattilsynet for this data, Nortura can simply retrieve one of Mattilsynets data-streams on the API-level. This also has the effect of creating a loose coupling between Mattilsynets and Norturas software, meaning that Norturas software will not fault if Mattilsynets software are unavailable.

%\textit{Participant \#9} explaining that easily accessible data is important for software to be of good quality. How data is store and retrieved in a system is important. If a systems quality should be assessed today it would be based on how its data interfaces with other applications. Today it is expected for an application to have accessible API so that its easy to get its data, and its not possible with everything that follows in good software quality such as the correct architecture and mindset on building a solution.

%\textit{Participant \#10} explaining that an initiative has started at Mattilsynet when a new piece of software is created, the API is created first. This by creating API-contracts using OpenAPI, and designing the API from this. This form of API-contract creating strategy is something that the product teams at Mattilsynet is going to start doing. This enables the product teams to discover if an API has already been created by another team, preventing duplicate work.

%\textit{Participant \#10} also explaining that the publish-subscribe pattern is used to share sets of data between applications through data-streams, both inside and outside Mattilsynet. There was a case where Nortura wanted data from Mattilsynet on what restrictions an animal has, such as what medication an animal is using. Instead of asking Mattilsynet for this data, Nortura can just retrieve one of Mattilsynets data-streams on the API-level. This also gives a loose coupling between Mattilsynet and Norturas software, meaning that Norturas systems will not crash if Mattilsynets systems are down.

\section{Methodological} \label{sec:methodological}
In the interviews from the case study, participants \#1, \#2, \#3, \#4, \#5, \#7, \#8, \#9, and \#10 described aspects relating to software development methods used to assure software quality in the Norwegian public sector.

\subsection{Agile} \label{sec:agile}
The Norwegian public sector is described by participant \#2 as not being concerned with assuring software quality strict regimes, with projects which were quite common before. Instead describing it to be based on getting feedback from its users. If the user does not experience any faults with the software, then it is of high quality. If the user finds that the software is of value, then it is of high quality.

%The Norwegian public sector is not concerned with assuring software quality through strict regimes, through projects which were quite common before. However based upon getting feedback from its users. If the user do not experience any faults with the software, then it is of high quality. If the user find that the software is of value, then it is of high quality.

Retrieving continuous feedback on a piece of software is a method described by participant \#1, used to assess the quality of the piece of software. Explaining that the method for receiving feedback through demonstration of the software being developed, and at each demonstration retrieving feedback on the quality of the software. Is a method which was performed frequently during the development of the "Perform project" at the Norwegian Public Service Pension fund.

%Retrieving continuous feedback on piece of software is a method which is used to assess the quality of the piece of software. A method for feedback is to demonstrate the software being developed, and at each demonstration retrieve feedback on the quality of the software. Such demonstrations were also performed frequently at the development in the "Perform project" at the Norwegian Public Service Pension fund.

The principle of retrieving continuous feedback is described as a challenge by participant \#2. As the citizens of Norway usually are not interested in giving feedback, only being interested in the software functioning correctly. And explaining that these challenges increase as handling proper feedback in a public sector from its citizens is important due to the democratic principle of the public sector having power over its users. Participant \#2 describes this challenge as larger in the public sector than in the private sector. Due to the relationship of power being different, and therefore the public sector users have to trust that the particular public sector will not use the feedback data against their users.

%The principle of retrieving continuous feedback in the Norwegian public sector can be a challenge, as the citizens of Norway usually are not interested in giving feedback, only being interested in that the software functions correctly. This challenge increases as handling proper feedback in a public sector from its citizens is important due to the democratic principle of the public sector having power over their users. Also for the users to be willing to give their feedback in a public sector, the users have to trust that the particular public sector will not use the data against their user. This is not seen as a challenge in the private sector, due to them not having the same power over their users.

Other uses of agile practices to ensure software of high quality, are described by participant \#1 to be used at NAV in their project to develop a new system for distributing parental benefits. The same agile practices are described by participant \#2 to be used at Entur to ensure software of high quality. Participant \#8 explains how the agile practices of delivering MVPs, development velocity, and how each release is to be published. Is used to ensure that Skatteetatens software is of high quality.

%In NAV's project to develop a new system for distributing parental benefits, most agile practices where followed in order to obtain good quality in the systems software. Such agile practices are also used by Entur to ensure good quality in their software.

%At Skatteetaten, agile methods are used to ensure that their software is of high quality, including delivering MVP's of the software to ensure that their software delivers the value as expected. Including in these agile methods are development velocity, what is happening in the development of a piece of software, and how each new release is to be published. 

It is described by participant \#9 that Mattilsynet could have more established routines for assuring quality in its software. Explaining that Mattilsynet's current focus is on development speed: "Mattilsynet are a very young organisation when it comes to development ... there has been a focus on productivity and getting the solution out and testing it, rather than on routines and quality". Describing once development speed is at a satisfactory level, the focus will be shifted from development speed to quality, testing and management of Mattilsynets software.

%It is described that Mattilsynet could have more established routines for assuring quality in its software. However Mattilsynet is a young organisation when it comes to software development, and is currently most focused on increasing the development speed for its new product teams and increasing productivity. Once this is at a satisfactory level, the focus will be shifted from development speed to quality, testing and management of Mattilsynets software.

%\textit{Participant \#1} explaining that by using agile methods you demonstrate the software being developed, and at each demonstration feedback will be given about the software quality.

%\textit{Participant \#1} explaining that at the perform project at the Norwegian pension fund, they had frequent demonstrations of the system they were creating.

%\textit{Participant \#1} explaining that in agile methods it is primarily a customer which is receiving a product, and that it is primarily the product owner which follows the development team, and must be able to give feedback on the most important aspects in software quality, and if it is good enough.

%\textit{Participant \#1} explaining that in agile methods, it is meant to be used as a method to maintain high software quality throughout the lifetime of the software. That you should not compromise on quality to quickly finish a feature, instead compromising on the amount of user stories or features of the software.

%\textit{Participant \#1} explaining that in NAV's project to develop an new system for distributing parental benefits, most agile practices were followed in order to obtain good quality in the systems software.

%\textit{Participant \#1} explaining that in agile methods there should not be a head of architecture who controls everything. Rather the head of architect should have dialogue with each team and should with the teams agree on good architectural decisions. 

%\textit{Participant \#2} explaining that in the democratic principle where a country has power over their users, the country has to make sure they treat their feedback from their users properly. However this can be a challenge, as not many users in the Norwegian public sector wants to give their feedback, they just want the software to work.

%\textit{Participant \#2} also explaining that the Norwegian public sector is not concerned about assuring quality through strict regimes, like through projects which were very common before. However now it is based upon getting feedback from the users. If software do not crash on the user, the quality is high, if the user find it valuable and good, it is of good quality.

%\textit{Participant \#2} also explaining that a challenge in the Norwegian public sector is to gather data about their users. The users must be willing to tell if they are satisfied, and since the Norwegian public sector have power over their users, its important that the user trusts the public sector to give away their data, and their feedback will not be used against them. This is easier in the private sector as they do not have the same power over their users.

%\textit{Participant \#3} describing that software quality does not have so much to do with ISO standards, but being able for a team to be able to change in their development of a system. Setting ability to change as similarities to quality in the software NAV creates. This because the development teams might not know what is correct or best when developing software in a world which is always changing. The development teams does not want any faults in their software, however since the world around the software is always changing, the teams have to guard against eventual faults and ensure that the consequences are as low as possible.

%\textit{Participant \#8} describing that software quality is described quite simply, how simple is it to modify the software. Also describing that in Skatteetaten agile methods to ensure quality, including delivering MVP's of the software and ensuring that their software delivers as expected.

%\textit{Participant \#9} explaining that an important measure to ensuring software quality is development speed. What is happening in the development of a piece of software and how is new releases published.

%\textit{Participant \#9} explaining that Mattilsynet could have better routines for assuring quality in its software. However Mattilsynet is a young organisation when it comes to software development and is currently most focused on development speed for its new product teams and focus on productivity and releasing solutions, instead of quality and testing. It will then become a challenge to change from this focus on development speed to quality, testing and managing its software.


\subsection{DevOps} \label{sec:devops}
There are allegedly studies on the difference of the review of software through demonstration or deployed software described by participant \#1. Where the alleged studies conclude with the customer and users find more faults when the software is deployed, rather than in a demonstration. This is because when the software is presented in a demonstration, the context is artificial. Participant \#1 explains as this is because user reviewing the software does not get to experience how it works in their own organisation or in their day-to-day life. Hence other aspects of quality are reviewed when the software is deployed and available to the user in their normal context.

%There are allegedly studies on the difference of the review of software through demonstration or deployed software. And such studies concluding with the customer and users finding more faults when the software is deployed, rather than in an demonstration. This because when the software is presented in an demonstration, the context is artificial. The user reviewing the software does not get to experience how it works in their own organisation or in their day to day life. Hence other aspects of quality is reviewed when the software is deployed and available to the user in their normal context.

The use of DevOps to receive feedback about quality in a piece of software is described by participant \#1 to be used at NAV in the development project for disbursement of parental benefits. Explaining that the project was initially planned to be delivered through three large deployments, with the deployment plan changing for the last delivery: "They combined both development resources and business resources and where they then worked towards having a continuous deployment". And participant \#1 explained that this change gave the teams more control, due to the feedback from the users being more authentic on a wider range of quality aspects. 

%NAV uses continuous deployment to gain feedback about quality in their software. In the development project for the disbursement of parental benefits, it was initially planned to be delivered through three large deployments, however for the last deployment the delivery plan was changed. For the last deployment, the teams were combined with both software development and management resources, and continuous deployment was performed instead of a single deployment. What NAV experienced from this change in the project, is that it gave the teams more control, due to the feedback from the users being more authentic on a wider range of quality aspects. 

For ensuring the development of robust software, participant \#4 describes the need for continuous development. As well as high development speed- and flow are needed for the development teams to ensure that other quality aspects of the software are high when being developed. Further explaining that if software is not maintained it will naturally decrease in quality. Leading to any eventual fault in their software needs to be fixed quickly to ensure that the software constantly is of quality. It is for these reasons that the development teams at NAV follow the DevOps Principles, explained by participant \#4.

%For the ensuring of developing robust software, continuous deployment of that software will need to be performed. To ensure that other quality aspects of the software is high when being developed, high development speed and good development flow is needed for the development teams. Also if the software is not maintained, it will naturally decrease in quality. Any eventual fault in their software needs to be fixed quickly as to ensure that the software constantly is of quality. It is for these reasons that the development teams at NAV follow the DevOps Principles.

For helping the development teams at NAV follow the DevOps principles, participant \#4 explained that NAV has created its own application platform, \gls{nais}. With \gls{nais} being designed in a way which gives strong recommendations on how software should be developed and deployed to NAV's infrastructure, including standardised services for typical NAV applications. Participant \#4 describes this as enabling software of high quality: "The sum of \gls{nais} makes it easier to create good software, which leads to increased quality. The teams can make their own choices, but it is recommended to make good choices with \gls{nais}".

%Helping the development teams of NAV to follow the DevOps principles, NAV have created their own application platform, \gls{nais}. \gls{nais} is designed in a way which gives strong recommendations on how software should be developed and deployed to NAV's infrastructure, including standardised services for typical NAV applications. The sum of \gls{nais}, is that it makes it easier to create software at NAV, which increases software quality. The teams at NAV are not forced to followed the recommendations of \gls{nais}, however are encouraged to make good decisions. These are important incentives for the development teams themselves to operate the software they have developed, resulting in increased software quality.
During the development of software, the negative consequences are described by participant \#3 to be as low as possible, needing to be handled with continuous delivery. Explaining that in the past,  NAV had 4 large releases every year for their new software, where the largest release included about 116 000 development hours. The only way to ensure that 116 000 development hours do not cause negative consequences, is to perform a large number of manual tests and a large number of checklists. Participant \#3 now believes that more frequent and smaller releases decrease the consequences of eventual faults at NAV. This is why NAV has gone from 4 releases a year to 1500 releases a week.

%During the development of software, the negative consequences has to be as low as possible, which needs to be handled with continuous delivery. In the past, NAV had 4 large releases every year for their new software, where the largest release included about 116 000 development hours. The only way to ensure that 116 000 development hours do not cause negative consequence, is to perform a large amount of manual tests and a large amount of check lists. However NAV believes that more frequent and smaller releases decreases the consequences of an eventual fault. This is why AV has gone from 4 releases a year to 1500 releases a week.

The teams themselves at Skatteetaten are described by participant \#7 as having the responsibility for delivering production-ready software. And that Skatteetaten has worked with continuously deploying production-ready software since 2013. Explaining that Skatteetaten has a focus on deploying smaller changes early: "This is all about getting things out early. If there is an error, it is sometimes limited in scope and we can quickly roll back or fix it". This as large and infrequent releases need large acceptance tests, resulting in large faults.

%The teams themselves at Skatteetaten have the responsibility of delivering production ready software, where from 2013 Skatteetaten have worked with continuously deploying production ready software. And focusing on deploying code that is at user-story level, not at the delivery level. This to be able to deliver production ready software as early as possible. If there are any faults in the software, it is most likely of limited scope, and the development team is able to quickly roll back the software to an earlier versions and fix the fault. This is in contrast to large and infrequent releases, where large acceptance tests are needed, resulting in large faults.

New software being developed at Mattilsynet is described by participant \#10 as being created using the \gls{microservices} or \gls{service_oriented_architecture}. Which is then deployed to Mattilsynets own \gls{kubernetes} cluster, which gives a declarative way of deploying software. And explaining that Mattilsynet has a set of requirements for their software to be deployed to their servers. Being that the code has to stored in a GitHub repository which is a part of Mattilsynets GitHub organisation. Enabling Mattilsynet's platform team to know what the code is, and what it does, and are able to scan the code for different reasons.

%New software being developed at Mattilsynet is created using the \gls{microservices} or \gls{service_oriented_architecture}. Which is then deployed to Mattilsynets own \gls{kubernetes} cluster, which gives a declarative way of deploying software. Mattilsynet also have a set of requirements for their software to be deployed to their cluster. The code has to be stored in a GitHub repository which is a part of Mattilsynets GitHub organisation. This for Mattilsynets platform team to know what the code is, what it does, and are able to scan the code for different reasons.

%\textit{Participant \#1} explaining that there has been done studies on the difference of demonstration of deployment of software. And the studies concluding with the customer and users finding more faults when the software is deployed, instead of in demonstrations. This because when the piece of software is being presented in an demonstration, the context is artificial. The user testing the software does not get to experience if it works in their own organisation or how it works with other tools used in their normal life. Hence a other kind of quality is tested when the software is deployed and available to the user in their normal context.

%\textit{Participant \#1} further explaining that NAV uses continuous deployment to gain feedback on its software. And in the project for parental benefits, the project was initially planned to be three large deployments, however for the last deployment the project switched tactics. For the last deployment, where teams were combined with both development and management resources, where continuously deployment was performed. And thinking that this gave the teams more control, due to the feedback from the users being more authentic on a wider range of quality aspects.

%\textit{Participant \#3} explaining that in the development of software, the negative consequences has to be as low as possible. This needs have to be handled with continuous delivery. In the past, NAV had 4 big releases every year, where the largest release had about 116 000 development hours. The only way to ensure that 116 000 development hours does not cause negative consequences is to do a lot of manual testing, a lot of check lists. However NAV believes that doing more frequent releases will decrease the consequences of an eventual fault. This is why NAV has gone from 4 releases a year to 1500 releases a week.

%\textit{Participant \#4} explaining that to ensure the development of robust software it needs to be continuously developed. If not maintained it will naturally decrease in quality. Also explaining that in order to create software of quality, speed and good flow is needed for the development teams. The teams at NAV follows the DevOps principles.

%\textit{Participant \#4} explaining that NAV has their own application platform, \gls{nais}. \gls{nais} is designed in a way which gives strong recommendations on how software should be developed and deployed to NAV's infrastructure. The sum of \gls{nais} is that it makes it easier to create software at NAV, which increases quality. The teams at NAV are not forced to follow the recommendations of \gls{nais}, however are encourage to make good decisions. These are important incentives for the teams to operate the software they develop, as this increases quality.

%\textit{Participant \#4} also explaining that to increase quality it is important to be able to fix faults quickly. Faults will always happen, so it is important to be able to fix these faults quickly.

%\textit{Participant \#5} explaining that NAV has a whole range of software, from COBOL software which is deployed to mainframes, to software deployed to the \gls{nais} platform. NAV works the most with software that is deployed to the \gls{nais} platform, where the platform itself maintains a higher level of quality by introducing standardised services and expectations on how software should be deployed.

%\textit{Participant \#7} explaining that at Skatteetaten, the teams themselves has the responsibility of delivering production ready software, where from 2013 they have worked with continuously deploying production ready software. And deploying code that is at user-story level, not at the delivery level. All of this to deliver production ready software as early as possible. If there are any faults, it is most likely of limited scope, and the team is able to quickly roll back to an earlier version of the software and can fix the fault quickly. This in contrast to large and infrequent releases, where large acceptance tests are needed, which gives large faults.

%\textit{Participant \#10} explaning that if not looking at its older software, Mattilsynets newer software is build using the \gls{microservices}s or \gls{service_oriented_architecture}. The software is deployed to a \gls{kubernetes} cluster, which gives a declarative way of deploying software. Mattilsynet has a set of requirements for their software to be deployed to their cluster. The code has to be on GitHub and part of Mattilsynet's GitHub organisation, so that their platform team knows what it is, what it does and then can scan it for different reasons.

\subsection{Software Development Techniques} \label{sec:software_development_techniques}
During the "Perform project" at the Norwegian Public Service Pension fund, participant \#1 describes \gls{pair_programming} to be used as one of their software development methods. Explaining that not all development teams in the Norwegian public sector perform \gls{pair_programming}, but the ones who do, use it for the development of more complex features. And the development teams try to make sure that all team members have pair-programmed in a development iteration. The benefits of this are described by participant \#1 as: "It gives a common ownership of the code base, which is quite preventive for introducing new bugs, and that developers understand what is happening in the rest of the code".

%During the "Perform project" at the Norwegian Public Service Pension fund, \gls{pair_programming} was used as one of their software development methods. Not all development teams in the Norwegian public sector perform \gls{pair_programming}, but the ones who do, use it for the development of more complex features. The development teams try to make sure that all team members has pair-programmed in an development iteration. This has the effect of increasing ownership in the software that the team is developing, which prevents future faults. It also increases the development team's understanding of how the software function.

Similar software development methods are described by participant \#1 to be used by some teams at NAV, methods such as software teaming or \gls{mob_programming}. Where the whole development team sit together and work on the same feature. Explaining that this enables more discussion about the feature than by performing \gls{pair_programming}. With also has the effect of increasing knowledge in the whole development team.

%A development method used by some teams at NAV is software teaming or \gls{mob_programming}, meaning that the whole development team sit together and work on the same feature. This enables more discussion about the feature than by performing \gls{pair_programming}. This also has the effect of increasing knowledge in the whole development team.

NAV have a technical direction, which describes what software development methods are desired to be followed. This technical direction is described by participant \#5 to recommend \gls{pair_programming} as a preferred software development method, and is recommended to be performed by the development teams as often as possible. The technical direction is also described by participant \#5 to recommend a focus on test-driven development and test coverage in NAV's development teams. 

%NAV also have a technical direction, which describes what software development methods are desired to be followed. This technical direction describes that \gls{pair_programming} is a preferred software development methods, and is recommended to be performed by the development teams as often as possible. The technical direction also recommends focus on test-driven development and focus on test coverage in the development teams.

Skatteetaten is described by participant \#7 to use something similar to NAV, but instead being called a method-framework. Which explains the best practice in software development and what development patterns should be followed at Skatteetaten. Further explaining that part of this method framework is knowledge sharing of these methods: "Then we enter it as part of our methodological work to communicate it out to share experiences". The method framework is also explained by participant \#8 to include development practices such as KISS (Keep it simple stupid), Clean Code, and Separation of Concern.

%Skatteetaten have something similar, but called instead their method-framework which explains the pest practice in software development and what development patterns should be followed. This method-framework is then also used as a way for the development teams to communicate and share experience in development methods. It is important to think of this method-framework as a way to share knowledge between the development teams. For instance development practices which should be followed can be KISS (Keep it simple stupid), Clean Code, and Separation of Concern.

One of the development patterns in Skatteetaten's method-framework described by participant \#8 as creating good quality is doing things in the right order and having a good workflow. So that the competency of the developer is used in a good way. Explaining that as a software developer, there might be tasks that he/she gets stuck on, such as not understanding logic, organisation rules, or architectural decisions. In these scenarios, good quality is being preventative in making the correct decisions.

%One of the development patterns in Skattetatens method-framework is doing things in the right order and a good workflow, that the competency of the developer is used in a good way. As a software developer there might be tasks that he/she get stuck on, such as not understanding logic, organisation rules, or architectural decisions. In these scenarios good quality is being preventative in making the correct decisions.

It is also described by participant \#7 that Skatteetaten has also changed the time of when the software should be quality assured away from the traditional ISO mindset where the quality is controlled after the software is created. Explaining that for the last 15 years, Skatteetaten has instead been controlling the quality from the start of the development project, as a part of their development method. This control mechanism includes new code requiring to be through a test phase for the new code to be deployed to Skatteetatens production servers.

%Skatteetaten has also changed the the time of when the software should be quality assured away from the traditional ISO-mindset where the quality is controlled after the software is created. For the last 15 years Skatteetaten has instead been controlling the quality from the start of the development project, as a part of their development method. This control mechanism including new code requiring to be through a test-phase for the new code to be deployed to the production cluster.

%\textit{Participant \#1} explaining that there is little difference in the methods that the Norwegian public sector and the Norwegian private sector use in the development of their software.

%\textit{Participant \#1} also explaining that at the "Perform project" at the Norwegian pension fund, \gls{pair_programming} was used as a software development method. However not all development teams in the Norwegian public sector perform \gls{pair_programming}, but \gls{pair_programming} is done for the more complex tasks. The development teams do make sure that all team members has pair-programmed in an development iteration. This has the effect of increasing ownership in the code the team is developing, which prevents future faults and the development team has a better understanding of how all the code function.

%\textit{Participant \#1} also explaining that at NAV software teaming or mob-programming is used, meaning that the whole development team sit together and write code. This enables more discussion about the code than by using \gls{pair_programming}. This also has the effect of increasing knowledge in the whole development team.

%\textit{Participant \#2} explaining that software developers are using agile methods to ensure quality in their software by releasing it to the user, and seeing if the user gains any value from the software. These principles are used at NAV and Entur, which focuses MVP's and let the user test the MVP's. If the user do not care about a feature, NAV and Entur do not care either. By repeating this with their users, NAV and Entur ensure high quality in their software.

%\textit{Participant \#5} explaining that through NAV's technical direction, which describes what methods are desired in software development. This describes that \gls{pair_programming} is preferred and it recommends that the development teams to it as often as possible. It also recommends focus on test-driven development and focus on test coverage in the development teams.

%\textit{Participant \#7} explaining that he thinks that the traditional ISO-mindset controls the quality of software after it has been created. For the last 15 years at Skattetaten the mindset has been that the control of software quality should be done from the start, instead of doing it after the fact.

%\textit{Participant \#7} also explaining that Skatteetaten has a method-framework which explains the best practice in software development and describes what working patterns should be followed. This is then added as part of the developments teams methods and is communicated to share experience on it. It is not seen as a method just for the methods sake, but is used as a way to share knowledge.

%\textit{Participant \#7} also explaining that part of Skatteetatens development patterns is doing things in the right order and having a good work flow, that competency is used in a good way. As a software developer there might be things to get stuck on, such as not understanding logic, organisation rules, architectural decisions. In these sceniarios good quality is being preventative in making the right decisions.

%\textit{Participant \#8} explaining that Skatteetaten follows certain software development methods, such as KISS (Keep it simple stupid), Clean Code and Separation of Concern. These are all used to assure good software quality.

%\textit{Participant \#8} also explaining that there are several mechanisms at Skatteetaten to ensure good software quality. Such as code being through a test-phase is required for anything to be deployed to the production cluster. The developers never take any shortcuts. 

\section{Organisational} \label{sec:organisational}
Organisational aspects influencing software quality and its assurance in the Norwegian public sector were described by all participants in the interviews of the case study. Organisational aspects were also the most talked about by the participants, whereas about 50\% of the data was about organisational aspects. 

\subsection{Team} \label{sec:team}
It is suspected by Participant \#2 that the assurance of quality used to be a more formal part of software development, when the software development processes were more defined, like the waterfall model. Explaining that now development teams in the Norwegian public sector themselves choose what should be prioritised and what is important. This is suspected by Participant \#2 to have led to software quality assurance being prioritised lower and seen as less important. And the development teams instead think that if the users are satisfied, then the software is of high quality.

%It is suspected that the assurance of quality used to be a more formal part of software development, when the software development processes was more defined, like the waterfall model. Now the development teams in the Norwegian public sector themselves choose what should be prioritised and what is important. Which is suspected that this has led to software quality assurance being prioritised lower and seen as less important. The development teams instead thinking that if the users are satisfied, then the software is of high quality.

This change from quality assurance being a formal part of software development is described by participant \#5 as being the case at NAV. With NAV changing from centralised requirements and checklists for the quality assurance of their software, to having no centralised management of quality assurance. Explaining that it is now up to each development team at NAV to understand how the software they develop functions. And that NAV's management trust that the teams have the best knowledge of their software, and know how to create software of quality.

%NAV has changed from centralised requirements and check-lists for the quality assurance of their software, to having no centralised management of quality assurance. Now it is up to each development team at NAV to understand how the software they develop function. NAV's management trust that the teams has the best knowledge of their software, and know have to create software of quality.

Entur is described by participant \#2 as basing their development teams on being autonomous and deciding what should be done at what point in the development project. With this autonomy in decisions including which methods should be used for quality assurance in the software being developed.

%Entur base their development teams on being autonomous and decide what would be done at what point in a development project. This autonomy in decisions also including that the team themselves decide which methods are used for assure quality in the software the teams are developing. 

Autonomy is described by participant \#10 as also being incorporated into Mattilsynets 10-11 product teams. However describing the lack of centralised authority is starting to become a challenge for Mattilsynet, and might lead to Mattilsynet implementing a central authority for architecture. Participant \#10 explains that such lack of central authority has been deliberate, due to Mattilsynet's current strategy is to increase development velocity in the product teams before adjusting their direction. And instead iterating the product teams as new problems arise.

%Mattilsynet currently have about 10-11 product teams with great autonomy, with no centralised authority for architecture. A centralised authority for architecture is something which Mattilsynet might implement soon, as it is a growing challenge at Mattilsynet that there is no central authority for architecture. Mattilsynets current strategy is increasing the development velocity of the product teams before giving the product teams direction, and instead iterating the product teams as new problems arise.

The autonomy of the product teams at Mattilsynet is also explained by participant \#10 to extend to the choice of technology in their software. However explaining that when recruiting new developers, skills in a certain set of programming languages are preferred, such as Kotlin/TypeScript. And Mattilsynet recommends that a certain set of programming languages be used: "You won't get any pushback if you want to use it, but if you use another language, for example, then you probably have to be prepared to argue for it". Participant \#10 describes this as deliberate, as it should be easy to follow the "main path" at Mattilsynet, without setting hard limits. As hard limits hinder experimentation by the product teams and slow their development velocity. However explaining that this practice enables the product teams to understand that going beyond the "main path" is going to increase costs, motivating the product teams not taking uncommon decisions without having a good justification.

%The product teams at Mattilsynet have full autonomy in the choice of technology etc. However when recruiting new developers, skills in a certain set of programming languages are preferred, such as Kotlin/TypeScript. The product teams will not get any pushback from choosing a language which is not widely deployed at Mattilsynet, but the team must be prepared to have good arguments for why they have made the particular choice. Mattilsynet has chosen this deliberately, as it should be easy to follow the "main path" at Mattilsynet, but not imposing any hard limitations on what is allowed. This to not hinder experimentation for the product teams, as hindering this slows down their development velocity. Going beyond the "main path" is going to increase costs, so the product teams needs to bear this increased cost, motivating the product teams not taking un-common decisions without having a good justification.

Mattilsynets product teams are described by participant \#10 as being a mix of consultant and permanent employees, no longer having teams purely consisting of consultants. Explaining that this ensures that both consultants and permanent employees understand the costs of creating long-lasting software. As in the past, Mattilsynet has had experiences with consultants not taking into account the costs of maintaining software when creating it. Participant \#10 also explained the product teams to be long-lasting, with no plan for these teams to be dissolved, a view shared by the leadership of the IT department of Mattilsynet. Explaining that this ensures that the teams most familiar with the software have long-lasting ownership of responsibility for that software. Ensuring that any \gls{technical_debt} are decreased.

%Mattilsynets product teams has a mix of consultants and permanent employees, no longer having teams purely consisting of consultants. This ensures that both consultants and permanent employees understands the costs of creating long-lasting software. This due to Mattilsynet having experiences with consultants not taking into account the costs of maintaining software when creating it.

%Mattilsynets product teams are set up to be long-lasting and there is no plan for these teams to be dissolved, which is a view shared by the leadership of the IT-department of Mattilsynet. This ensures that the teams most familiar with the software have a long-lasting ownership of responsibility for that software. Ensuring that any \gls{technical_debt}s are decreased.

It is also explained by participant \#10 that Mattilsynet has recently done a large reorganisation, by changing from project-based development to product teams. The product teams are also organised to have tech leads, where each tech lead has the responsibility for technical quality. This is described as enabling Mattilsynet to solve its strategic problems by making sure that to the highest degree possible, the correct problems are being solved. This helps to solve the issue of having software that is high in quality but does not solve the correct problems. Hence participant \#10 explained that reorganising to product teams and making sure these product teams are looked after, has been an important decision for Mattilsynet. However, also describes that the reorganisation of product teams has led to a loss of routines at Mattilsynet.

%Mattilsynet has recently done a large reorganisation, by changing from project based development to product teams. Which is enabling Mattilsynet to solve their strategic problems by making sure that in the highest degree possible, the correct problems are being solved. This helping solving the issue of having software that is high in quality, but do not solve the correct problems. Thus reorganising to product teams and making sure these product teams are looked after, has been an important decision for Mattilsynet. However it is described that this reorganisation to product teams has led to a loss of routines at Mattilsynet. These product teams was also organised to have tech-leads, where each tech lead having the responsibility for technical quality.

Governance principles are something described by participant \#11 which the product teams at Mattilsynet have to take into account when developing software. Including for instance security, archive law, and GDPR. The product teams at Mattilsynet are described as quite new, with a high degree of newly employed members. Leading to Mattilsynet currently focusing on governance principles required by Norwegian law, being GDPR and security. To help with this, participant \#11 explained that Mattilsynet has a central team for security, which works closely with the platform team to create a good security foundation for the product teams.

%Governance principles are something which the product teams at Mattilsynet have to take into account when developing software. These principles including for instance security, archive-law, and GDPR. Mattilsynets product teams are quite new, with a high degree of newly employed members, so Mattilsynet is currently only focusing on governance principles required by Norwegian law, being GDPR and security. To help with this, Mattilsynet has a central team for security, which works closely with the platform team to create a good security foundation for the product teams.

The teams at Skatteetaten are described by participant \#8 as being cross-functional, with a range of roles being described in a team. Each team has a member responsible for the architecture and it follows the guidelines and patterns which are approved in Skatteetaten. There is also a member responsible for security which makes sure that the has good security in their product. This is also a member who has the responsibility of DevOps who makes sure that the CI/CD pipeline is properly configured. The teams do also have a product owner, which is responsible for choosing what should be prioritised and what should be done at what times. There are also members who are representing the organization and making sure the organisational requirements are met. There is also a member who has an extra focus on testing and testing-related activities. Participant \#8 explained that roles such as security and testing responsible does not have full responsibility for this, they are only supposed to make sure that it happens, the whole development team are responsible for these activities.

%The teams at Skatteetaten are cross-functional. Each team has a member responsible for the architecture and that it follows the guidelines and patterns which are approved in Skatteetaten. There is a also a member responsible for security which makes sure that the has good security in their product. These is a also a member which have the responsibility of DevOps who makes sure that the CI/CD pipeline is properly configured. The teams does also have a product owner, which is responsible for choosing what should be prioritised and what should be done at what times. There are also members who are representing the organisational and making sure the organisational requirements are met. There is also a member who has an extra focus on testing and testing related activities. Roles such as security and testing responsible does not have the full responsibility for this, they are only supposed to make sure that it happens, the whole development team is responsible for these activities. 

%It is also not fully known if the development teams have a specific role which have the responsibility for quality assurance. The teams do have for instance a product owner or UX-designers which have the focus on the user, and that the user should feel value in their software and that it is usable. It might be a challenge that the responsibility of quality assurance is not placed on a single person, but the whole team.

%If the development of a piece of software is large enough to need more than one person, the need for software quality assurance increases. Thus the quality assurance needs to be adapted to the people who create the software, and how the members in the development team cooperate.

%In order to create software of good quality, cross-functional teams is needed. In software development there are a lot of aspect to be weary of, and it is not possible for a single person to handle all these aspects on their own.



%\textit{Participant \#1} explaining that it is the environment who is creating the solution who should have opinions on what is good software quality in the context of the solution they are creating.

%\textit{Participant \#1} also explaining that in the larger project, there has been sub-projects which has focused on software quality. With people with knowledge in testing techniques and testing tools working both with the team and outside the team.

%\textit{Participant \#2} explaining that at Entur is based on that their development teams should be autonomous and decide what should be done at what point in a project. Assuming that this includes the assurance of software quality is part of the team, and the team can therefore decide what methods they should use to assure quality.

%\textit{Participant \#2} also explaining it is the teams themselves that have the responsibility on what methods should be used to assuring quality dependant on the context which they are in.

%\textit{Participant \#2} also explaining that assurance of quality might have been a more formal part of software development, when the software development process was more defined, like in the waterfall model. Now it has changed to development teams choose themselves what should be prioritised and what is important. Further suspecting that this change has led to software quality assurance being prioritised lower and is seen as less important. The development teams instead thinking that if the users are happy, then the software is of high quality.

%\textit{Participant \#2} also explaining that he does not know if any teams have a person which is responsible for quality assurance. However the teams has a for example a product owner or designers which have the user in mind, and that the user should feel that the software is usable and valuable. Further stating that it could be a challenge that the responsibility is not placed on a single person, however there is a responsibility throughout the team.

%\textit{Participant \#5} explaining that NAV has changed from centralised requirements and check-lists to where there is no centralised management. It is up to each team to understand how the software they develop function. Management trusts that the teams has the best knowledge of their product and can create good products.

%\textit{Participant \#6} explaining that if a piece of software is large enough to need more than one person, the need for software quality assurance increases. Thus the quality assurance needs to be adapted to the people who own the software and how the team developing it co-operate. 

%\textit{Participant \#6} also explaining that NAV is making efforts to make it easier for the development teams to change, test and own the software they are developing or maintaining.

%\textit{Participant \#7} also explaining that to achieve good quality, cross-functional teams is needed. In software development there are a lot of aspects to think about, and it is not possible for a single person to handle all these aspects alone.

%\textit{Participant \#8} explaining that at Skatteetaten the teams are cross-functional. Each team has a member responsible for the architecture and that it follows the guidelines and patterns which are approved in Skatteetaten. There is a also a member responsible for security which makes sure that the has good security in their product. These is a also a member which have the responsibility of DevOps who makes sure that the CI/CD pipeline is properly configured. The teams does also have a product owner, which is responsible for choosing what should be prioritised and what should be done at what times. There are also members who are representing the organisational and making sure the organisational requirements are met. There is also a member who has an extra focus on testing and testing related activities. Roles such as security and testing responsible does not have the full responsibility for this, they are only supposed to make sure that it happens, the whole development team is responsible for these activities. 

%\textit{Participant \#10} explaining that Mattilsynet has recently done a big job of moving away from project based development to product teams. This is making Mattilsynet solve their strategic problems by making sure that in the highest degree possible, the correct problems are being solved. This helping solving the issue of having software that is high in quality, but does not solve the correct problems, which are not useful. Thus switching to product teams has been an important decisions and making sure the product teams are well looked after.

%\textit{Participant \#10} also explaining that the teams at Mattilsynet has about 10-11 product teams with great autonomy, with no centralised authority for architecture. This is something which Mattilsynet might implement very shortly, as it is becoming to be a problem that there is no centralised authority for architecture, and the problem is only going to grow. However the strategy for now is simple increasing the velocity of the product teams before giving the product teams the direction, and instead iterating the project-teams as new problems arise.

%\textit{Participant \#10} also explaining that there are no boundaries in the choice of technology etc by the product teams. However when recruiting new developers, a skill in a certain set of programming languages are preferred, such as Kotlin/TypeScript. The product teams will not get any pushback from choosing a language which is not widely deployed at Mattilsynet, but the team must be prepared to have good arguments for why they made the particular choice. This has been a deliberate choice, as it should be easy to follow the "main path" at Mattilsynet, but not making any hard borders on what is allowed, so that the product teams can experiment, as hindering this slows town the development velocity. However going beyond the "main path" is going to increase costs, so the product teams needs to bear this increased costs, this motivating the product teams not to just experiment without having a good justification.

%\textit{Participant \#10} also explaining that the product teams at Mattilsynet has a mix of consultants and permanent employees, no longer having teams purely consisting of consultants. This ensures that both consultants and permanent employees understands the costs of creating long-lasting software. This due to Mattilsynet having experiences with consultants not taking into account the costs of maintaining a software when creating it.

%\textit{Participant \#10} also explaining that the product teams at Mattilsynet are set up to be long lasting and there is no plan for these teams to be dissolved, which is a view shared by the leadership of the IT-department of Mattilsynet. This ensures that the teams most familiar with the software have a long-term ownership and responsibility for that software. Thus ensuring that any \gls{technical_debt}s are decreased.

%\textit{Participant \#11} also explaining that Mattilsynet has discussed a range of governance-principles which each team have to deal with. These principles include things such as security, archive-law and GDPR. However the product teams are quite new, and has a high percentage of newly employed members, so Mattilsynet is currently only focusing on governance-principles which are required by Norwegian public sector agencies by Norwegian law, these being GDPR and security. To help with this, Mattilsynet has a central team for security which works closely with the platform team to create a good security foundation for the product teams. 

\subsection{Knowledge} \label{sec:knowledge}
Participant \#1 explains that the Norwegian public sector participates in conferences to discuss competence development for their agency. Describing it as something which is not only done by the larger agencies: "It's not just Skatteetaten and NAV as the big communities that are present in places where this sort of thing is discussed, so I think it's quite widespread".

%The Norwegian public sector participates in conferences to discuss competence development for their agency. It is not only the larger agencies such as NAV and Skatteetaten who attend such conferences, but other agencies in the Norwegian public sector.

Sharing of knowledge is described by participant \#11 as important for assuring quality in software, as well as high competence throughout an organisation is important for software quality. Participant \#11 explains that Mattilsynet is trying to achieve this by having a tech lead forum for all tech leads at Mattilsynet to increase knowledge sharing between all the tech leads.

%Sharing of knowledge is important for assuring quality in software, as well as high competence throughout a organisation is important for software quality. Mattilsynet is trying to achieve this by having a tech lead forum for all tech leads at Mattilsynet to increase knowledge sharing between all the tech leads.

Lack of competency is described by participant \#6 as a challenge for NAV, as it is quite clear that the people with the least experience in software development have little knowledge in testing or continuous delivery, so these have to be trained in such things. Lack of competency in the people maintaining Skatteetaten's older IT systems is described as a challenge by participant \#7, as these people lack the competency to develop and maintain Skatteetatens newer software. These people are described by participant \#7 as working with Oracle databases and do not know how to create a user story or a Java component. Explaining the need for different competencies at Skatteetaten, due to the large changes in their modernisation plan.


%Competency can be a challenge, as it is quite clear the the people with the least experience in software development has little knowledge in testing or continuous delivery, so these have to be trained in such things.

%The people maintaining Skatteetatens older IT-systems do not have the competency required to develop and maintaining their newer software. These people are described as working with Oracle databases, and do not know how to create a user-story or create a Java component. Skatteetaten need different competency as described, as there is a large change in their modernisation plans.

For the health platform developed for the counties in mid-Norway, it is described by participant \#1 that it was known beforehand that it would not be user-friendly. As well as activities to improve usability were not performed, technical faults in the platform, and a testing rig was placed too far away from the real environment in which the platform would operate. Participant \#1 thinks that this could be due to a lack of IT competence in the planning of the project.

%For the health platform, it was known beforehand that it would not be user friendly, and activities to improve usability were not performed. Including technical faults in the platform and a testing rig which were too far away from the real environment which the platform would operate in. This could be due to a lack of IT-competence in the planning of the project.

%\textit{Participant \#1} explaining that the Norwegian public sector participates in conferences to discuss competence development for their agencies. It is not only the larger agencies such as NAV and Skattetaten who attend such conferences, but other agencies in the Norwegian public sector.

%\textit{Participant \#4} explaining that the sharing of knowledge is important for assuring quality in software, as well as high competence throughout a organisation is important for software quality.  

%\textit{Participant \#6} explaining that competency can be a challenge. It is quite clear that the people with the least experience in software development has little knowledge in testing or continuous delivery, so these have to be trained in such things.

%\textit{Participant \#7} explaining that the people maintaining Skatteetatens older IT-system does not have the competency needed to develop and maintaining their newer software. These people usually work with stuff like Oracle databases do not know how to create a user-story or create a Java component. Skatteetaten need a different competency than that as there is a large change in their modernisation plans.

%\textit{Participant \#11} Mattilsynet also have a tech lead forum for all tech leads at Mattilsynet to increase knowledge sharing between all the tech leads.

\subsection{Development Projects} \label{sec:development_projects}
The majority of the data from the interviews about organisational aspects described organisational aspects relating to development projects as affecting software quality and its assurance. These aspects are (I) the Norwegian digitisation agency's project wizard, (II) Financing, (III) Contracting and Tenders, (IV) Budgeting, (V) Off-the-shelf Software, (VI) Ownership, and (VII) Strategy.

\subsubsection{Project Wizard} \label{sec:project_wizard}
Described by participant \#1 as of great importance to the Norwegian public sector, is the Norwegian digitisation agencies project wizard. However describing the project wizard as not suited for agile projects, rather projects following methods such as the waterfall method. Participant \#1 explains that this means that there are more projects following the waterfall method in the Norwegian public sector than in the Norwegian private sector.

%Being of great importance for the Norwegian public sector, is the Norwegian digitisation agencies project wizard. The project wizard is not suited for agile projects, rather projects following methods such as the waterfall method. This means that there are more projects following the waterfall method in the Norwegian public sector than in the Norwegian private sector. 

The project wizard is described as a challenge for NAV and the rest of the Norwegian public sector by participant \#3. As the project wizard is necessary for NAV to obtain the correct funding for their larger software proposals, however describing its quality assurance regime as more suited for building roads than software: "It is based on having a fairly perfect knowledge of what you are going to make before you start making it. We've never had that, and I don't think we ever will either". Participant \#3 explains further when proposing software which will replace 40-year-old software, which has been altered to fit new laws and user needs through 40 years. And actually start creating the replacement, you uncover new things which could never be planned ahead of time.

%This project wizard is a challenge for NAV and the rest of the Norwegian public sector. The project wizard is necessary for NAV to obtain the correct funding for their larger software proposals, has a quality assurance regime more suited for building roads than software. The project wizard assumes perfect understanding of the software proposed before it is created, something which NAV has never had, and never will. When proposing software which will replace 40 year old software, which has been altered to fit new laws and user needs through 40 years. And actually start creating the replacement, you uncover new things which could never be planned ahead of time.

The challenges with new knowledge and the project wizard were also explained by participant \#11, describing the project wizard as hostile towards project scope creep. When in reality scope creep might mean new insights have been discovered about the problem being solved by the software. Thus having the possibility to give more value to the end user. Participant \#11 also explains that scope creep can also mean that the original scope of the software has changed, so not changing the scope can mean solving the wrong problem.

%Project scope creep is something which might be describe as something to avoid. However scope creep might mean that new insights have been discovered about the problem being solved by the software. Thus having the possibility to give more value to the end user. Scope creep can also mean that the original scope of the software has changed, so not changing the scope can mean solving the wrong problem. This is the issue with the project wizard, which do not take changing scope into account and the dynamic approach which can be achieved by product teams.

The project wizard is also described as a resting pillow for not reorganising to product teams by participant \#10. Explained as being the case with the old IT leadership at Mattilsynet, with Mattilsynet being focused on becoming better at projects suited for the project wizard. And participant \#10 explained the real need was to stop with such projects: "We needed to end the project. And then it is very stupid to spend a lot of resources on getting better at projects. And then actively recruited a lot of people who will improve Mattilsynet on projects. And then you create slowness in the organisation". Further explaining that it then became a challenge for the staff specialising in projects, that Mattilsynet changed from project wizard projects to product teams.

%The project wizard is also described as a resting pillow for not reorganising to product teams. Which were the case with the old IT-leadership at Mattilsynet, as Mattilsynet was focused on becoming better at projects suited for the project wizard. The real need for Mattilsynet was to stop with such projects, it was therefore unfortunate that Mattilsynet was spending resources on becoming better at project wizard projects. Including actively recruiting staff that specialises in project wizard projects, creating slowness in the organisation. It would then become a challenge for the staff specialising in projects, that Mattilsynet changed from project wizard projects to product teams.

Both opportunities and problems are described by participant \#7 to be given by the project wizard. Explaining that NAV had a financial model which was heavily dependent on funds from the project wizard, and buying software from external vendors. And participant \#7 describes this as not being the case for Skatteetaten, as Skatteetaten has always developed and operated its own software. Explaining that this means that NAV has had an opportunity to save money as a consultant might cost 2.5-3 Million NOK, while a permanent employee might cost 1 Million NOK. Meaning that NAV has had a good opportunity to build up a sizeable development environment. This opportunity is explained by participant \#7 as which Skatteetaten has never had, relying on the funds from the project wizard for their large initiatives. As well as being able to deliver as required by the Norwegian government. The IT department of Skatteetaten has a total of 1000 personnel, in addition to 300 consultants. The 300 consultants would never be possible without the funding from the project wizard.

%The project wizard gives opportunities, but also give problems. NAV had a financial model which were heavily dependant on funds from the project wizard, and buying software from external vendors. This is not the case for Skatteetaten. Skatteetaten has always developed and operated their own software. This means that NAV has had an opportunity to save money as a consultant might cost 2.5-3 Million NOK, while a permanent employee might cost 1 Million NOK. Meaning that NAV has had a good opportunity to build up a sizeable development environment. This is a opportunity which Skatteetaten has never had, relying on the funds from the project wizard for their larger initiatives. As well as being able to deliver as required by the Norwegian government. The IT department of Skatteetaten has a total of 1000 personnel, in addition to 300 consultants. The 300 consultants would never be possible without the funding from the project wizard.

Another challenge with the project wizard explained by participant \#3, is that its funding stops when the software its funding is put to use. Describing that NAV wishes to create software that is long-lasting, which requires steady funding for development and maintenance long after the software has been put to use. As there will always be changes in the laws surrounding NAV's software and other needs. And participant \#3 explained that the stop in funding is hurting the Norwegian public sector: "It is one of the major headaches for the entire public sector, but perhaps now we at NAV are the most aware of it since we have come the furthest in this digitisation".

%Another challenge with the project wizard is that its funding stops when the software its funding is put to use. NAV wish to create software that is long-lasting, which requires steady funding for development and maintenance long after the software has been put to use. This because there will always be changes in the laws surrounding NAV's software and other needs. The stop in funding is hurting NAV the most of all the public sector agencies in Norway, as NAV has gotten the furthest with digital transformation. And is something which will increase to hurt NAV, as NAV get better at delivering digital solutions to the citizens of Norway.

The project wizard is described by participant \#11 as not being a challenge for Mattilsynet. As Mattilsynet stopped its use after Mattilsynets recently re-organising from projects to product teams.

\subsubsection{Financing} \label{sec:financing}
However, the challenges relating to financing from the project wizard in \autoref{sec:project_wizard} is described as an issue of prioritisation, rather than financing by participant \#5. And it usually being a discussion about financing, when in reality it is a prioritisation issue, independent of financing. Explaining that financing might not always mean low funding, but the expectation of what should be delivered compared to the given budget is too high. And over-promising can create issues which are expensive to fix and clean up.

%However this has been described this as an issue of prioritisation, rather than financing. It being a prioritisation between improving technical quality and new functionally. It usually being a discussion about financing, when in reality it is an prioritisation issue, independent of financing. Financing might not always mean low funding, but the expectation of what should be delivered compared to the given budget being to high. And over-promising can create issues which are expensive to fix and cleanup. By not prioritising correctly and the wish to create as much as possible, as quickly as possible, and not prioritising to clean up can be an issue.

Firmer boundaries and more trust in the financing of NAV's software are described as desired by participant \#6. As this would enable NAV to more flexibly prioritise their funds. Yet also explaining that firmer boundaries and more trust in financing might not be correct for the whole Norwegian public sector, as it might lead to all agencies asking for too much funding. 

%Firmer boundaries and more trust in the financing of NAV's software would be good. Thus being able to more flexibly prioritise their funds. It is difficult to calculate money saved by using the project wizard, and the rewards gained becomes an illusion. However firmer boundaries and more trust might not be correct for the whole Norwegian public sector, as it might lead to all agencies asking for to much funding.

A more predictable financing model is described by participant \#8 as being better for Skatteetaten. Describing that a potential risk in a project with an unpredictable financing model is that the financing model does not take into account unforeseen events. This is described by participant \#8 as being able to have negative consequences for the quality of what is being developed.

%A more predictable financing model would be better for Skateetaten. If all the money for a project is used, which is a risk with an un-predictable financing model, the financing model do not take into account unforeseen event. This can have have negative consequences on the quality of what being developed.

\subsubsection{Contracting and Tenders} \label{sec:contracting_and_tenders}
The health platform developed for the counties in mid-Norway is described by participant \#1 as mainly created using the waterfall method, switching to an agile method later in the project. However explaining that most software professionals would agree not to follow waterfall methods, as most of the time it leads to the software quality suffering. Participant \#1 explains that in such projects, it is common to define a set of requirements, send them to multiple contracts, and contractors fight to deliver the cheapest system. Describing that in such processes, the software quality usually suffers: "This will often mean that you have to take certain shortcuts. And if everything going into the solution is by contract. Then the shortcuts will usually be in the quality of the software". Further explaining the issue with sending software systems on tender, as it also distances these requirements from the user of the software.

%The health platform developed for the counties in mid-Norway, was mainly created using the waterfall method, switching to an agile method later in the project. However most software professionals would agree to no follow waterfall methods, as most of the times it leads to the software quality suffering. In such projects it is common to define a set of requirements, send them to multiple contractors, and the contractors fight to deliver the cheapest system. It is then usually necessary to take some shortcuts, and if the requirements are written in a contract, it will be on the software quality. This problem only increases if the receiver of the software do not know how to evaluate software quality. This is the issue with sending software systems on tender, it distances the requirements from the user of the software.

Before 2016, participant \#3 explained how NAV had little knowledge or resources to develop software themselves. And all systems required for NAV's operations had to be put on tenders. Explaining that NAV has stopped this practice, as it was not sustainable and had negative consequences. Participant \#3 explained how this was the case when NAV received delivery of their original system for delivering pensions, where NAV did not know how to maintain the system themselves. NAV then has to put the maintenance and further development of their core business on tender.

%Before 2016 NAV had little knowledge or resources to develop software themselves. All system required for NAV's operations had to be put on tenders. Now NAV has stopped with this, as it is not sustainable and has negative consequences. For instance when NAV received delivery of their original system for delivering pension, NAV did not know how to maintain the system themselves. NAV then had to put the maintenance and further development of their core business on tender, which is bad for a public sector agency.

This way of putting software of tender is described by participant \#7 as never being done at Skatteetaten. Explaining that since 2003 the operational and IT at Skatteetaten have worked together in product teams for IT-related projects. This is because when there is a drastic change in operations at Skatteetaten, it cannot simply be given to a single team. Described by participant \#7 as requiring a change in the organisation on how things are solved and how people cooperate both inside and outside the agency so that the software is created in a new way.

%Since 2003 the operational and IT at Skatteetaten has worked together in product teams for IT-related projects. Skatteetaten has never put software on tender, which is common in the Norwegian public sector. It has always been the case that the operational and IT work together on creating software. This because when there is a drastic change in operations at Skatteetaten, it cannot simply be given to a single team. It requires a change in organisation on how things are solved and how people cooperate both inside and outside the agency, so that the software is created in a new way.

%It is described that Mattilsynet will have to be better at assurance quality in the suppliers they use to develop their own software. Inducing routines and questions asked for possible suppliers, which might need to be deeper ingrained into Mattilsynets procurement process.

One of Mattilsynets's older systems, called "Mats", is a system explained by participant \#10 to be tailor-made for Mattilsynet by a contractor. Explaining that this system had a problem of Mattilsynet wanting functionality, without the contractor being motivated to refuse. Leading to the system's complexity growing organically. Participant \#10 explained that Mattilsynet has not understood the consequences of their orders: "Functionality has been ordered and technical debt has been ignored. And just thought that somebody else will fix the problems, which they do not".

%One of Mattilsynets older systems, called "Mats", is a system which were tailored for Mattilsynet by a contractor. This system has had a problem of Mattilsynet wanting functionality, without the contractor being motivated to say no. The system then grows complexity organically. The customer in this matter being Mattilsynet, has not understood over a longer period of time the consequences of their orders. Mattilsynet has ordered functionality while ignoring \gls{technical_debt}, and believing someone else will fix it, however nobody does.

A similar previous problem for Mattilsynets described by participant \#10, is Mattilsynet having quite weak requirement specifications for their contractors. As there were not enough internal resources at Mattilsynet to create proper requirement specifications for their contractors. Resulting in the contractors themselves almost creating the whole requirements specification for Mattilsynet. Participant \#10 explained that this was the case with Mattilsynets photo app, where Mattilsynets inspectors can take pictures and store them in Mattilsynets databases, without having to use iCloud. And since the requirement specification was weak, the team making it did not think that someone had to maintain it long after it was developed. Making the decisions for technology and the solution quickly, and \gls{technical_debt} being created from the start.

%A previous problem for Mattilsynet is having quite weak requirement specifications for their contractor. As there was not enough internal resources at Mattilsynet to create proper requirement specifications for the contractors. Resulting in the contractors themselves almost creating the whole requirements specification for Mattilsynet. This was the case with Mattilsynets photo app, where Mattilsynets inspectors can take pictures and store them in Mattilsynets databases, without having to use iCloud. However since the requirement specification was weak, the team making it did not think that someone had to maintain it long after it was developed. Thus making the decision making for technology and solution easier than it might should be, and \gls{technical_debt} being created from the start.

The assurance of software quality is described by participant \#10 as quite variable in the Norwegian public sector. Describing areas such as healthcare, municipalities and national defence to be heavily invested in putting software on tender and assigning contractors. While others such as NAV, Skatteetaten, NRK and Norsk Tipping are described by participant \#10 as creating software themselves. The difference is that the consultants are hired as competency to an already existing development team.

%The assurance of software quality is described as quite variable in the rest of the Norwegian public sector. Areas such as healthcare, municipalities and national defence are still heavily invested in putting software on tender and assigning contractors. Others such as NAV, Skatteetaten, NRK and Norsk Tipping create the software themselves. The difference being is that the consultants are hired as competency to an already existing development team.

\subsubsection{Budgeting} \label{sec:budgeting}
Yearly budgeting in the Norwegian public sector is described as a challenge by participant \#2. Explaining that in the private sector, if you can prove that a development project will be beneficial, it is easier to gain and spend money for that development project. While in the Norwegian public sector budgets are planned yearly. so it is not easy to change how the budget should be spent on different development projects. Participant \#2 explains that if a certain software in the public sector experiences any sudden problems which need larger funding, it will be difficult to get the necessary resources. Describing this as an issue for software quality: "If you find out something, like a quality issue, it's not like you can just magically get more money".

%Yearly budgeting in the Norwegian public sector is a challenge. In the private sector, if you can prove that a development project will be beneficial, it is easier to gain and spend money for that develop project. However in the Norwegian public sector budgets are planned yearly, so it is not easy to change how the budget should be spent on different development projects. If a certain software in the public sector experiences any sudden problems which needs larger funding, it will be difficult to get the necessary resources. This problem is larger for software quality, as this is not a good enough reason for extraordinary funding from the yearly budget. 

Participant \#2 explains that the challenge of yearly budgeting in the Norwegian public sector also affects the creation of new software. This as the budget is being funded by the Norwegian tax-payers, leading to the development projects in the Norwegian public sector usually not having enough funding. This situation is described by participant \#2 as usually resulting in the software quality being prioritised lower than required functionality, due to there not being enough funding for both.

%The issue of yearly budgeting in the Norwegian public sector also affects the creation of new software. This because of the budget being funded by the Norwegian tax payers, hence a development project in the Norwegian public sector not having enough funding. This usually results in the software quality being prioritised lower than required functionality, as there is not enough funding for both.

The IT budget of NAV is described by Participant \#4 as not set correctly, as the budget should instead be placed under the problem, not the technology. Explaining that for NAV, this means increasing the budget for a specific benefit, so that that benefit can create the software that it needs. Participant \#4 explained that this way, the management of NAV can be more flexible with their budget on what should be created for the benefit that needs it the most. However not the budget is unreachable for the management of NAV and therefore not prioritised in a way that is beneficial for the users of NAV.

%The IT budget of NAV is described as not set correctly. The budget should instead be placed under the problem, not the technology. For NAV, this means increasing the budget for a benefit, so that the benefit can create software that it needs, rather than all budgeting being placed under the IT budget. This way the management of NAV can be more flexible with their budgeting on what software should be created for the benefit that needs it the most. Now the budget is unreachable for the management of NAV and therefore not prioritised in a way that is beneficial for the users of NAV.

\subsubsection{Off-the-shelf Software} \label{sec:off_the_shelf_software}
Adjusting off-the-shelf software as a way of saving money is described as a challenge by participant \#2 in the Norwegian public sector. As trying to adjust off-the-shelf software might lead to lower quality, or having to spend a lot of resources to adjust it to the point of its users being satisfied. Participant \#2 explained that therefore might be a divide between the agencies in the Norwegian public sector which have their own in-house development environment, and the agencies that do not. This is due to the agencies with in-house developers who can create software themselves, having full control of the software quality. While the agencies without will need to adjust off-the-shelf software.

%Adjusting off the shelf software as a way of saving money is a challenge in the Norwegian public sector. Trying to adjust off the shelf software might lead to lower quality, or having to spend a lot of resources to adjust it to the point of its users being satisfied. Therefore there might be a divide between the parts of the Norwegian public sector which have their own development environment. This due to agencies who have in-house developers who create software themselves, have full control of the software quality. Then those who do not have the opportunity for their own development environment will need to adjust off the shelf software.

It is also explained by participant \#2 that agencies with an in-house development environment are likely to know their agency quite well. Making the job of adjusting off-the-shelf software easier. However, when using off-the-shelf software, the rules of the software are already set which have to be taken into account. Participant \#2 explains that when creating software from scratch, these rules do not apply anymore, letting the software be highly specified for the user's needs.

%In-house development environment in an agency are likely to know their agency and their agency quite well. This making the job of adjust off the self software easier. However when using off the shelf software, the rules of the software is already set which have to be taken into account. When creating software from scratch, these rules do not apply anymore, letting the software being highly specified for the users needs.

\subsubsection{Ownership} \label{sec:ownership}
NAV's success in digital transformation is described by participant \#6 because of their focus on ownership. Including stopping NAV's software single-handedly created by consultants: "It is a problem with consultants who do not have the long perspective. That they think that consultants and projects are connected at NAV, and when the project has been created it is done. To me, that is the opposite of quality". Participant \#6 explains that as time goes on and a piece of software at NAV is not maintained, it loses quality, so a constant fight against quality withering is necessary.

%NAV's success in digital transformation is described because of their focus on ownership. Including stopping with NAV's software single-handed created by consultant who do not understand the long-term perspective when creating NAV's software. Not thinking that a piece of software can be created and the project is complete. This being the opposite of quality. As time goes on and a piece of software at NAV is not maintained, it looses quality, so a constant fight against quality withering is necessary.

Participant \#3 explained how organisations want to deliver software to their users as NAV has achieved. Explaining that it sounds quite attractive, and is spoken about at conferences, yet it requires high effort from the organisation. And it mainly requires not having software delivered single-handed from consultants, instead software developed by in-sourcing at the organisation. Participant \#3 explaining as soon as the software is created outside the organisation, the quality can decrease. Describing to do as NAV requires the organisation to take ownership of their software, however, not all agencies in the Norwegian public sector have this opportunity.

%Organisations wants do deliver software to its users like NAV has achieved. It sounds quite attractive, and is spoken about at conferences, however requiring high effort from the organisation. It mainly requires to not having software delivered single-handed from consultants, but software developed by in-sourcing at the organisation. As soon as software is created outside the organisation, the quality can decrease. To do as NAV it requires the organisation to take ownership in their software, however not all agencies in the Norwegian public sector has this opportunity.

%Different contract models can be used in the Norwegian public sector. Where competency is hired into and already existing development project, something which agencies such as NAV and Skatteetaten is doing. Here the agency is the owner and responsible part of the projct, but are hiring resources needed to complete the project. This is something that require a higher degree of maturity in the agency.

\subsubsection{Strategy} \label{sec:strategy}
Other agencies in the Norwegian public sector are described by participant \#4 as following the methods NAV are using to improve their digital transformation methods. The methods include rigging the agency around DevOps, product teams, and small specialised applications. Participant \#4 describes the Norwegian public sector as improving in these fields, while still having some way to go, to be as good as the private sector. And participant \#4 emphasises that the agencies in the Norwegian public sector do not follow NAV because NAV is successful, but because the methods NAV are following being clever.

%Other agencies are following the methods NAV are using to improve their digital transformation efforts. These methods include rigging the agency around DevOps, product teams and small specialised applications. Even though the Norwegian public sector is improving, they still have some way to go to be as good as the private sector. The agencies do not follow NAV just because NAV is successful, but because the methods NAV are following are smart.

%Software quality assurance at NAV used to be centralised. When getting delivery of a new system, there would be quality assurance of the delivered system at NAV, however the quality assurance being disconnected from the operations of the system. Large amount of resources was used to control quality, but faults were not hindered. Today the quality assurance at NAV is directly connected to the development of their systems.

The researcher Torgeir Dinsgrøyr from NTNU was mentioned by participant \#7 to talk about how NAV is coordinating teams and steering of their development direction. Participant \#7 mentioned that the mission of Skatteetaten will always change: "Our politicians are constantly inventing new things, or updating existing things for us". Thus creating the need for coordination between product areas, multiple teams and multiple parts of the business operations at Skateetaten.

%The researcher Torgeir Dinsrøyr from NTNU was mention to talk about how NAV is coordinating teams and steering of their development direction. The politicians in Norway will always give Skatteetaten new things to do, or change existing things. Thus creating the need for coordination between product areas, multiple teams and multiple parts of the business operations at Skateetaten.

It is the perception of participant \#7, that many might think of product teams as something which is quite stable. However explaining that this is something which can be unwanted from an organisational perspective, instead wanting product teams to be more flexible. Explaining when the organisation tries to change the staffing in a product team, a high degree of resistance can be encountered. This is due to the staff enjoying the team or area to which they belong, with the people they already work with. Therefore organising around product teams can create stiffness in an organisation.

%Many might think of product teams as something which is quite stable. This is something what can be unwanted from an organisational perspective, wanting product teams to be more flexible. When the organisation tries to change the staffing in an product-team, a high degree of resistance can be encountered. This due to the staff enjoying the team or area to which they belong, with the people they already work with. Therefore organising around product teams can create stiffness in an organisation.

%Skatteetaten has a different strategy and a higher architecture than NAV. Skatteetaten has an increased focus on common components or the re-use of components. When having a high degree of reuse, it enables Skattetaten to have a high velocity in their development of software.

Well-managed projects are described by participant \#10 as not without their issues. Such as the project at Mattilsynet for meat controlling, which was described as in general well managed and delivered with an \gls{service_oriented_architecture}. It was so well managed that it became the template for projects at Mattilsynet. However, participant \#10 explained that it had no real strategy on how it should be maintained and how further operations should be conducted after it was delivered. Even if the project was well done, it still left a large problem for Mattilsynet. Now explaining that the same system has been split up between two product teams, something which was described as not without problems by participant \#10: "Now both of those product teams are sitting with components that they are not in a position to understand. With technology that they do not know ... And any functional changes in these components do not occur".

%It is better to be good at projects than bad at projects. A project at Mattilsynet which was delivered with \gls{service_oriented_architecture} and in general was well managed, being the project for Meat controlling. It was so well managed it became the template for projects at Mattilsynet. However it had no real strategy on how it should be maintained and how further operations should be conducted after it was delivered. Even if the project was well done, it still left a large problem for Mattilsynet. The same system has now been split up between two product teams, something which has not been without problems. Both product teams now have to develop components which they do not understand, with technology they do not know. Updating functionality is such systems are difficult, due to the high degree of \gls{technical_debt}.

%\textit{Participant \#1} explaining that a thing which is important for the Norwegian public sector is the Norwegian digitisation agencies project wizard, based upon PRINCE2. This project wizard is not suited for agile project, rather project following methods relating to the waterfall method. This means that you will see the methods such as the waterfall method which require more planning and defining of a project in the Norwegian public sector rather than in the Norwegian private sector.

%\textit{Participant \#1} also explaining that the health platform developed for the counties in mid-Norway, was mainly created using the waterfall method, switching to an agile method later in the project. However most professionals would agree to not follow such waterfall methods, as it most of the times leads to the software quality suffering. In such projects it common to define a set of requirements, send them to multiple contractors, the contractors fight to deliver the cheapest system. Then its necessary to take some shortcuts, and if the requirements are written in a contract, it will be on the software quality. And this problem gets even larger the receiver of the project is not good at evaluation software quality. That is the problem with sending software systems on tender, it distances the requirements from the user of the system.

%\textit{Participant \#1} also explaining that there are different contract models which can be used in the Norwegian public sector. Where competency is hired into an already existing development project which agencies such as NAV and Skatteetaten is doing. Here the agency are the owner and responsible part of the project, but are hiring resources need to complete the project. However this requires a higher degree of maturity in the agencies than before.

%\textit{Participant \#1} also explaining that if using the health platform as a case, it was known beforehand that it wouldn't be very user friendly. And special activities to get the usability better were not performed, as well as being technical faults in the platform. Including a testing rig which were to far away from the real environment which the platform should operate in. This could be due to a lack of IT-competence in the planning of the project.

%\textit{Participant \#2} explaining that the yearly budgeting in the Norwegian public sector is an issue. In the private sector if you can prove that a development project will earn money, then its easier to spend money for that development project. However in the Norwegian public sector budgets are planned yearly, so it is not easy to change how its budged should be spent to different development projects underway. If a certain software in the public sector experiences any sudden problems which needs larger funding, it will be difficult to get the necessary resources. This problem is larger for software quality, as this is not a good enough reason for changing the funding underway in a budget year or retrieving extra funding.

%\textit{Participant \#2} also explaining that the issue of yearly budgets also affects the creation of new software in the Norwegian public sector. As the money is funded by the Norwegian tax payers, a development project usually do not have enough funding, hence the software quality is where it is usually saved on to fulfil the requirements of the new software.

%\textit{Participant \#2} also explaining that a challenge in the Norwegian public sector is to adjust off the shelf software as a way to save money. Instead of doing it like NAV which has in-house developers who create the software themselves, and have full control of the software quality. By trying to adjust off the self software might lead to lower quality or spend a lot of resources of adjust it to a point where the users are satisfied. So there might be a divide between the parts of the Norwegian public sector which have their own development environment, so they can create high quality software themselves, and those who cannot and have to adjust off the shelf software.  

%\textit{Participant \#2} also explaining that his impression that if an agency has an in-house development environment, it know that agency and their challenges quite well, so the job of adjusting off the shelf software might be easier. However when using off the shelf software, the rules of the software are already set which has to be taken into account. When building something from scratch, these rules do not apply anymore, which means that the software can be highly specified for the users needs.

%\textit{Participant \#3} explaining that a lot of organisations wants to deliver software to its users like NAV has achieved. It sounds quite attractive and is spoken about in conferences, however it requires quite a lot from the organisation. Mainly it requires to not have software delivered solely from a consultants, but software developed by in-sourcing at the organisation. As soon as software is created outside the organisation and is delivered, the quality can decrease. To do as NAV it requires the organisation to take ownership in their software, however not all agencies in the Norwegian public sector have the opportunity for this.

%\textit{Participant \#3} also explaining that before 2016 NAV had little knowledge or resources do develop software themselves. All systems required for NAV's operations had to be put on tenders. However NAV has stopped with this, as it is not sustainable and has negative consequences. In particular when NAV got their original system for delivering pension benefits, NAV did not know how to maintain it themselves. This meaning that NAV had to contract the maintenance and further develop the systems which are responsible for their core business, which is bad for a public sector agency.

%\textit{Participant \#3} also explaining that the Norwegian digitisation agency's project wizard is a challenge for NAV and the rest of the Norwegian public sector. The project wizard which is necessary for NAV to follow to get the correct funding for their larger software proposals, has a quality assurance regime that is more suited for building roads, rather than software. The project wizard assumes a perfect understanding of the software proposed before it is created, which NAV has never had, and never will. When proposing something which will replace 40 year old software, which as been altered to fit new laws and needs through 40 years. And actually start creating the replacement, you uncover new things which could never be planned ahead of time. The fact that the funding from the project wizard stops after the software put to use is an issue. NAV wish to create software wish will live for a long time, and this requires a steady funding for development, long after the software has been put to use. This because there will always be changes in the laws surrounding NAV's software and other needs for the software, and the project wizard is not suited for such changes. This is something which is hurting NAV the most, as it has gotten the furthest in its digitisation process of the Norwegian public sector agencies. And will increase to hurt NAV as becomes better at delivering digital solutions to the citizens of Norway.

%\textit{Participant \#4} explaining that before the software quality assurance was centralised at NAV. When getting delivery of a new system there would be quality assurance of the delivered software. The quality assurance was disconnected from the operations of the system. A lot of resources was used to control quality, but faults were not hindered. Today at NAV, the quality assurance is connected to the development of the software.

%\textit{Participant \#4} also explaining that there are agencies in the Norwegian public sector who are following NAV's methods for improving their digitisation transformation efforts. This by rigging themselves around DevOps, product teams and small specialised applications. Even though the the Norwegian public sector is improving, they still have some way to go to be as good as the private sector. It is worth mentioning that the agencies are not follow NAV just because NAV is successful, but because the methods NAV are using are smart trends.

%\textit{Participant \#4} also explaining that there are challenges in how the development of software at NAV is financed and has a to strong dependence on project funds from the Norwegian digitisation agency project wizard. The assumption that you can plan, build and be done with the software is a faulty assumption. NAV wants to build small applications which can be iterated, and the developers can learn as they go, since you learn more about application this way than in the planning phase. The funding from the project wizard does not fit this mindset, there should instead be continuous funding.

%\textit{Participant \#4} also explaining that the IT budget of NAV is set wrong. The funding should instead be put to the problem, not the technology. For NAV's case, this means increased funding for a benefit, so that the benefit can create software that it needs, rather than all the funding going under the same "IT-fund". This way the management of NAV can be more flexible with their funds on what software should be created for the benefit that needs it the most. However now it is all "unreachable" for the management and not prioritised in a way that is beneficial for NAV's users.

%\textit{Participant \#5} explaining that the issue of financing might rather be an issue of prioritisation. It being a prioritisation between improving technical quality and new functionality. Thinking that its usually a discussion about financing, when in reality its an prioritisation issue, that there is to much to do, independent of the financing. Financing might not always mean that to little money, but the expectations of what should be delivered compared to the given budget being to much. And over-promising can create issues which are expensive to fix, the cleanup being expensive. So by not prioritising correctly and the wish to create as much as possible as quickly as possible and not prioritising to clean up can be an issue.

%\textit{Participant \#6} explaining that NAV has gotten ahead on their digital transformation because of their focus on ownership. This being the solely use of consultants who does not have the long perspective when creating NAV's software. That the consultants think that they should create a piece of software for NAV, and then be finished. This being the opposite of good quality. As time goes on and a piece of software at NAV is not maintained, it looses quality, so you constantly have to fight against quality weathering.

%\textit{Participant \#6} also explaining that NAV has a financial model which push then towards projects with a set deadline and a reduction of resources after the deadline has been reached. Both of which is bad for quality.

%\textit{Participant \#6} also explaining that firmer boundaries and more trust in the financing of NAVs software would be good. Thus being able to more flexibly prioritise. It is hard to calculate money saved by using the project wizard, the rewards gained becomes an illusion. However firmer boundaries and more trust might not be smart for the Norwegian public sector, as it might lead to all agencies asking for to much funding.

%\textit{Participant \#6} also explaining that the Norwegian digitisation agency's project wizard is a challenge of software quality. However it is the only way to get funding for the larger software initiatives at NAV. If not for the funding from the project wizard, NAV might had to simplify their services, leading to a reduction in what NAV can deliver for the citizens of Norway.

%\textit{Participant \#7} explaining that the Norwegian digitisation agency's project wizard gives opportunities, but it also gives problems. Before NAV had a financial model which were heavily dependant on funds from the project wizard, and buying software from external vendors. This has Skatteetaten never done, Skatteetaten has always developed and managed their own software. This means that NAV has had an opportunity to save money as consultant might cost 2.5-3 million NOK, while a full time employee might cost 1 million NOK. That means that NAV has had good opportunity to build up a sizeable development environment of over 800 employees (\textcolor{red}{sleng in detsombetyrnoe.no sitering}). That opportunity Skatteetaten has never had, we rely upon the funding from the project wizard for for their larger initiatives and being able to delivered as required by the Norwegian government. The IT department of Skatteetaten has a total of 1000 personnel, where about 300 is consultants, these 300 consultants would never be possible without the funding from the project wizard.

%\textit{Participant \#7} also explaining that when working with funds from the project wizard, only the most necessary and what is legally required should be implemented. This means that it might not be possible to prioritise things such as user needs, which is an disadvantage.

%\textit{Participant \#7} also explaining that since 2003 the operational and IT has worked together in product teams for IT-related project. However for other projects in the Norwegian public sector, it is common to put something out on tender, a contractor builds it, and delivers it. Skatteetaten has never done this, it has always been the case that operational and IT has worked together on creating software. This because when there is a drastic change in operations at Skatteetaten, it cannot just be given to a single team. It requires a change in organisation, how things are solved, how people cooperate both inside and outside the agency so that the IT-systems is created in new way.

%\textit{Participant \#7} also explaining that the research by Torgeir Dingsrøyr from NTNU about NAV talks a lot about coordinating teams and steering of development direction. The politicians in Norway will always give Skatteetaten new things to do or existing things to change. Thus creating the need for coordination between product areas, multiple teams and multiple parts of the business operations at Skateetaten.

%\textit{Participant \#7} also explaining that many thinks of product teams as something quite stable. However this is something which can be unwanted from the perspective of the organisation, wanting for the product teams to be more flexible. However when the organisation tries to change the staffing in an product team or development are, a lot of resistance is encountered. This due to the staff enjoying the team or area to which they belong, with the people they already work with. So organising around product teams can create stiffness in the organisation.

%\textit{Participant \#7} explaining that Skatteetaten as a different strategy and a higher architecture than NAV has. Skatteetaten has an increased focus on common components or the re-use of components. When having a high degree of reuse, it lets Skatteetaten get more quickly done with certain tasks.

%\textit{Participant \#8} explaining that in the process of making software, 20\% of the time is used to create the software. While the last 80\% is used to maintaining the software.

%\textit{Participant \#8} also explaining that a more predictable financing model would be better for Skatteetaten. If all the money for a project is used, which is a risk with an un-predictable financing model, that the financial model does not take into account unforeseen events. This can have effect on the quality of what is being made.

%\textit{Participant \#9} explaining that when it comes to the software Mattilsynet develops themselves, they have a lot to do to assure quality in their suppliers. It comes down to routines and questions are asked for potential suppliers, which might need to be deeper ingrained into Mattilsynets procurement process.

%\textit{Participant \#9} also explaining that Mattilsynet does not work project based anymore. Recently having moved from two large consultant contracts in development and operations to having their own product teams. However this change has lead to a loss of routines which were in place.

%\textit{Participant \#10} explaining that Mattilsynet has a multiple group which can define their software. There are third-party off the self software which have been acquired through tenants and a customer supplier model. The software is made by smart people, but the supplier model do not reflect well on them. The off the shelf software tends to be outdated, and their dependencies can be old. The supplier of the off the self software rarely say "no" to a feature request from a customer, so there grows complexity in the software.

%\textit{Participant \#10} also explaining that Mattilsynet has an old system called "Mats". It is not an off the shelf software, but system tailored for Mattilsynet by a contractor. This has the problem of Mattilsynet wanting functionality, without the contractor being motivated to say no. So the system has grown complexity organically. The customer in this matter being Mattilsynet, has not understood over a longer period of time the consequences of own orders. Mattilsynet has ordered functionality while ignore \gls{technical_debt}, and thinking that someone else will fix the \gls{technical_debt}, but nobody does.

%\textit{Participant \#10} also explaining that Mattilsynet has had quite weak requirement specifications for their contractors. There was not enough internal resources at Mattilsynet to create proper requirements specifications for the contractors, and the contractors themselves almost created the whole requirement specification for Mattilsynet. This was the case with Mattilsynets photo-app, where Mattilsynets inspectors can take pictures and store them in their databases, without having to use iCloud. However since the specification was weak, the ones making it did not think that someone had to maintain it for over 5 years after it was made. Thus making the decision making for technology and solution easier than it might should be, \gls{technical_debt} being created from the start.

%\textit{Participant \#10} also explaining that the assurance of software quality is quite variable in the rest of the Norwegian public sector. Areas such as healthcare, municipalities and national defence are still heavily in putting projects on tender and assigning contractors. However there are others such as NAV, Skatteetaten, NRK and Norsk Tipping which create the software themselves. The big difference is that the consultants are hired as competency as part of an already existing development team. 

%\textit{Participant \#10} also explaining the Norwegian digitisation agency's project wizard is something which can become a resting pillow for not moving to product teams. Before the new IT leadership at Mattilsynet, this was the case, that Mattilsynet should be better at projects suited for the project wizard. Mattilsynet had the need to stop with such projects, hence it was unfortunate that Mattilsynet was spending resources on becoming better at such projects. Such as actively recruiting staff that specialises in such projects, which creates slowness in the organisation. Then suddenly receiving leadership that does a 180 and switches to product teams instead. So it might have created a challenge for those with specialisation with project, it might not be big, but some challenge.

%\textit{Participant \#10} also explaining that it is better to be good at projects and bad at projects. There is a project at Mattilsynet which was delivered as a service-oriented modern architecture, the project for Meat controlling, which was a well managed project. It was so well that it became the template on how projects should be done at Mattilsynet. However it had no real strategy on how it should be maintained and how further operations should be conducted after it is delivered. So even if the project was well done, it still left a big problem for Mattilsynet. The needs for the system has also changed before it was delivered. Now this system has been has been split up between two product teams which has not been problem free, as both product teams now have to develop components which they do not understand, with technology they do not know. Therefore functional changes in such systems do not happen, as there are lots of \gls{technical_debt}s.

%\textit{Participant \#11} explaining that Mattilsynet has moved away from projects with central functions, to a model with distributed product teams. Where the responsibility for software quality is up to each team. 

%\textit{Participant \#11} also explaining that the project wizard from the Norwegian digitisation agency is not a challenge for Mattilsynet, as they have stopped using it.

%\textit{Participant \#11} also explaining that in projects scope creep is a bad thing, something to be avoid. However scope creep only mean that new insights have been discovered about the problem being solved by the software. Thus having the possibility to give more value to the end user. Scope creep can also mean that the original scope of the project has changed, so not changing the scope can mean solving the wrong problem. This is the issue with things like the project wizard, which do not take changing scope into account and the dynamic approach which can be achieved by product teams.

\subsection{Resources} \label{sec:resources}
The main challenge for the Norwegian public sector is described by participant \#1 as the lack of staff with knowledge in IT. Also explains as more agencies in the Norwegian public sector are following NAV to achieve continuous deployment and maintenance, however, it requires the agency to have their own IT resources.

%The main challenge for the Norwegian public sector is describes as the lack of staff with knowledge in IT. More agencies in the Norwegian public sector are following NAV to achieve continuous deployment and maintenance, however this requires the agency to have their own IT-resources.

The challenge of a limited amount of resources is described by Participant \#7 as being true for Skatteetaten. Leading to Skatteetaten having to prioritise what should be included when creating new software. Participant \#7 explains how this means that  Skatteetatens projects usually end up with a backlog of tasks which have been given a lower priority. As there is not enough capacity to meet a large number of wishes and needs.

%Skatteetaten have a limited amount of resources and time available. So in creating new software, prioritisation have to be made. This meaning that Skatteetatens projects usually end up with a backlog of tasks which have been give a lower priority. This backlog is usually due to lack of capacity to meet the larger amount of wishes and needs.

Some of the older systems at NAV are described by participant \#6 as being of low quality. And that NAV does not have the necessary resources to handle the lower quality at this moment. Explaining that this is the case with NAV's payment system, which each year pays out 500 Billion NOK to the Norwegian citizens, being a third of Norway's national budget. Participant \#6 explaining how it needs to be fixed: "It has to be rewritten, and we don't have the people for that right now, because we have other things that are even worse. The people who can change something like that are over 60 years old". Further explaining that general NAV is losing competency for their older systems.

%NAV have some older systems with low quality, which NAV do not have the necessary resources to handle at this moment. This is the case with NAV's payment system, which each year pays out 500 Billion NOK to the Norwegian citizens, being a third of Norway's national budget. It is something which needs to be fixed, but nobody know how to fix the system. The people who have the knowledge to fix it are over 60 years old, as the system is written in COBOL on a mainframe. In general NAV is loosing competency for their older systems.

Mattilsynet is an agency which is described by participant \#10 as having a large domain-complexity, while also having relaxed non-functional requirements. Explaining the domain complexity to be large, due to Mattilsynet being a healthcare provider for a wide range of animal species. Leading to Mattilsynet not having the capacity to maintain its operational logic, being its largest challenge. As In a year over 2000 new animal husbandry's are registered, with each being unique.

%Mattilsynet is a agencies with a large domain-complexity, while also having relaxed non-functional requirements. The domain-complexity is quite large, due to what is described that Mattilsynet is a healthcare provider, only for more animal species than just humans. Meaning that Mattilsynet not having the capacity to maintain their operational logic, being their largest challenge. In a year over 2000 new animal husbandry's are registered, with each being unique.  

The importance of central resources in an organisation which the product teams can take advantage of is described by participant \#11. Explaining that resources such as security experts and legal advisers which the product teams can use in their projects being hired at Mattilsynet. There are not enough resources for the product teams themselves to have this expertise, but this allows them to be somewhat available for the teams, which is important for software quality.

%It is important to have central resources in an organisation which the product teams can take advantage of. Such as security experts and legal advisers, which the product teams can use in their projects. There are not enough resources for the product teams themselves to have this expertise, but this allows them to be somewhat available for the teams, which is important for software quality.

%\textit{Participant \#1} explaining that the main challenge for Norwegian public sector is the lack av people with IT-knowledge.

%\textit{Participant \#1} also explaining that more agencies in the Norwegian public sector are doing like NAV. That what is needed to create good software is continuous development and maintenance. However this requires the agency to have their own IT resources.

%&\textit{Participant \#6} explaining that NAV has a old systems with low quality, which NAV does not have the resources to handle at this moment. An example of this is their payment system, which each year pays out 500 billion NOK to the Norwegian citizen, a third of the Norway's national budget. It is something which needs fixing, but nobody can fix it. The people who are able to fix it are over 60 years old, as it are written in COBOL on a mainframe. NAV is loosing competency on the older systems.

%\textit{Participant \#7} explaining that Skatteetaen have a limited amount of resources and time available. So in creating new software prioritisation have to be made. Thus meaning that Skatteetaten's projects usually end up with a backlog of tasks which have been have de-prioritised, because there is a limited amount of time. This backlog is also due to lack of capacity to meet to meet the large amount of wishes and needs.

%\textit{Participant \#10} explaining that Mattilsynet has a very large domain-complexity, while also having relaxed non-functional requirements. The domain-complexity is quite large, due to it being a mini-healthcare, only for more than just humans, but a wide range of animal species. This means that Mattilsynet does not have the capacity to maintain their operational logic, which is their largest challenge. In a year over 2000 new animal husbandry's are registered, which may not sound like a lot, but each of these are unique.

%\textit{Participant \#11} explaining that it is important to have central resources in an organisation for the product teams. Such as security experts and legal advisers. Then teams can use this special expertise in their projects. There are not enough resources for the teams themselves to have this expertise, but this allows them to be somewhat available for the teams, which is important for software quality.

\subsection{Legal} \label{sec:legal}
Software quality is described by participant \#7 as a piece of software solving a problem, and that the software covers certain requirements. Not just user requirements, but operational requirements such as legal requirements. Participant \#7 explains that Skatteetaten is driven by legal requirements: "Skatteetaten are heavily driven by legal requirements. It sets a lot of demands on how things should be developed. And we cannot create a solution which does not comply with the legal requirements". Explaining that Skatteetatens software should have good usability, but is mainly driven by legal requirements.

%Software quality is described as a piece of software solves a problem, and that the software cover certain requirements. Not just user requirements, but operation requirement such as legal requirements. The requirements of software at Skatteetaten are heavily driven by legal requirements, which drives how a piece of software at Skatteetaten should function. Skatteetatens software should have good usability, but are manly driven from legal requirements.

Legal requirements are also described by Participant \#7 as a challenge for Skatteetaten's mission to create trust in the Norwegian populous. So that each Norwegian citizen is motivated to pay their taxes to fund the Norwegian welfare state. This is described by participant \#7 as being achievable through sharing data about Norwegian society, as Skatteetaten has a wide knowledge of this. However difficult, as the legal requirements are described as putting limitations on data sharing.

%One of Skatteetatens missions is to create trust in the Norwegian populous, so that each Norwegian citizen is motivated to pay their taxes, for funding the Norwegian welfare state. A way to achieve this is the sharing of data about the Norwegian society, as Skatteetaten has a wide knowledge about the Norwegian society. However this sharing is difficult, due to legal requirements which are putting limitations this data sharing.

Another challenge with legal requirements described by participant \#7 is that these are not digitisation friendly, being difficult to simplify for Skatteetatens operations. In a project in cooperation with the Brønnøysund Register Centre, participant \#7 describes how Skatteetaten and Brønnøysund wanted to create a single interface where businesses owners could see information about requirements from both Skatteetaten and Brønnøysund. Explaining that Skatteetaten did not receive any data from Brønnøysund, and Brønnøysund did not get any data from Skatteetaten. However, it is described by participant \#7 that the laws did not allow this, even though the business owners own the data. Yet Skattetaten is allowed to share data with other parts of the Norwegian public sector, such as the health sector. Leading to legal requirements setting hard limitations on delivering good user experiences.

%Skatteetaten are also struggling with legal requirements that are not digitisation friendly, and being difficult to simplify these for their operations. In a project in cooperation withe The Brønnøysund Register centre, Skatteetaten and Brønnøysund wanted to create a single panel where businesses owners could see information about requreients from both Skatteetaten and Brønnøysund. Skatteetaten did not get any data from Brønnøysund, and Brønnøysund did not get any data from Skatteetaten. However the laws did not allow this, even though the business owners owning the data, it was not allowed to combine the data and show it to the user. This at the same time, Skatteetaten is allowed to share data to other parts of the Norwegian public sector, such as the health sector. In general the legal requirements set hard limitations to deliver good user experiences.

%NAV and Skatteetaten are agencies which are described as having the ability to maintain the legal requirements for their software. However other agencies such as the Norwegian police are not always able to maintain the legal requirements for their software.

%It is described that at NAV, only 20\% of their IT operations are bound to legal requirements. However at Skatteetaten it is the opposite, 80\% of their IT operations are bound to legal requirements. Due to such a large percentage being bound to legal requirements, Skatteetaten has to de-prioritse user needs in order to fulfil the legal requirements, even tough Skatteeten always has their users in mind.

%\textit{Participant \#2} explaining that it is a challenge for the Norwegian public sector to gain feedback from their users, as there is a big focus on GDPR. And its bad for the public sector if they use data about their citizens, which they are not allowed to, due to the power they have over their users.

%\textit{Participant \#7} explaining that software quality is that it the software solves what it should solve, that it covers needs. Not just user needs, but operation need, such as legal requirements. The needs of Skatteetaten are heavily driven by legal requirements, as the legal requirements say how thing should be and how the software should function. The software should have good usability, however the software at Skatteetaten usually are created from legal requirements, not user needs.

%\textit{Participant \#7} also explaining that if looking at the legal requirements of a piece of software, then Skatteetaten and NAV are able to change their software due to new Norwegian laws. However the Norwegian police are not always able to change their software due to changes in the Norwegian laws.

%\textit{Participant \#7} also explaining that he was talking with the Head of Architecture at NAV, and heard that only 20\% of their IT operations were bound to legal requirements. However at Skatteetaten it is the opposite, 80\% of their IT operations bound to legal requirements. Due to such a large percentage being bound to legal requirements, means that Skatteetaten has to de-prioritise user needs in order to fulfil the legal requirements, even tough Skatteetaten always has their users in mind.

%\textit{Participant \#7} also explaining that Skatteetaten is struggling with legal requirements which are not digitisation friendly, and making these simpler for their operations. In a project in cooperation with The Brønnøysund Register centre, Skatteetaten and Brønnøysund wanted to create a panel where busuiness owners could see information about requirements from both Skatteetaten and Brønnøysund at the same place, which was a wish from the Norwegian business owners. Skatteetaten did not get any data from Brønnøysund, and Brønnøysund did not get any data from Skatteetaen. However the laws did not allow this, even tough the user owned the data, they were not allowed to combine it and show it to the user. On the other side, Skatteetaten are allowed to share data to other parts of the Norwegian public sector, such as the health sector. Overall the legal requirements set hard limitations to deliver good user experiences.

%\textit{Participant \#7} also explaining that Skatteetaten shall create trust in the Norwegian populous so that each citizen is motivated to pay their taxes, to pay for the Norwegian welfare state. The sharing of data and information is also part of Skatteetatens social missions for Norway. This is not only about taxes, but Skatteetaten know alot about the Norwegian society in general and the Norwegian private sector, and a lot of this data is used to improve the Norwegian society. However this sharing is difficult, due to legal requirements which are putting limitations on this data sharing.

%\textit{Participant \#10} explaining that Mattilsynets core system, Mats, are affected by a wide range of legal requirements. When the system got a bug, a wide range of legal requirements was affected.

%\textit{Participant \#11} explaining that in the Norwegian public sector there are legal requirements related to archive laws, which states that specific documents should be archived or recorded in journals, which are things that should be handled automatically.

%\textit{Participant \#11} also explaining that to help their development teams with legal requirements and GDPR, a legal adviser has recently been hired, which will be of great help in raising knowledge on GDPR in the different product teams.

\subsection{External Revision} \label{sec:external_revision}
The process for software quality assurance at NAV is described by participant \#5 as not being fully documented. Explaining that If a project at NAV is subject to a revision, it would be a challenge to get a complete overview of the quality assurance methods used. And in such an event, the development teams would need to be contacted and asked to document these processes. However, participant \#5 describes this lack of documentation to have some advantages: "The advantage is that we do not create documentation that is not actually used in practice".

%Processes for software quality assurance at NAV are not fully documented. If a project at NAV is subject to a revision, it would be challenging to get a complete overview of the quality assurance methods. The development teams would need to be contacted and ask to document these processes. However this lack of documentation do have some benefits, like the teams at NAV not having to spend large amounts of time on documenting their processes, just for documentation sake.

It is explained by Participant \#5, that recently there has been an external evaluation of the development of the system for sick pay at NAV. Explaining that this is something which occurs rarely, only done in this case due to the system receiving a lot of attention due to its development progress not being satisfactory. Resulting in external quality assurance of the system, including software quality.

%Recently there has been an external evaluation for the development of the system for sick pay at NAV. However this is something which occurs rarely, only done in this case due to the system receiving a lot of attention due to its development progress not being satisfactory. Resulting in an external quality assurance of the system, including software quality.

The Norwegian Broadcasting Corporation is described by participant \#11 as still being focused on projects with external revision to ensure that their software is of quality. Which is described by participant \#11 as doing little to actually assure quality in software.

%The Norwegian Broadcasting Corporation is described as still having a focus on projects with external revisions to ensure that their software is of quality. Something that is described as doing to little to actually assure quality in the software

%\textit{Participant \#5} also explaining that some of the processes for software quality assurance at NAV are not fully documented. In a revision situation of a project, it would be challenging to get a overview of the quality assurance methods, the development teams would have to be asked to document their processes. This has benefits, which is that the teams at NAV do not have to spend much time on documenting their processes just for documentation sake.

%\textit{Participant \#5} explaining that there has been an external evaluation for the development of the system for sick pay at NAV. However this is not usually done, as this system had a lot of attention, and the progress was not satisfactory. So there has been external quality assurance of a lot of aspects of this system, including the software quality, which is not often done.

%\textit{Participant \#10} explaining that at NRK there was still a lot projects, with the focus on external revisions or performing quality assurance of the projects. However this did little to actually assure quality in the project.