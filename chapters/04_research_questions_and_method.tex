\chapter{Research Questions and Method}

\section{Research Questions} 
This study has the main objective of understanding how the different agencies in the Norwegian public sector assure quality in their software. The study also have the objective of understanding how well the software quality assurance in the Norwegian public sector compares with what is recommended by the relevant literature shown in \autoref{sec:related_research}. This leading to the following research questions:

\begin{itemize}
    \item \textbf{RQ1:} How do the different Norwegian public sector's assure quality of their software?
    \item \textbf{RQ2:} How does the process of software quality assurance in the Norwegian public sector compare to what is stated in relevant literature?
\end{itemize}

\section{Method}
In order to gain a understanding in how the Norwegian public sector assures quality in its software, it was decided to do a survey involving different departments in the Norwegian public sector. The method for the survey follows the recommendations of the book "Researching Information Systems and Computing", and includes the following steps: (I) Data requirements, (II) Data generation method, (III) Sampling frame, (IV) Sampling technique, (V) Response rate and non-respondents, and (VI) Sample size \cite{bjo_2022}.

\subsection{Data requirements}
The data needed to be collected is both directly and in-directly topic-related. The directly topic-related data requirements relates to specifics in the research questions, such as what methods are used for assuring quality in software. The in-direct topic-related data is used for connecting the directly-topic related data \cite{bjo_2022}, i.e. connecting software quality assurance to the context of the Norwegian public sector.

%Directly topic-related -> developers, designers etc are directly involved in assuring the quality of software

\subsubsection{Directly topic-related}
The following data is collected that is directly topic-related:
\begin{itemize}
    \item Non-functional requirements.
    \item Software quality assurance methods.
    \item Confidence in software quality assurance method.
\end{itemize}

\subsubsection{In-Directly topic-related}
The following data is collected that is in-directly topic-related:
\begin{itemize}
    \item Department/agency 
    \item Area of responsibility
\end{itemize}

\subsection{Data generation method}

\subsubsection{Questionnaire}
Self administered


\subsubsection{Sampling frame}
A list of the Norwegian public sector agencies with a substantial IT-systems and IT-sections in their agency

\begin{table}[H]
\begin{tabular}{ll}
\hline
Department & Agency \\ \hline
           &        \\
           &        \\
           &        \\
           &        \\ \hline
\end{tabular}%
\end{table}
\subsubsection{Sampling technique}

\subsubsection{Response rate}

\subsubsection{Non-respondents}


\subsection{Data synthesis}