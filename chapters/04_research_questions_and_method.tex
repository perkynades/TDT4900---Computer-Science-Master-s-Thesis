\chapter{Method}

%\section{Research Questions} 
%This study has the main objective of understanding how the different agencies in the Norwegian public sector assure quality in their software. The study also have the objective of understanding how well the software quality assurance in the Norwegian public sector compares with what is recommended by the relevant literature shown in \autoref{sec:related_research}. This leading to the following research questions:

%\begin{itemize}
%    \item \textbf{RQ1:} How do the different Norwegian public sector's assure quality of their software?
%    \item \textbf{RQ2:} How does the process of software quality assurance in the Norwegian public sector compare to what is stated in relevant literature?
%\end{itemize}

%\section{Method}
As stated, the objective of this study is to understand what the people with high influence or ownership over technology in the Norwegian public sector considers important in regards to assuring quality in its software and what factors might influence their view on this.

\textcolor{red}{Manglende konkret forskning på kvalitetssikring av programvare i offentlig sektor + min erfaring som utvikler hos NAV IT og ryktene man hører gjør at Case Study passer}

\textcolor{red}{Mange forskjellige synsvinkler på "fenomenet" kvalitetssikring av programvare i offentlig sektor}

\subsection{Case Study}

\textcolor{red}{Her må jeg skrive om hvorfor case study passer, fordi det er litt experimental field}

\paragraph{Exploratory study} \textcolor{red}{Det er ikke så mye definert litteratur på området og feltet trenger mer utforsking}

\paragraph{Short term, contemporary study} \textcolor{red}{Skal finne ut av hvordan situasjonen er nå}

\subsubsection{Selection of case}
\begin{enumerate}
    \item Typical instance -> Kan være for andre caser i software verden og andre offentlige sektorer
    \item Convenience -> Jeg jobber i offentlig sektor, derfor gjør det ting lettere å gjennomføre denne case studyen
\end{enumerate}

\subsection{Data generation methods}

\subsubsection{Interviews}

\textcolor{red}{Huske å få inn her at jeg vil også ha "expert interviews" fra forskere i feltet}

\paragraph{Semi-structured}

\paragraph{Recording}
\begin{itemize}
    \item Field notes
    \item Audio recording
\end{itemize}


\begin{comment}
In order to gain a understanding in how the Norwegian public sector assures quality in its software, it was decided to do a survey involving different departments in the Norwegian public sector. The method for the survey follows the recommendations of the book "Researching Information Systems and Computing", and includes the following steps: (I) Data requirements, (II) Data generation method, (III) Sampling frame, (IV) Sampling technique, (V) Response rate and non-respondents, and (VI) Sample size \cite{bjo_2022}.

\subsection{Data requirements}
The data needed to be collected is both directly and in-directly topic-related. The directly topic-related data requirements relates to specifics in the research questions, such as what methods are used for assuring quality in software. The in-direct topic-related data is used for connecting the directly-topic related data \cite{bjo_2022}, i.e. connecting software quality assurance to the context of the Norwegian public sector.

%Directly topic-related -> developers, designers etc are directly involved in assuring the quality of software

\subsubsection{Directly topic-related}
The following data is collected that is directly topic-related:
\begin{itemize}
    \item Non-functional requirements.
    \item Software quality assurance methods.
    \item Confidence in software quality assurance method.
\end{itemize}

\subsubsection{In-Directly topic-related}
The following data is collected that is in-directly topic-related:
\begin{itemize}
    \item Agency
    \item Role/technical responsibility
    \item Permanent employee or consultant
\end{itemize}

\subsection{Data generation method}

\subsubsection{Questionnaire}
A self-administered questionnaire is used to gain insight into the surveys research questions. As recommended  \cite{bjo_2022}, there are several reasons for using a self-administered questionnaire:

\begin{itemize}
    \item The survey intends to gather brief and uncontroversial data from a large number of people
    \item The data in the survey needs to be standardised
    \item Geographical location between potential respondents and researchers makes personal interviews unpractical
    \item Researcher does not have time for personal interviews, but has time to wait between distributing the questionnaire and getting responses
    \item In-experience by the researcher can lead to cognitive bias in the data if questionnaire was researcher-administered
\end{itemize}


\subsubsection{Structure of questionnaire}
\textcolor{red}{Når den er mer spikra, skrive om den her
\begin{enumerate}
    \item Hvilken etat tilhører du?
    \item Hvilken rolle har du? (Skal utvikler splittes opp i separate roller? Frontend, backend osv..)
    \item Er du internt ansatt eller innleid (konsulent)?
    \item Hvilke av disse har du fokus på i rollen din? (Liste med non-functionals requirements)
    \item Hvilke metoder bruker du for å sikre at disse er av kvalitet? (Liste med metoder? Fritekst?)
    \item På en skala fra 1-5, hvordan synes du disse funker? (Tall skala eller "ord" skala).
\end{enumerate}
}

\subsection{Sampling frame}
The following agencies in the Norwegian public sector will be used as the sampling frame for the survey, mainly targeting each agency's IT-department.

\begin{table}[H]
\begin{tabular}{ll}
\hline
Department & Agency \\ \hline
           &        \\
           &        \\
           &        \\
           &        \\ \hline
\end{tabular}%
\centering
\caption{Sampling frame of Norwegian public sector agencies included in the survey}
\label{tab:sample_frame}
\end{table}

\subsection{Sampling technique}
The sampling technique for the survey is primarily probabilistic, as it is believed that the respondents of the survey is representative to the population of the area being researched.

Cluster sampling

\subsection{Response rate}

\subsection{Non-respondents}


\subsection{Data synthesis}
\end{comment}