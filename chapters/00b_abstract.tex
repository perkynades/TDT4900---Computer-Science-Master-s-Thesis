\chapter*{Abstract}
This study aims to uncover the practices and challenges of software quality assurance in the Norwegian public sector. A multiple case study with interviews of employees from certain agencies in the Norwegian public sector was used to obtain the findings of this study. With the employees interviewed had a strong influence or ownership over technology in the agencies chosen in this study. The results show several practices being used to assure the technical quality of software and different software development methods used to assure quality. Challenges in software quality assurance relating to organisational aspects in the Norwegian public sector were discovered, such as budgeting and lack of resources. Forcing the agencies to use the criticised project wizard delivered by the Norwegian digitisation agency. Relevant scientific literature suggests that the practices followed by the Norwegian public sector are recommended for assuring quality software. However, the challenges presented in the study must be addressed before the Norwegian public sector can elevate the quality of its software.