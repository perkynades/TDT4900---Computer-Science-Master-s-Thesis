\chapter*{Abstract}
This study aims to uncover the practices and challenges of software quality assurance in the Norwegian public sector. A multiple case study with interviews of employees from certain agencies in the Norwegian public sector was used to obtain the results of this study. Where the employees interviewed have a strong influence or ownership over technology in the agencies chosen. The results show several practices being used to assure the technical quality of software and different software development methods used to assure quality. Challenges in software quality assurance relating to organisational aspects in the Norwegian public sector were discovered, such as budgeting and lack of resources. Forcing the agencies to use the criticised project wizard delivered by the Norwegian digitisation agency. Relevant scientific literature suggests that the practices followed by the Norwegian public sector are recommended for assuring quality software. However, the challenges presented in the study must be addressed before the Norwegian public sector can elevate the quality of its software.

% ---- PÅ NORSK
%Denne studien har som mål å avdekke praksiser og utfordringer innen kvalitetssikring av programvare i den norske offentlige sektoren. En flercase-studie med intervjuer av ansatte fra visse etater i den norske offentlige sektoren ble brukt for å oppnå resultatene i denne studien. De ansatte som ble intervjuet har en sterk innflytelse eller eierskap over teknologi i de utvalgte etatene. Resultatene viser flere praksiser som brukes for å sikre den tekniske kvaliteten i programvaren og ulike programvareutviklingsmetoder som brukes for å sikre programvarekvalitet. Utfordringer knyttet til kvalitetssikring av programvare omhandlet aspekter knyttet til det organisatoriske i den norske offentlige sektoren ble oppdaget, som budsjettering og mangel på ressurser. Noe som fører til at etatene blir tvunget til å bruke den kritiserte prosjektveiviseren levert av Digitaliseringsdirektoratet. Vitenskapelig litteratur antyder at praksisene fulgt av den norske offentlige sektoren anbefales for å sikre kvalitet på programvaren. Imidlertid må utfordringene som presenteres i studien løses før den norske offentlige sektoren kan øke kvaliteten på sin programvare.