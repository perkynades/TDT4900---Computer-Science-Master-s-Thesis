\chapter{Method}

\textcolor{red}{Skal jeg få inn litt om etikk her? Er jo litt fare at jeg er både utvikler i offentlig sektor og forsker. Kan gi "validity" problems. Er også problemer med at de andre ansatte kan lettere gi med sensitive dokumenter/opplysninger, som ikke er egnet å publiseres (større ansvar for det etiske pga min posisjon)}

%\section{Research Questions} 
%This study has the main objective of understanding how the different agencies in the Norwegian public sector assure quality in their software. The study also have the objective of understanding how well the software quality assurance in the Norwegian public sector compares with what is recommended by the relevant literature shown in \autoref{sec:related_research}. This leading to the following research questions:

%\begin{itemize}
%    \item \textbf{RQ1:} How do the different Norwegian public sector's assure quality of their software?
%    \item \textbf{RQ2:} How does the process of software quality assurance in the Norwegian public sector compare to what is stated in relevant literature?
%\end{itemize}

%\section{Method}
As stated, the objective of this study is to understand what the people with high influence or ownership over technology in the Norwegian public sector considers important in regards to assuring quality in its software and what factors might influence their view on this. Due to this being a broader theme or phenomena \cite{bjo_2022} without a clear answer as shown in \autoref{sec:related_research}, it is decided that a case study with the context of the Norwegian public sector is used as an appropriate method. The case study uses the experience of the researcher as a software developer in the Norwegian public sector and interview's as data generation methods.

\section{Selection of method}
There are several reasons for why case study is chosen as the method for this study \cite{bjo_2022}, where the main reason is that software quality assurance in the Norwegian has little research. As well as the term "software quality assurance" can be interpreted in multiple ways with different points of views. A case study therefor allows the researcher to gain a deeper understanding of the topic from multiple sources, both qualitative and quantitative, which gives the researcher a larger range of data to be interpreted \cite{bjo_2022}.

\section{Case Study}

\subsection{Type of case study}
\paragraph{Exploratory study}
As there is little research on the research topic in this study, and what research does exist is mostly broad and qualitative, it is decided that an exploratory study is used to gain an deeper understanding in the research topic \cite{bjo_2022}.

\paragraph{Short term, contemporary study} 
The case study is a short term, contemporary study as the researcher is interested in what is the understanding of software quality assurance in the Norwegian public sector at the time of this study \cite{bjo_2022}. It is also decided due to the limited time resources in the production of the study, hence a longitudinal study would not be possible to produce.

\subsection{Selection of case}
There are several for why software quality assurance in the Norwegian public sector is subject to a case study

\paragraph{Typical instance}
Software quality assurance is important for other areas than just in the Norwegian public sector. The results of this case study could be applied to other nations public sectors, or other organisations in different private sectors.

\paragraph{Convenience}
As the researcher works as a software developer in the Norwegian public sector, it is a convenient case to study. The researcher already has contact with potential participants for the study and in-direct contacts with other potential participants in the Norwegian public sector.

\paragraph{Unique opportunity}
The researchers unique position as both a researcher and software developer in the Norwegian public sector gives an unique opportunity for the chosen case. The researcher is able to bring first hand accounts with own experience while also conducting research which could support or contradict those experiences.

\subsection{Interviews}
The main data generation method in the case study is interviews as this allows the researched to gain a deeper understanding in a topic or area of research.


\paragraph{Experts}
Some experts in the field of software development in the Norwegian public sector and/or large Norwegian organisations is interviewed to gain their view on the study's objective. These experts are researchers in the mention field, and have several publications relative to the study's objective.

\paragraph{Agencies}
Employees at IT-departments in different agencies of the Norwegian public sector is interviewed, where the employees has a high influence on technology and/or methodology in the IT-department. The agencies where employees are interview is:

\begin{itemize}
    \item Arbeids- og velferdsdirektoratet (Norwegian Labour and Welfare Administration)
    \item Skatteetaten (Norwegian Tax Administration)
    \item Digitaliseringsdirektoratet (Norwegian Digitalisation Agency)
    \item Mattilsynet (Norwegian Food Safety Authority)
\end{itemize}

\subsubsection{Participants}
\autoref{tab:expert_participants} and \autoref{tab:agency_participants} shows the different interview objects participating in the case study.

\paragraph{Experts} \hspace{0cm}
\begin{table}[H]
\centering
\begin{tabular}{|l|l|l|}
\hline
\textbf{Participant ID} & \textbf{Organisation} & \textbf{Position} \\ \hline
\#1 & NTNU & Professor/Researcher \\ \hline
\#2 & Sintef & Researcher \\ \hline
\end{tabular}
\caption{Experts on the Norwegian public sector participating as interview objects}
\label{tab:expert_participants}
\end{table}

\paragraph{Agencies} \hspace{0cm}
\begin{table}[H]
\centering
\begin{tabular}{|l|l|l|}
\hline
\textbf{Participant ID} & \textbf{Agency} & \textbf{Position} \\ \hline
\#3 & NAV & Technology Principal \\ \hline
\#4 & NAV & Head of an IT-area \\ \hline
\#5 & NAV & Head of Architecture \\ \hline
\#6 & NAV & Technology Principal \\ \hline
\#7 & Skatteetaten & Head of Project Managers \\ \hline
\#8 & Skatteetaten & Head of an IT-area \\ \hline
\#9 & Mattilsynet & Head of Platform \\ \hline
\#10 & Mattilsynet & Project Leader/Architect \\ \hline
\#11 & Mattilsynet & Head of Product Development \\ \hline
\end{tabular}
\caption{Employees from Norwegian public sector agencies participating as interview objects}
\label{tab:agency_participants}
\end{table}


\subsubsection{Structure}
The interviews are mainly structured as semi-structured interviews, which allows the researcher to diverge from the interview plan to ask unplanned questions which the researcher find interesting to answering the objective of study. The interview held for the experts and the agencies are worded somewhat different, as there is a difference in perspective. The experts are asked about the whole of the Norwegian public sector, while the agency are asked about the situation in their agency.

\paragraph{Expert interview}
The expert interview is structured as follows:

\begin{enumerate}
    \item What kind of research are you conducting?
    \item What comes to mind when you hear "software quality"?
    \item Can you say something about software quality in the Norwegian public sector?
    \item What comes to mind when you hear "software quality assurance"?
    \item How does the Norwegian public sector assure quality in the software they develop and maintain?
    \item What challenges does the Norwegian public sector face in the assurance of quality in their software?
\end{enumerate}

\paragraph{Agencies interview}
The interview is structured as follows:

\begin{enumerate}
    \item What is your role in your agency?
    \item What comes to mind when you hear "software quality"?
    \item Can you say something about software quality in your agency?
    \item What comes to mind when you hear "software quality assurance"?
    \item How does your agency assure quality in the software you develop and maintain?
    \item How do you think the assurance of software quality is in the rest of the Norwegian public sector?
    \item What challenges does your agency have in the assurance of quality in its software?
\end{enumerate}

\subsubsection{Recording}
In order to gain a complete and contextual data from the interviews, field notes and audio recording is used as recording methods. The field notes allows the researcher to note the context of the interview and anything noteworthy to be explored later in the interview or after the interview. While the audio recording is used to listen to the interview multiple times, or to be transcribed and used in a textual data analysis \cite{bjo_2022}.


\subsection{Data analysis}
After each interview in the case study, the records were subject to data preparation through transcription. The transcriptions were then subject to a thematic analysis.

\subsubsection{Data preparation}
Before any analysis of the data collected in the case study could be analysed, it needed to be prepared. For the interviews this was done in two ways: (I) live transcript in Microsoft Teams, (II) transcription using Whisper, and (III) transcription using Autotekst.

\paragraph{Transcription using Microsoft Teams}
Transcription in Microsoft Teams is available in any ongoing meetings, and can be set to either be just visible during the meeting, or be persistent and stored in a Microsoft Teams group. After starting any meeting, the host can start to record the audio from the meeting, in the recording menu there is an option to also record transcription. When starting the transcribing, a specific language can be chosen, all of the digital interviews were held in Norwegian, so the transcription language was set to Norwegian. When the host stops recording, the transcription files will be available as \textbf{.docx} or \textbf{.vtt} files in the Teams group \cite{mt_2022}.

\paragraph{Transcription using Whisper}
Whisper is an open source speech recognition software, built by OpenAI to increase the performance and accuracy in recognising speech \cite{oa_2022}. Whisper has three modes of accuracy in speech recognition: (I) small, (II) medium, and (III) large. With small being the most efficient, but least accurate, and large being the least efficient, but most accurate. This study used the large model, as this yielded the most accurate transcribing. Whisper takes an \textbf{.wav} file as input, and can output in multiple file formats, such as \textbf{.txt} and \textbf{.vtt}.

\paragraph{Transcription using Autotekst}
\textcolor{red}{Snakke om denne her, endre paragrafen under også (og over)}

Both transcription in Microsoft Teams and Whisper do not provide a 100\% perfectly accurate transcription, but are accurate enough to understand what message and meaning is being communicated in the interview. This is considered accurate enough for this study, as it still enables the researcher to perform a thematic analysis on the transcribed interviews.

\subsubsection{Thematic analysis}
In order to generate results from the interviews in the case study, a thematic analysis is performed. The first step of the thematic analysis is to create general categorisations from the interviews in relation with the objective of the study \cite{bjo_2022}. The general categories are drawn from the field notes of each interview, which then later makes it easier to categorise the different sections of the interviews. \autoref{tab:example_interview_categorization} shows an example of this:

\paragraph{Example interview - John Doe/01.01.2022} \hspace{0cm}

\begin{table}[h]
\begin{tabular}{|p{0.75\linewidth}|p{0.25\linewidth}|}
\hline
\textbf{Section(s)} & \textbf{Theme} \\ \hline
We mostly use unit testing and regressions tests to ensure that the quality is good... & Technical testing \\ \hline
There is an increased focus on knowledge sharing though the organisation & Culture \\ \hline
... & ... \\ \hline
\end{tabular}
\caption{Example table for categorising sections of an interview}
\label{tab:example_interview_categorization}
\end{table}

To categorise which themes are shared between interviews, a table of interconnected themes between interviews participants is created \cite{bjo_2022}. \autoref{tab:interconnected_themes} is the following result of this. This table however is only an rough overview used to aid the researcher in presenting the results, some of the themes might be falsely identified, but it still of great help in presenting the results from the interviews.

\begin{table}[H]
\resizebox{\columnwidth}{!}{%
\begin{tabular}{|l|l|l|l|l|l|l|l|l|l|l|l|}
\hline
\textbf{Participant ID / Theme} & \#1 & \#2 & \#3 & \#4 & \#5 & \#6 & \#7 & \#8 & \#9 & \#10 & \#11 \\ \hline
Non-functional requirements     & X   & X   &     & X   & X   &     & X   & X   & X   &      & X    \\ \hline
Agile                           & X   &     & X   &     &     &     &     & X   & X   &      &      \\ \hline
Simulation                      & X   &     &     &     &     &     &     &     &     &      &      \\ \hline
Security                        & X   &     &     &     &     &     & X   & X   & X   &      &      \\ \hline
Inspection                      & X   &     &     &     &     &     &     &     &     & X    &      \\ \hline
DevOps                          & X   &     & X   & X   & X   &     & X   &     &     & X    &      \\ \hline
Architecture                    & X   &     &     &     &     &     & X   &     &     &      &      \\ \hline
Team                            & X   & X   &     &     & X   & X   &     & X   & X   & X    & X    \\ \hline
Testing                         & X   &     & X   & X   &     & X   & X   & X   &     & X    &      \\ \hline
Technical debt                  & X   &     & X   & X   & X   &     &     & X   & X   & X    & X    \\ \hline
Knowledge                       & X   &     &     & X   &     & X   & X   &     &     &      &      \\ \hline
Method                          & X   & X   &     &     & X   &     & X   & X   & X   & X    &      \\ \hline
Project                         & X   & X   & X   & X   & X   & X   & X   & X   & X   & X    & X    \\ \hline
Resources                       & X   &     &     &     &     & X   & X   &     &     &      &      \\ \hline
Feedback                        & X   & X   & X   &     &     &     &     &     &     &      &      \\ \hline
Definition                      &     & X   &     &     &     & X   &     &     & X   & X    & X    \\ \hline
User                            &     & X   &     &     &     &     &     &     &     &      &      \\ \hline
Contextual                      &     & X   &     &     &     & X   &     &     &     &      &      \\ \hline
Measurements                    &     &     &     &     & X   &     & X   &     &     &      &      \\ \hline
Revision                        &     &     &     &     & X   &     &     &     &     &      & X    \\ \hline
Legal                           &     & X   &     &     &     &     & X   &     &     & X    & X    \\ \hline
Modelling                       &     &     &     &     &     &     &     &     &     & X    &      \\ \hline
Languages                       &     &     &     &     &     &     &     &     &     & X    &      \\ \hline
Domain                          &     &     &     &     &     &     &     &     &     & X    &      \\ \hline
\end{tabular}%
}
\caption{Table of interconnected themes between interview participants}
\label{tab:interconnected_themes}
\end{table}

\begin{comment}
In order to gain a understanding in how the Norwegian public sector assures quality in its software, it was decided to do a survey involving different departments in the Norwegian public sector. The method for the survey follows the recommendations of the book "Researching Information Systems and Computing", and includes the following steps: (I) Data requirements, (II) Data generation method, (III) Sampling frame, (IV) Sampling technique, (V) Response rate and non-respondents, and (VI) Sample size \cite{bjo_2022}.

\subsection{Data requirements}
The data needed to be collected is both directly and in-directly topic-related. The directly topic-related data requirements relates to specifics in the research questions, such as what methods are used for assuring quality in software. The in-direct topic-related data is used for connecting the directly-topic related data \cite{bjo_2022}, i.e. connecting software quality assurance to the context of the Norwegian public sector.

%Directly topic-related -> developers, designers etc are directly involved in assuring the quality of software

\subsubsection{Directly topic-related}
The following data is collected that is directly topic-related:
\begin{itemize}
    \item Non-functional requirements.
    \item Software quality assurance methods.
    \item Confidence in software quality assurance method.
\end{itemize}

\subsubsection{In-Directly topic-related}
The following data is collected that is in-directly topic-related:
\begin{itemize}
    \item Agency
    \item Role/technical responsibility
    \item Permanent employee or consultant
\end{itemize}

\subsection{Data generation method}

\subsubsection{Questionnaire}
A self-administered questionnaire is used to gain insight into the surveys research questions. As recommended  \cite{bjo_2022}, there are several reasons for using a self-administered questionnaire:

\begin{itemize}
    \item The survey intends to gather brief and uncontroversial data from a large number of people
    \item The data in the survey needs to be standardised
    \item Geographical location between potential respondents and researchers makes personal interviews unpractical
    \item Researcher does not have time for personal interviews, but has time to wait between distributing the questionnaire and getting responses
    \item In-experience by the researcher can lead to cognitive bias in the data if questionnaire was researcher-administered
\end{itemize}


\subsubsection{Structure of questionnaire}
\textcolor{red}{Når den er mer spikra, skrive om den her
\begin{enumerate}
    \item Hvilken etat tilhører du?
    \item Hvilken rolle har du? (Skal utvikler splittes opp i separate roller? Frontend, backend osv..)
    \item Er du internt ansatt eller innleid (konsulent)?
    \item Hvilke av disse har du fokus på i rollen din? (Liste med non-functionals requirements)
    \item Hvilke metoder bruker du for å sikre at disse er av kvalitet? (Liste med metoder? Fritekst?)
    \item På en skala fra 1-5, hvordan synes du disse funker? (Tall skala eller "ord" skala).
\end{enumerate}
}

\subsection{Sampling frame}
The following agencies in the Norwegian public sector will be used as the sampling frame for the survey, mainly targeting each agency's IT-department.

\begin{table}[H]
\begin{tabular}{ll}
\hline
Department & Agency \\ \hline
           &        \\
           &        \\
           &        \\
           &        \\ \hline
\end{tabular}%
\centering
\caption{Sampling frame of Norwegian public sector agencies included in the survey}
\label{tab:sample_frame}
\end{table}

\subsection{Sampling technique}
The sampling technique for the survey is primarily probabilistic, as it is believed that the respondents of the survey is representative to the population of the area being researched.

Cluster sampling

\subsection{Response rate}

\subsection{Non-respondents}


\subsection{Data synthesis}
\end{comment}