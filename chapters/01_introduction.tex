\chapter{Introduction}

\section{Motivation}

\textcolor{red}{Her må jeg først få inn 3 ting:
\begin{itemize}
    \item Hva software quality og software quality assurance er
    \item Hvorfor dette er viktig
    \item Hvorfor dette er viktig i kontekts av norsk offentlig sektor
\end{itemize}
}
Software quality assurance works to ensure that a software development team delivers a product that is of expectation to the end-users and meets the projects requirement specification.

%This increase in co-operation in the Norwegian public sector using digital tools. Thus leading to common quality attributes for the end-users of the different departments in the Norwegian public sector. \textcolor{red}{Forskjell i metode som de forskjellige bruker -> forskjellige metoder for testing -> kvalitets attributtene blir kanskje ikke møtt}

Little research has been done on the research of software development and how its quality requirements are met in the public sector. By example in Sweden this has not been done as recently as 2020 \cite{mb_2020}, where Sweden's public sector is similar the Norwegian public sector when it comes to structure and purpose. 88\% of agencies in the Swedish public sector claim to use agile methodologies in their software development in order to meet its user-requirements. Of this 88\%, only 45\% agree/strongly agree that they use test automation, even though papers suggest that test automation and continuous testing is necessary is important to meet user-requirements in an agile development workflow \cite{vk_2010}.

The Norwegian government is pushing that the Norwegian public sector should move towards offering connected social-services, provided by a shared digital platform throughout the Norwegian public sector \cite{r_2019}. This includes the municipalities, counties and the Norwegian state to share data and increase co-operation by using digital tools \cite{r_2019}.

It would then be interesting to do similar research to Sweden in how the different agencies in the Norwegian public sector tests their software to assure its quality, and how it aligns with the Norwegian governments goal of co-operation trough a shared digital platform. This because the measurement of quality will determine if the strategy pushed by the Norwegian government will succeed or not, and if other nations public sectors should pursue a similar strategy in delivering benefits and social services to their citizens.

Papers such as \cite{ad_2013} are calling for more research to be done of how quality assurance in software is accomplished through testing. And how this has evolved since the publishing date in 2013 with the evolving of agile methodologies in Norwegian software organizations.

\textcolor{red}{Få inn at jo tidligere programvaren er bra kvalitetsjekket, jo mer sparer skattebetaleren penger, siden feil i programvare er dyrere å fikse når det er i produksjon}

\section{Project Goals and Research Questions}
\textcolor{red}{Skrive om denne til å passe det over}

The thesis aims to answer how different departments in the Norwegian public sector assure quality in its software. And how the software quality assurance methods differ from department to department. Interviews and questioneers is intended to be used to answer these questions. 

\begin{itemize}
    \item \textbf{RQ1:} How do the different Norwegian public sector's assure quality of their software?
    \item \textbf{RQ2:} How does the process of software quality assurance differ between departments in the Norwegian public sector?
\end{itemize}

\section{Report outline}

\textcolor{red}{Skrive om denne når hele er ferdig :)}