\chapter{Introduction}

\section{Motivation} \label{sec:motivation}
Software quality is how well a piece of software is measured against a chosen set of non-functional requirements, chosen based on the software's functional requirements \cite{iso_25010:2011}. These non-functional requirements are defined as when it can be determined that a piece of software delivers value \cite{iso_25010:2011}. Software quality assurance is the planning, controlling, and executing of processes which measure these non-functional requirements or other defined quality standards defined by a product team or at the organisational level \cite{ieee_730_2014}\cite{sqa_wiki_2023}. 

%Software quality assurance works to ensure that a software development team delivers a product that is of expectation to the end-users and meets the projects requirement specification.

%This increase in co-operation in the Norwegian public sector using digital tools. Thus leading to common quality attributes for the end-users of the different departments in the Norwegian public sector. \textcolor{red}{Forskjell i metode som de forskjellige bruker -> forskjellige metoder for testing -> kvalitets attributtene blir kanskje ikke møtt}

As of 2020 in the Swedish public sector, 88\% of agencies claim to use agile methodologies in their software development to assure quality \cite{mb_2020}. Out of this 88\%, only 45\% agree/strongly agree that they include the use of test automation to assure quality in its software, even though papers suggest that test automation and continuous testing are necessary is important to meet user requirements in an agile development workflow \cite{vk_2010}. Other than this, little research has been done on software quality assurance in the context of the Norwegian public sector, nor the public sector in general. 

The Norwegian government is pushing that the Norwegian public sector should move towards offering connected social services, provided by a shared digital platform throughout the Norwegian public sector \cite{r_2019}. This includes the municipalities, counties and the Norwegian state to share data and increase cooperation by using digital tools \cite{r_2019}. As the digital service in the Norwegian public sector moves closer together in responsibility and services, it can be speculated that they will have the same non-functional requirements which need to be measured to assure software quality. This has been described as being the case for similar systems, such as Sharing Economy systems. Having non-functional requirements elicited from requirements such as information security, privacy, and personal data protection due to their importance in creating trust \cite{is_2019}. These requirements are also described as important in public sector software for increasing trust among its users \cite{la_2017}, and trust among its users being of importance for a well-functioning Norwegian public sector \cite{oecd_2022}.

%As the digital services in the Norwegian public sector moves closer together in responsibility and services, it can be speculated that they will have the same non-functional requirements which needs to be measured to assure software quality. It can therefore be speculated that they will have the same processes for measuring these non-functional requirements. 

The researcher's role as a software developer in the Norwegian Labour and Welfare Administration (NAV) has given him perspectives on software quality assurance in the Norwegian public sector. Perspectives such as agility are important so that a piece of software can change with ease due to unforeseen events or user feedback. As well as organisational culture is important in software quality assurance, with a culture for psychological security where employees are able to learn from their mistakes and not be punished. However, the agency of NAV is large, handling about 1/3 of the Norwegian national budget \cite{okp_nav_r_2022}\cite{bud_r_2021} and employing about 22 000 \cite{org_nav_2023}. Which could mean that the researcher's views are not representative of all employed at NAV, or the Norwegian public sector at large.

%The researcher's role as a software developer in the Norwegian public sector at the IT department of the Norwegian Labour and Welfare Administration (NAV). This has led to the understanding that agility, the ability to change software due to unforeseen events such as bugs or user feedback, increases software quality. The researcher also believes from his experiences that organisation culture is the main force in assuring software quality, having a culture of knowledge and experience sharing, and being able to learn from mistakes, not punished. However the agency of NAV is large, handling about 1/3 of the Norwegian national budget \cite{okp_nav_r_2022}\cite{bud_r_2021} and having about 22 000 employees \cite{org_nav_2023}, which could mean that the researcher's views are not representative of all employed at NAV, or the Norwegian public sector at large.

%Little research has been done on the research of software development and how its quality requirements are met in the public sector. By example in Sweden this has not been done as recently as 2020 \cite{mb_2020}, where Sweden's public sector is similar the Norwegian public sector when it comes to structure and purpose. 88\% of agencies in the Swedish public sector claim to use agile methodologies in their software development in order to meet its user-requirements. Of this 88\%, only 45\% agree/strongly agree that they use test automation, even though papers suggest that test automation and continuous testing is necessary is important to meet user-requirements in an agile development workflow \cite{vk_2010}.

%It would then be interesting to do similar research to Sweden in how the different agencies in the Norwegian public sector tests their software to assure its quality, and how it aligns with the Norwegian governments goal of co-operation trough a shared digital platform. This because the measurement of quality will determine if the strategy pushed by the Norwegian government will succeed or not, and if other nations public sectors should pursue a similar strategy in delivering benefits and social services to their citizens.

%Papers such as \cite{ad_2013} are calling for more research to be done of how quality assurance in software is accomplished through testing. And how this has evolved since the publishing date in 2013 with the evolving of agile methodologies in Norwegian software organizations.


The objective of this study is to understand what the IT professionals with high influence or ownership over technology in certain Norwegian public sector agencies consider important in assuring quality in the software being used to serve their users. What challenges the agencies might be encountering in the quality assurance of their software. And how these findings compare with the Norwegian government's digitisation strategy. Leading to the following research questions:

\begin{itemize}
    \item \textbf{RQ1:} What practices do agencies of the Norwegian public sector use to assure quality in its software?
    \item \textbf{RQ2:} What challenges are agencies in the Norwegian public sector encountering in the quality assurance of its software?
\end{itemize}

%The objective of this study is to understand what the people with high influence or ownership over technology in the Norwegian public sector considers important in regards to assuring quality in its software and what factors might influence their view on this. And how these factors compare with the Norwegian government's digitisation strategy.

%The primary objective of this study is to understand what methods the different agencies in the Norwegian public sector use to assure quality in its software. This study also has the goal of how well the software quality assurance methods of the agencies in the Norwegian public sector compare to what is stated in relevant literature. How well it compares, meaning to what the literature states or recommends is good methods for assuring quality in software. 

\newpage

\section{Report outline}
The rest of the report is structured as follows.

\paragraph{\hyperref[sec:theoretical_background]{Chapter~\ref*{sec:theoretical_background}} - Theoretical Background} 
Introduces scientific literature relevant to this study.

\paragraph{\hyperref[sec:case_background]{Chapter~\ref*{sec:case_background}} - Case Background}
Introduces background knowledge needed for the case study of this report.

\paragraph{\hyperref[sec:method]{Chapter~\ref*{sec:method}} - Method}
Contains a description of the methodology used to obtain the results of the case study.

\paragraph{\hyperref[sec:results]{Chapter~\ref*{sec:results}} - Results}
Presents the results collected in the study.

\paragraph{\hyperref[sec:discussion]{Chapter~\ref*{sec:discussion}} - Discussion}
Contains a discussion of the findings of the study and the scientific literature in \autoref{sec:theoretical_background} in relation to the objective of this study.

\paragraph{\hyperref[sec:conclusion]{Chapter~\ref*{sec:conclusion}} - Conclusion}
Concludes the study and its findings.